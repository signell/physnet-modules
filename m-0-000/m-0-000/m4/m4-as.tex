\revhist{10/19/94, pss; 8/16/95, pss}

\Sect{}{}{\SectType{SpecialAssistance}}{

\AsItem{1}{TX-3c}
{Evaluate Eq.(4) with one more term, so you include \m{f'''}, substituting
 these values successively for \m{\Delta}: (\m{2\Delta}, \m{\Delta}, \m{-\Delta},
 and \m{-2\Delta}).
 This gives you four equations that you can properly substitute into
 the right side numerator of \Eqnref{9}.
 You should find the terms up through the second derivative canceling and
 just the proper quantity remaining.
}

\AsItem{2}{PS, problem~8a}
{\m{f'(x) = -2 x (x^2 + 4)^{-2}} hence \m{f'(1) = -2/25}.}

\AsItem{3}{PS, problem~8b}
{Students have asked: \Quote{What are \m{x} and \m{\Delta} when \m{x = 1}?}
 
 Answer: if \m{x = 1} and, say, \m{\Delta = 0.01}, then in equations such
 as \Eqnref{7}: \m{x + \Delta = 1.01} and \m{x - \Delta = 0.99}.
 You have to evaluate some particular problem's \m{f(x)} at some such values
 of \m{x}.
 The smaller the value of \m{\Delta} you use, the better your numerical
 approximation to the true value of \m{f'(1)}, but see Sect.\,3e in the
 \textit{text}.
}

}% /Sect
