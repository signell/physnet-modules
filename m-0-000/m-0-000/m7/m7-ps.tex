\revhist{9/7/90, pss; 7/7/91, pss; 9/27/93, pss; 10/4/94, pss; 11/7/95, pss;
         11/7/97, pss; 2/28/00, pss; 12/11/00, pss; 4/16/02, pss; 10/10/02, pss}

\Sect{}{}{\SectType{ProblemSet}}{

\noindent Note: Problems~14-17 are also on this module's \textit{Model Exam}.

\begin{two-digit-list}
\item [1.] A particle moving along a straight line has the following positions
at the indicated times:

\begin{tabular}{|c c c c c c c c c c|}\hline
\m{t}(in \unit{s}) & 0.0 & 0.1 & 0.2 & 0.3 & 0.4 & 0.5 & 0.6 & 0.7 & 0.8 \\
\m{x}(in \unit{m}) & 5.2 & 5.5 & 5.9 & 6.4 & 7.0 & 7.7 & 8.5 & 9.4 & 10.4 \\ \hline
\end{tabular}

\begin{one-digit-list}
\item [a.] Use the table to determine the average velocity:
\begin{itemize}
\item [i.] for the interval \m{t = 0\unit{s}} to \m{t = 0.5\unit{s}},
\item [ii.] for the interval \m{t = 0.5\unit{s}} to \m{t = 0.8\unit{s}},
\item [iii.] for the interval \m{t = 0\unit{s}} to \m{t = 0.8\unit{s}}.
\end{itemize}
\item [b.] Determine the approximate instantaneous velocity from the \m{x} vs
\m{t} curve:
\begin{itemize}
\item [i.] at \m{t = 0.4\unit{s}},
\item [ii.] at \m{t = 0.5\unit{s}}.
\end{itemize}
\item [c.] Does the instantaneous velocity become equal to the average velocity
at the midpoint of displacement or the midpoint of time?
Why?
\item [d.] Indicate how to determine the above velocities on a position-time
graph.
\end{one-digit-list}

\item [2.] \CenteredUnframedFixedFigure{m7gr13}{a. A graph of \m{x}\,vs\,\m{t} for a particle in
straight-line motion is shown in the sketch.
For each interval, indicate (above the curve) whether the average velocity
\m{v_{av}} is \m{+}, \m{-}, or 0, and (below the curve) whether the acceleration
\m{a_x} is \m{+}, \m{-}, or 0.}

\begin{one-digit-list}
\item [b.] Locate all points on the graph where the instantaneous velocity is
zero.
\end{one-digit-list}

\item [3.] The position of an object moving in a straight line is given by
\m{x = A + B t + C t^2}, where \m{A = 1.0\unit{m}}, \m{B = 2.0\unit{m/s}} and \m{C = -3.0\unit{m/s\up{2}}}.
\begin{one-digit-list}
\item [a.] What is its average velocity for the interval from \m{t = 0} to
\m{t = 2.0\unit{s}}?
\item [b.] What are its (instantaneous) velocities at \m{t = 0} and \m{t = 2.0\unit{s}}?
\item [c.] What is its acceleration at each of these times?
\end{one-digit-list}

\item [4.] A rocket is fired vertically, and ascends with a constant vertical
acceleration of \m{+20.0\unit{m/s\up{2}}} for \m{80.0\unit{s}}.
Its fuel is then all used and it continues with an acceleration
\m{g = - 9.8\unit{m/s\up{2}}}.
Air resistance can be neglected.
\begin{one-digit-list}
\item [a.] What is its altitude \m{80.0\unit{s}} after launching?
\item [b.] How long does it take to reach its maximum altitude?
\item [c.] What is this maximum altitude?
\end{one-digit-list}

\item [5.] A particle moves along the \m{x}-axis with acceleration
\m{a(t) = A + Bt^2}, starting from rest at \m{x = 5.0\unit{m}} and \m{t = 0}.
Find its position \m{x(t)}.

\item [6.] You have leveled an air track and then placed a block under one end
of the track.
Using photocell gates and a timer, you find the length of time \m{t} it takes a
glider on the track to move some convenient distance \m{x - x_0}.
Determine the acceleration \m{a_x} of the glider from these data:
\m{x - x_0 = 100.0\unit{cm}}, \m{t = 4.053\unit{s}}.

\item [7.] Water drips from a shower nozzle onto the floor \m{72\unit{inches}} below.
Neglect air resistance.
\begin{one-digit-list}
\item [a.] How fast are the drops falling when they strike the floor?
\item [b.] How long does it take a drop to fall?
\end{one-digit-list}

\item [8.] A lifeguard is standing on the edge of a swimming pool when she drops
her whistle.
The whistle falls \m{4.0\unit{ft}} from her hand to the water.
It then sinks to the bottom of the pool at the same constant velocity with
which it struck the water.
It takes a total of \m{1.0\unit{s}} to go from hand to bottom.
\begin{one-digit-list}
\item [a.] How long was it falling through the air?
\item [b.] How long was it falling through the water?
\item [c.] With what velocity did it strike the water?
\item [d.] How deep was the pool?
\end{one-digit-list}

\item [9.] A truck traveling at \m{60.0\unit{mph}} (\m{88\unit{ft/s}}) passes a car pulling out
of a gas station.
The driver of the car instantaneously steps on the gas and accelerates at
\m{8.0\unit{ft/s\up{2}}} and catches the truck in \m{0.200\unit{mi}} (\m{1056\unit{ft}}).
How fast was the car traveling when the truck passed it and how long did it
take to catch the truck?

\item [10.] In a certain amusement park, a bell will ring when struck from
below by a weight traveling upward at \m{10.0\unit{ft/s}}.
How fast must a weight be projected upward to ring a bell which is 36\unit{feet}
above the ground?
How long does it take to hit the bell?

\item [11.] Suppose that after many years of patient waiting, a radar tracking
station was able to track an unidentified flying object (UFO).
Initially the UFO was at rest, but as soon as it was sighted it started to
move away from the station in a straight line.
Its speed along this line was measured to be \m{v = \alpha t - \beta t^3}
where \m{\alpha = 300\unit{mi/s\up{2}}} and \m{\beta = 0.75\unit{mi/s\up{4}}} during the time it
was observed, until it disappeared \m{20\unit{s}} after first sighting.
\begin{one-digit-list}
\item [a.] How fast was the UFO going when it disappeared?
\item [b.] What was its acceleration when it first started to move?
\item [c.] How far did the UFO go during the \m{20\unit{s}}?
\end{one-digit-list}

\item [12.] A particular lightning flash is seen \m{5.0\unit{s}} before the thunder is
heard.
How far away is the thunderstorm?

\item [13.] A cyclist accidentally drops a padlock off the side of a high
bridge.
One second later he disgustedly throws the key downwards at \m{12\unit{m/s}} after it.
Does the key overtake the padlock?
If so, when and where?

\item [14.] The position of a particle is given by: \m{x =
A - B t + D t^3 - E t^4}.
\begin{one-digit-list}
\item [a.] Find the velocity.
\item [b.] Find the acceleration.
\item [c.] Find average velocity for the interval \m{t = 0} to \m{t = 3\unit{s}}.
\end{one-digit-list}

\item [15.] A physics professor at the football stadium drives two miles home at
30\unit{mph} to get her football tickets, discovers them in her purse, and
immediately drives back at \m{20\unit{mph}} because the traffic is worse.
What was her average velocity for the round trip?

\item [16.] A salesman brags that a car will accelerate from \m{10\unit{mi/hr}}
(\m{4.47\unit{m/s}}) to \m{75\unit{mi/hr}} (\m{33.5\unit{m/s}}) in \m{12\unit{s}}.
\begin{one-digit-list}
\item [a.] Find the average acceleration in \unit{m/s\up{2}} during this time interval
(do not assume constant acceleration).
\item [b.] Assuming constant acceleration, find the distance and time at which
the car would attain the speed of \m{55\unit{mi/hr}} (\m{24.6\unit{m/s}}),
starting from \m{10\unit{mi/hr}}.
\end{one-digit-list}

\item [17.] a. A graph of \m{x} vs \m{t} for a particle in straight line motion is
shown in the sketch.

\CenteredUnframedFixedFigure{m7gr14}

For each interval between the hash marks:
\begin{itemize}
\item [i.] mark, above the curve, whether the average velocity \m{v_{av}} is \m{+},
\m{-}, or 0; and,
\item [ii.] mark, below the curve, whether the acceleration \m{a} is \m{+}, \m{-}, or
0.
\end{itemize}
\begin{one-digit-list}
\item [b.] Identify all points on the graph where the instantaneous velocity is
zero.
\end{one-digit-list}
\end{two-digit-list}

\BriefAns

\begin{two-digit-list}
\item [1.] \NullItem
\begin{one-digit-list}
\item [a.] \m{v_{av(0-5)} = \dfrac{x_5 - x_0}{t_5 - t_0} =
            \dfrac{7.7\unit{m} - 5.2\unit{m}}{0.5\unit{s}} =
            \dfrac{2.5\unit{m}}{0.5\unit{s}} = 5.0}\unit{m/s}
\item [  ] \m{v_{av(5-8)} = \dfrac{x_8 - x_5}{t_8 - t_5} =
           \dfrac{10.4\unit{m} - 7.7\unit{m}}{0.8\unit{s} - 0.5\unit{s}} =
           \dfrac{2.7\unit{m}}{0.3\unit{s}} = 9.0\unit{m/s}}.
\item [  ] \m{v_{av(0-8)} = \dfrac{x_8 - x_0}{t_8 - t_0} =
           \dfrac{10.4\unit{m}-5.2\unit{m}}{0.8\unit{s}} =
           \dfrac{5.2\unit{m}}{0.8\unit{s}} = 6.5}\unit{m/s}
\item [b.] \m{v_4 = \dfrac{x_5-x_3}{t_5-t_3} =
           \dfrac{7.7\unit{m} -6.4\unit{m}}{0.5\unit{s}-0.3\unit{s}} =
           \dfrac{1.3\unit{m}}{0.2\unit{s}} = 6.5}\unit{m/s}
\item [  ] \m{v_5 = \dfrac{x_6-x_4}{t_6-t_4} =
           \dfrac{8.5\unit{m}-7.0\unit{m}}{0.6\unit{s}-0.4\unit{s}}=
           \dfrac{1.5\unit{m}}{0.2\unit{s}} = 7.5\unit{m/s}}
\item [  ] Note: You might have chosen different time intervals.
\item [c.] Inspection of the table shows that the instantaneous velocity, as
           indicated by the increases in displacement in each time interval, is
           increasing uniformly, indicating that the acceleration is constant.
           Combining \m{v = v_0 + a t} and \m{v_{av} = (v + v_0)/2}, we get
           \m{v_{av} = v_0 + (a\,t/2)} which shows that the average velocity occurs at
           time \m{t}/2, as shown by the calculations above.
\item [d.] See the sketches in the statements of Problems 2 and 17 and in the answer for Problem 2.
\end{one-digit-list}

\item [2.] \NullItem
\begin{one-digit-list}
\item [a.] See the sketch.
           \CenteredUnframedFixedFigure{m7gr15}
\item [b.] Highest point and lowest segment of the curve.
\end{one-digit-list}

\item [3.] \NullItem
\begin{one-digit-list}
\item [a.] \m{v_{av} = \dfrac{\Delta s}{\Delta t} =
           \dfrac{(A + B t_2 + C t_2^2) - (A + B t_1 + C t_1^2)}{t_2 - t_1}}
\item [  ] At \m{t_1} = 0, \m{t_2} = 2\unit{s}:
\item [  ] \m{v_{av} = \dfrac{(A + B t_2 + C t_2^2) - (A)}{t_2} =
           \dfrac{t_2 ( B + C t_2)}{t_2} = B + C t_2}
\item [  ] \m{\phantom{v_{av}} = (2.0\unit{m/s}) + (-3.0\unit{m/s\up{2}})(2.0\unit{s}) =
           2.0\unit{m/s} - 6.0\unit{m/s}}
\item [  ] \m{\phantom{v_{av}} = - 4.0\unit{m/s}}.
\end{one-digit-list}
\CaptionedFullFramedFixedFigure{13}{The triangles show how to calculate the average velocities for the
                               intervals \m{t_0 - t_5} and \m{t_5 - t_8}.}{m7gr16}
\vspace*{-2pc}
\enlargethispage*{1pc}
\CaptionedFullFramedFixedFigure{14}{The triangles show that the average velocity for the interval
                                     \m{t_0 - t_8} equals the instantaneous velocity at \m{t_4}.}{m7gr17}

\CaptionedFullFramedFixedFigure{15}{The triangles show that the instantaneous velocity at \m{t_5}
                                     (the approximate midpoint of displacement) does not equal the
                                     average velocity for the interval \m{t_0 - t_8}.}{m7gr18}
\begin{one-digit-list}
\item [b.] \m{v(t)  = \dfrac{dx(t)}{dt} =
           \dfrac{d}{dt}(A + B t + C t^2) = B + 2 C t}
\item [  ] \m{v(0) = B = 2.0\unit{m/s}}
\item [  ] \m{v(2.0\unit{s}) = 2.0\unit{m/s} + 2(-3.0\unit{m/s\up{2}})(2.0\unit{s}) =
            - 10\unit{m/s}}.
\item [c.] \m{a(t) = \dfrac{dv(t)}{dt} =
           \dfrac{d}{dt}(B + 2 C t) = 2 C = -6.0\unit{m/s\up{2}}} at \m{t_1} and \m{t_2}.
\end{one-digit-list}

\newpage

\item [4.] Given \m{v_0 = 0}, \m{a = + 20\unit{m/s\up{2}}}, \m{t = 80\unit{s}},
           \m{g = - 10\unit{m/s\up{2}}},
\begin{one-digit-list}
\item [a.] \m{x_1 = v_0 t + \dfrac{1}{2} a t^2}
\item [  ] \m{\phantom{x_1} = 0 + \dfrac{1}{2}(20\unit{m/s\up{2}})(80\unit{s})^2 =
           6.4 \times 10^4\unit{m}}
\item [  ] \m{v = v_0 + at = 0 + (20\unit{m/s\up{2}})(80\unit{s}) = 1600\unit{m/s}}.
\item [  ] The rocket continues upward until it stops;
\item [  ] \m{t = \dfrac{v - v_0}{g} =
           \dfrac{0-1600\unit{m/s}}{-10\unit{m/s\up{2}}} = 160\unit{s}}.
\item [b.] \m{\text{Total time to rise } = 80\unit{s} + 160\unit{s} = 2.4 \times 10^2\unit{s}}.
\item [  ] Distance upward after burnout:
\item [  ] \m{x = v_b t + \dfrac{1}{2} g t^2}
\item [  ] \m{\phantom{x} = (1600\unit{m/s})(160\unit{s}) +
           \dfrac{1}{2}(- 10\unit{m/s\up{2}})(160\unit{s})^2 = 128,000\unit{m}}.
\item [  ] Alternatively,
\item [  ] \m{v^2 - v_0^2 = 2\,a\,x}
\item [  ] \m{x = \dfrac{-v_0^2}{2a} =
           \dfrac{-(1600\unit{m/s})^2}{2(-10\unit{m/s\up{2}})} = 128,000\unit{m}}.
\item [c.] \m{\text{Maximum altitude } = 64,000\unit{m} + 128,000\unit{m} = 1.92 \times 10^5\unit{m}}
\end{one-digit-list}

\item [5.] Since \m{a_x = \dfrac{dv_x}{dt}}, you can write
\item [  ] \m{v_x = \int dv_x = \int a_x dt = \int (A + Bt^2) dt = A\int dt + B \int t^2 dt}
\item [  ] \m{\phantom{v_x} = A t + \dfrac{1}{3} B t^3 + C}.
\item [  ] Set \m{t} = 0 to obtain 0 = \m{v_x(0) = C}.
\item [  ] Next,
\item [  ] \m{x = \int dx = \int v_x dt = \int (A t + \dfrac{1}{3} B t^3) dt =
           \dfrac{1}{2} A t^2 + \dfrac{1}{12} B t^4 + D}
\item [  ] or:
\item [  ] \m{x(t) = \dfrac{1}{2} A t^2 + \dfrac{1}{12} B t^4 + D}.
\item [  ] This time, the initial condition tells us that \m{D = 5.0\unit{m}}; so the final
           expression is
\item [  ] \m{x(t) = \dfrac{1}{2} A t^2 + \dfrac{1}{12} B t^4 + 5.0\unit{m}}.

\item [6.] \m{v_{av} = \dfrac{x - x_0}{t} =
           \dfrac{100.0\unit{cm}}{4.053\unit{s}} = 24.67\unit{cm/s}}.
\item [  ] \m{x - x_0 = v_0 t + \dfrac{1}{2} a t^2}
\item [  ] \m{a = \left[(x - x_0) - v_0 t\right] 2/t^2}.
\item [  ] If we set \m{v_0} = 0 at \m{t} = 0,
\item [  ] \m{a = \dfrac{2(x - x_0)}{t^2} =
           \dfrac{2 (100.0\unit{cm})}{(4.053\unit{s})^2} =
           12.18\unit{cm/s\up{2}}}.
\item [  ] If \m{v_0 > 0} at \m{t} = 0, \m{a < 12.18\unit{cm/s\up{2}}}.

\item [7.] Since this problem is one-dimensional, it is convenient to take the direction for the positively-increasing
\m{x}-axis as downward.
Then \m{x} = 72\unit{in} = 6\unit{ft}, \m{v_0} = 0 at \m{t} = 0, \m{a = g = 32\unit{ft/s\up{2}}},
\begin{one-digit-list}
\item [a.] \m{v^2 - v_0^2 = 2\,a\,x}
\item [  ] \m{v = (2\,a\,x)^{1/2} = \left[(2)(32\unit{ft/s\up{2}})(6\unit{ft})\right]^{1/2}}
\item [  ] \m{\phantom{v} = 20\unit{ft/s}}.
\item [b.] \m{v - v_0 = a t}
\item [  ] \m{t = \dfrac{v}{a} = \dfrac{20\unit{ft/s}}{32\unit{ft/s\up{2}}} =
           0.62\unit{s}}.
\item [  ] Alternatively,
\item [  ] \m{x = v_0 t + \dfrac{1}{2} a t^2}
\item [  ] \m{t = (2x/a)^{1/2} =
           \left[ \dfrac{(2)(6\unit{ft})}{32\unit{ft/s\up{2}}}\right]^{1/2} =
           0.615\unit{s}}.
The difference in time results from rounding error.
\end{one-digit-list}

\item [8.] \NullItem
\begin{one-digit-list}
\item [a.] \m{x(t) = x(0) + v(0) t + \dfrac{1}{2} a t^2}.
\item [  ] We orient the \m{x}-axis to increase positively downward so \m{a = g}.
           We put \m{t = 0} at the instant of drop so \m{v(0) = 0} and we put
           the origin at the hand so \m{x(0) = 0}.
           Let \m{d_a \equiv} distance through air; by (1) it is:
\item [  ] \m{d_a = \dfrac{1}{2} g t_a^2},
\item [  ] where \m{t_a \equiv \text{ time through air}}.
\item [  ] \m{t_a = (2d_a/g)^{1/2} =
           \left[2(4.0\unit{ft})/(32\unit{ft/s\up{2}})\right]^{1/2}}
\item [  ] \m{\phantom{t_a} = \left(\dfrac{1}{4}\unit{s\up{2}}\right)^{1/2} =
           \dfrac{1}{2}\unit{s}}.
\item [b.] Let \m{t_w \equiv} time in water, \m{t_t} = total time from hand to
           bottom \m{ = t_w + t_a}
\item [  ] \m{t_w = t_t - t_a = 1.0\unit{s} - \dfrac{1}{2}\unit{s} = \dfrac{1}{2}\unit{s}}.
\item [c.] Velocity at water \m{\equiv v_w = v(t_a) = g t_a =
           (32\unit{ft/s\up{2}})\left(\dfrac{1}{2}\unit{s}\right) = 16\unit{ft/s}}.
\item [d.] \m{\text{Let } d_w \equiv \text{ distance in water } = v_w t_w
            = (16\unit{ft/s})\left(\dfrac{1}{2}\unit{s}\right) = 8\unit{ft}}.
\end{one-digit-list}

\item [9.] \m{\text{Time } = \dfrac{\text{distance traveled}}{\text{velocity of truck}} =
           \dfrac{1056\unit{ft}}{88\unit{ft/s}} = 12\unit{s}}.
\item [  ] For the car, \m{x = v_0 t + \dfrac{1}{2} a t^2},
\item [  ] \m{v_0 = (x-\dfrac{1}{2}at^2)/t}
\item [  ] \m{\phantom{v_0} = x/t - \dfrac{1}{2} a t =
           \dfrac{1056\unit{ft}}{12\unit{s}} -
           \dfrac{1}{2}\left(8\unit{ft/s\up{2}}\right)(12\unit{s}) = 40\unit{ft/s}}.
\item [10.] Take \m{x = 36\unit{ft}}, \m{v = 10\unit{ft/s}}, \m{a = g = - 32\unit{ft/s\up{2}}}.
\item [  ] \m{v^2 - v_0^2 = 2 a x}
\item [  ] \m{v_0 = (v^2 - 2 a x)^{1/2} =
           \left[(10\unit{ft/s})^2 - 2(-32\unit{ft/s\up{2}})(36\unit{ft})\right]^{1/2}},
\item [  ] \m{\phantom{v_0} = (2400\unit{ft\up{2}}/\unit{s\up{2}})^{1/2} = 49\unit{ft/s}}
\item [  ] \m{t = \dfrac{v - v_0}{a} =
           \dfrac{10\unit{ft/s} - 49\unit{ft/s}}{-32\unit{ft/s\up{2}}} =
           1.22\unit{s}}.
\item [  ] Checking:
\item [  ] \m{x = v_0 t + \dfrac{1}{2} a t^2}
\item [  ] \m{\phantom{x} =
           (49\unit{ft/s})(1.22\unit{s}) +
           \dfrac{1}{2}(-32\unit{ft/s\up{2}})(1.22\unit{s})^2 = 36.0\unit{ft}}.
\item [  ] It is often convenient to carry an extra significant figure in calculations.

\item [11.] \m{v(t) = \alpha t - \beta t^3}; \m{\alpha = 300\unit{mi/s\up{2}}},
            \m{\beta = \dfrac{3}{4}\unit{mi/s\up{4}}}.
\begin{one-digit-list}
\item [a.] \m{v(20\unit{s}) = (300\unit{mi/s\up{2}})(20\unit{s}) -
           (\dfrac{3}{4}\unit{mi/s\up{4}})(20\unit{s})^3}
\item [  ] \m{\phantom{v(20\unit{s})} =
           6.0 \times 10^3\unit{mi/s} - 6.0 \times 10^3\unit{mi/s} = 0.}
\item [b.] \m{a(t) = \dfrac{dv(t)}{dt} = \alpha - 3 \beta t^2}.
\item [  ] When the object \Quote{first started to move} probably means \m{t = 0} since that is
           the first time when \m{v = 0}.
           The acceleration at that time was:
\item [  ] \m{a(0) = \alpha = 300\unit{mi/s\up{2}}}.
\item [c.] \m{x(t) = \int (\alpha t - \beta t^3) dt =
           \dfrac{\alpha t^2}{2} - \dfrac{\beta t^4}{4} + C}.
           \m{x(0) = C}, hence
           \m{x(t) - x(0) = \dfrac{\alpha t^2}{2} - \dfrac{\beta t^4}{4}}.
           Now let \m{d(t) \equiv} distance traveled since \m{t = 0}, which is also the
           distance traveled since \m{v = 0}.
           Then:
\item [  ] \m{d(t) = x(t) - x(0) =
           \dfrac{\alpha t^2}{2} - \dfrac{\beta t^4}{4}},
\item [  ] \m{d(20\unit{s}) = \dfrac{(300\unit{mi/s\up{2}})(20\unit{s})^2}{2} -
           \dfrac{\left( \dfrac{3}{4}\unit{mi/s\up{4}}\right)(20\unit{s})^4}{4}}
\item [  ] \m{\phantom{d(20\unit{s})} = 6.0 \times 10^4\unit{mi} -
           3.0 \times 10^4\unit{mi} = 3.0 \times 10^4\unit{mi}}.
\end{one-digit-list}

\item [12.] Velocity of light \m{= 3.0 \times 10^8\unit{m/s}}.
\item [   ] Velocity of sound \m{= 3.3 \times 10^2\unit{m/s}}.
\item [   ] We may neglect the time it takes for light to reach us.
\item [   ] \m{x = v t = (3.3 \times 10^2\unit{m/s})(5.0\unit{s}) = 1.7 \times 10^3\unit{m}}.

\item [13.] \m{x(t) = x(0) + v(0) t + a t^2/2}.
\item [   ] We orient \m{x(t)} downward, so \m{a = g = 9.8\unit{m/s\up{2}}}.
\item [   ] For the padlock, \m{x = g t^2/2}.
\item [   ] For the key, \m{v_0 = 12\unit{m/s}},
            \m{x = v_0 (t - 1.0\unit{s}) + \dfrac{1}{2} g(t - 1.0\unit{s})^2}.
\item [   ] Solving simultaneously,
           \m{\dfrac{1}{2} g t^2 =
           v_0(t - 1.0\unit{s}) + \dfrac{1}{2} g (t - 1.0\unit{s})^2}
\item [   ] When: \m{t = 3.2\unit{s}} after dropping the padlock.
\item [   ] Where: \m{x = 51\unit{m}}.

\item [14.] \NullItem
\begin{one-digit-list}
\item [a.] \m{v = - B + 3 D t^2 - 4 E t^3}.
\item [b.] \m{a = 6 D t - 12 E t^2}.
\item [c.] \m{v_{av} = - B + 9 s^2 D - 27 s^3 E}.
\end{one-digit-list}

\item [15.] Average Velocity
            \m{ = v_{av} = \dfrac{x(t_{final}) - x(t_{initial})}{t_{final} - t_{initial}} = 0}.
            Note tht the average \emph{speed} is not zero.

\item [16.] \NullItem
\begin{one-digit-list}
\item [a.] \m{a = 2.42\unit{m/s\up{2}}}, \m{x = 228\unit{m}}.
\item [b.] \m{x = 121\unit{m}}, \m{t = 8.32\unit{s}}.
\end{one-digit-list}

\item [17.] \CenteredUnframedFixedFigure{m7gr19}
\end{two-digit-list}

}% /Sect