\tcsc {1. Introduction}{}
\tcpc {a. Understanding Circular Motion is Important}{1}
\tcpc {b. A Special Case of General Motion}{1}
\tcpc {c. Prerequisites}{1}
\tcsc {2. Quantitative Kinematics}{}
\tcpc {a. The General Equations That We Particularize}{2}
\tcpc {b. The Radius Vector: Path and Cartesian Coordinates}{3}
\tcpc {c. The Radius Vector in Polar Coordinates}{3}
\tcpc {d. Velocity and Speed}{4}
\tcpc {e. Radial}{4}
\tcsc {3. Uniform Circular Motion}{}
\tcpc {a. Constant Speed}{5}
\tcpc {b. The Acceleration Direction is Radially Inward}{6}
\tcsc {4. Parameters of the Motion}{}
\tcpc {a. The Example We Will Use}{6}
\tcpc {b. Cycle}{6}
\tcpc {c. Period}{6}
\tcpc {d. Frequency}{7}
\tcpc {e. Angular Speed}{7}
\tcpc {f. Angular Velocity}{8}
\tcsc {5. Why Forces are Not Considered}{}
\tcpc {a. The Problem}{9}
\tcpc {b. The Answer}{10}
\tcsc {Acknowledgments}{10}
\tcsc {A. Derivations: for Those Interested}{10}
\tcsc {}{}
\tcsc {}{}
\tcsc {}{}
\defmodlength {28}
