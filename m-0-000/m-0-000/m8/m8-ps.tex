\revhist{7/19/85, pss; 9/13/88, pss; 7/31/91, pss; 3/12/93, pss; 6/8/94, pss;
         10/24/94, pss; 3/3/95, pss; 3/31/95, pss; 7/10/95, abs; 8/27/95, pss;
         9/18/96, pss; 3/21/97, pss; 11/13/97, pss; 3/2/00, pss; 2/8/02, pss}

\Sect{}{}{\SectType{ProblemSet}}{

\BlackTriangle If you get really stuck on a problem, turn to the appropriate
hint in this module's \textit{Special Assistance Supplement} and then try to
continue with the solution.
For example, problem~1a, below, has a hint in the box labeled [S-1] in
the \textit{Special Assistance Supplement}.
If you still can't solve problem~1a after using the hint, the [S-1] box
contains a reference to another box containing a further hint.

\BlackTriangle Problems~11, 13-15 also occur in this module's \textit{Model Exam}.

\BlackTriangle Look out! Units and \textit{variables} are set in
different typefaces. \help{14}

\begin{two-digit-list}
\item [1.]  The position of a particle is given by the expression:
\Eqn{}{\vect{r} = (2\unit{m/s\up{2}}) t^2 \uvec{x} + (3\unit{m/s}) t \uvec{y}\,.}
\begin{one-digit-list}
\item [a.] What is the shape of the path? \help{1}
\item [b.] What is the velocity at the point \m{x = 2\unit{m}}, \m{y = 3\unit{m}}? \help{2}
\end{one-digit-list}

\item [2.]  Which of the following time-dependent positions are
consistent with a constant but non-zero acceleration? \help{3}
\begin{one-digit-list}
\item [a.] \m{(3\unit{m/s\up{3}}) t^3 \uvec{x} + (1\unit{m/s\up{2}}) t^2 \uvec{y}}
\item [b.] \m{(2\unit{m/s}) t \uvec{x} + (1\unit{m}) \uvec{y}}
\item [c.] \m{(2\unit{m/s\up{2}}) t^2 \uvec{x} + (1\unit{m}) \uvec{y}}
\item [d.] \m{(5\unit{m/s\up{2}}) t^2 \uvec{y}}
\item [e.] \m{(4\unit{m}) \uvec{x} + (3\unit{m}) \uvec{y}}
\end{one-digit-list}

\item [3.]  Which of the above (a - e, Prob.\,2) correspond to constant but
non-zero velocity? \help{4}

\item [4.]  Which of the above (a - e, Prob.\,2) correspond to an object at
rest? \help{5}

\item [5.]  A particle has the following information given about its motion:
\begin{itemize}
\item [\m{x}-dir.:] A constant velocity of 5\unit{m/s}, with initial condition
\m{x_0 = 0}\,m.
\item [\m{y}-dir.:] A constant acceleration of \m{- 2\unit{m/s\up{2}}}, with initial
conditions \m{y_0 = 10\unit{m}}, \m{v_{0y} = - 3\unit{m/s}}.
\end{itemize}
\begin{one-digit-list}
\item [a.] Recognizing that these motions fit the special cases of constant
velocity and constant acceleration, write the equations \m{x(t)}, \m{v_x(t)},
\m{a_x(t)}, \m{y(t)}, \m{v_y(t)} and \m{a_y(t)}. \help{6}
\item [b.] Write the vector equations for \m{\vect{r}(t)}, \m{\vect{v}(t)} and
\m{\vect{a}(t)}. \help{7}
\end{one-digit-list}

\item [6.]  A particle is at rest at \m{t = 0}.
After \m{t = 0}, the acceleration of the particle is given by
\m{\vect{a} = (3\unit{m/s\up{2}}) \uvec{x} - (2\unit{m/s\up{2}}) \uvec{y}}.
What is the shape of the trajectory? \help{8}

\item [7.]  The \m{x} and \m{y} components of velocity of a particle
are given below.
\CenteredUnframedFixedFigure{m8gr05}
\begin{one-digit-list}
\item [a.] What is the average velocity over the time interval
\m{0 \leq t \leq 40\unit{s}}? \help{9}
\item [b.] What is the acceleration at \m{t = 20\unit{s}}? \help{10}
\item [c.] What is the average acceleration over the interval
\m{20\unit{s} \leq t \leq 60\unit{s}}? \help{11}
\end{one-digit-list}

\item [8.]  A parachutist jumps from an airplane at a height of 300\unit{m}.
The parachute immediately opens and she descends at a constant
rate of 8\unit{m/s}.
As she descends, a steady wind of 3\unit{m/s} is blowing toward the south.
How far south of the point where she left the airplane will the parachutist
strike the earth? \help{12}

\item [9.]  A particle starts from rest and moves with constant acceleration
\m{\vect{a} = A \uvec{x} + B \uvec{y}}, where \m{A} and \m{B} are constants.
Show that the average velocity over the time interval 0 to \m{t}
is half the instantaneous velocity at \m{t}. \help{13}

\item [10.] A boy stands on an inclined surface which makes an angle
of {30\degrees} with the horizontal (see sketch).
He throws a ball so that it leaves his hand horizontally, down near the
surface of the incline, with a speed of 10\unit{m/s}.
At what distance down the incline (measured from the boy) will the ball
strike the incline?
(Assume he releases the ball from a point at the surface and assume that
the only acceleration is that of gravity.)

\CenteredUnframedFixedFigure{m8gr07}

\item [11.] The velocity of a particle is given by the expression:
%
\Eqn{}{\vect{v} = [2 (t/\unit{s}) \uvec{x} + \uvec{y}]\unit{m/s}\,.}
%
\begin{one-digit-list}
\item [a.] At \m{t = 0} the particle is at \m{x = 0}, \m{y = 0}.
What is the position of the particle at a later time \m{t}?
\item [b.] What is the average velocity over the interval \m{t = 2\unit{s}} to
\m{t = 3\unit{s}}?
\end{one-digit-list}

\item [12.] A particle moves with position vector
%
\Eqn{}{\vect{r} = \left[\left(\dfrac{t}{\unit{s}}\right) \uvec{x} +
                       \left(\dfrac{4}{3} \dfrac{t^3}{\unit{s\up{3}}} -
                             4\dfrac{t}{\unit{s}}\right)
                       \uvec{y} \right]
                 \unit{m}\,.}
%
\begin{one-digit-list}
\item [a.] Plot the path on the graph below.
Hint: First, fill in the table.\help{15}
\renewcommand{\arraystretch}{1.2}%
 \begin{tabular}[t]{|r r r|}\hline
   t(s) & x(m) & y(m) \\ \hline
   \m{-2.1} & \m{-2.1} & \m{-3.9} \\ \hline
   \m{-2.0} &        &        \\ \hline
   \m{-1.0} &        &        \\ \hline
   \m{ 0.0} &        &        \\ \hline
   \m{ 1.0} &        &        \\ \hline
   \m{ 2.0} &        &        \\ \hline
   \m{ 2.1} &        &        \\ \hline
 \end{tabular}
 
\CenteredUnframedFixedFigure{m8gr08}
 \item [b.] On the same curve, plot the velocity at \m{t = 1\unit{s}}.
 The scale for plotting the velocity can be any you choose.
 A convenient scale for velocity is 1 unit of velocity = 1/2\,unit of length.
 \help{16}%
 \item [c.] On the same curve, plot the velocity at \m{t = - 2\unit{s}}. \help{17}
 \item [d.] On the same curve, plot the acceleration at \m{t = - 1\unit{s}}. \help{18}
 \end{one-digit-list}
%}%
%{\renewcommand{\arraystretch}{1.2}
% \begin{tabular}[b]{|r r r|}\hline
%   t(s) & x(m) & y(m) \\ \hline
%   \m{-2.1} & \m{-2.1} & \m{-3.9} \\ \hline
%   \m{-2.0} &        &        \\ \hline
%   \m{-1.0} &        &        \\ \hline
%   \m{ 0.0} &        &        \\ \hline
%   \m{ 1.0} &        &        \\ \hline
%   \m{ 2.0} &        &        \\ \hline
%   \m{ 2.1} &        &        \\ \hline
% \end{tabular}
%}
%

\item [13.] \ItemFigure{A block is projected up an inclined surface which
makes an angle \m{\theta} with the horizontal.
The initial speed is \m{v_0} and the inclined surface is frictionless.
Use a coordinate system defined by \m{x} positive to the right
in a horizontal direction, \m{y} positive up (as illustrated).}{m8gr09}
\begin{one-digit-list}
\item [a.] What is the initial velocity?
\item [b.] The acceleration of the block is observed to be:
%
\Eqn{}{\vect{a} = - g \sin\theta\,\cos\theta\,\uvec{x} - g \sin^2\theta\,\uvec{y}.}
%
Develop expressions for \m{x(t)} and \m{y(t)}.
Divide one by the other to check that they (properly) predict that y(x) is a
straight-line function with slope \m{\tan\theta}.
\end{one-digit-list}

\item [14.] A particle moves with acceleration:
%
\Eqn{}{\vect{a} = \left[- 6\,(t/\unit{s})^2 \uvec{x} + (t/\unit{s}) \uvec{y}\right]\unit{m/s\up{2}}\,.}
%
At \m{t = 0} the particle is at rest at the origin.
What is the equation of the path (i.e., the trajectory)?

\item [15.] A football is thrown by a quarterback at a speed of
20\unit{m/s} at an angle of {45\degrees} with the horizontal.
A receiver is running such that he will pass under the
ball downfield from the point where it was released.
Assuming the quarterback and the receiver are the same
height, and the receiver can jump or reach one meter higher
than the point of release by the quarterback, what is the
minimum distance from the quarterback the ball can be caught?
\CenteredUnframedFixedFigure{m8gr10}

\end{two-digit-list}

\BriefAnsNewPage

If you do not understand an answer, refer to the last comment (after
the \Quote{Hints}) in the Special Assistance Supplement for that problem.
\begin{two-digit-list}
\item [1.] \NullItem
\begin{one-digit-list}
\item [a.] \m{x = (2\unit{m/s\up{2}}) t^2;\; y = (3\unit{m/s}) t};
\item [  ] \m{t = y/(3\unit{m/s})};
\item [  ] \m{x = (2\unit{m/s\up{2}}) \cdot [y/(3\unit{m/s})]^2 =
              \left[ 2/(9\unit{m}) \right] y^2},
           which is a parabola.
\item [b.] From \m{x = (2\unit{m/s\up{2}}) t^2}, \m{y = (3\unit{m/s}) t}, the
particle will pass through the point \m{x = 2\unit{m}}, \m{y = 3\unit{m}} at \m{t = 1\unit{s}}.
From part (a), \m{\vect{v}(1\unit{s}) = (4 \uvec{x} + 3 \uvec{y})\unit{m/s}}.
\end{one-digit-list}

\item [2.] In order to have a constant acceleration, \m{d^2\vect{r}/dt^2 = \text{ constant}}.
%
\begin{one-digit-list}
\item [a.] \m{\dfrac{d\vect{r}}{dt} = \vect{v} =
              [9\dfrac{t^2}{\unit{s\up{2}}} \uvec{x} +
              2 \dfrac{t}{\unit{s}} \uvec{y}]\unit{m/s}}\newline
%
           \m{\dfrac{d^2{\vect{r}}}{dt^2} = \dfrac{d\vect{v}}{dt} = \vect{a} =
              [18\dfrac{t}{\unit{s}} \uvec{x} + 2 \uvec{y}]\unit{m/s\up{2}}}:
           This is not constant in time.
\item [b.] \m{\vect{v} = 2 \uvec{x}\unit{m/s}};
           \m{\vect{a} = 0}.
           This is constant, but zero.
\item [c.] \m{\vect{v} = 4 t \uvec{x}\unit{m/s\up{2}}};
           \m{\vect{a} = 4 \uvec{x}\unit{m/s\up{2}}}.
           This is constant, and \m{\neq} zero.
\item [d.] \m{\vect{v} = 10 t \uvec{y}\unit{m/s\up{2}}};
           \m{\vect{a} = 10 \uvec{y}\unit{m/s\up{2}}}.
           This is constant and \m{\neq} zero.
\item [e.] \m{\vect{v} = 0};
           \m{\vect{a} = 0}.
           This is a zero acceleration and a zero velocity.
\end{one-digit-list}

\item [3.] Answer: b.

\item [4.] Answer: e.

\item [5.] a. \m{a_x = 0};
              \m{v_x = 5\unit{m/s}};
              \m{x = 5\unit{m/s}\,t};
\begin{one-digit-list}
\item [] \m{a_y = - 2\unit{m/s\up{2}}};
         \m{v_y = - 3\unit{m/s} - (2\unit{m/s\up{2}}) t};
\item [] \m{y = 10\unit{m} - (3\unit{m/s}) t - (2\unit{m/s\up{2}}) t^2/2}
\item [b.] \m{\vect{r}(t) = \uvec{x} x(t) + \uvec{y} y(t)}
           with \m{x(t)} and \m{y(t)} given in part a.
           \m{\vect{v}(t) = \uvec{x} v_x(t) + \uvec{y} v_y(t)}, etc.
\end{one-digit-list}

\item [6.] Given: \m{\vect{v}_0 = 0}.
           Choose the origin so \m{\vect{r}_0 = 0}.
           Then:\newline
\ItemFigure{\m{\vect{v} = (3 t \uvec{x} - 2 t \uvec{y})\unit{m/s\up{2}}};\newline
            \m{\vect{r} = \left[ (3/2) t^2 \uvec{x} - t^2 \uvec{y} \right] \unit{m/s\up{2}}}.\newline
From this,

\eqnline{x = (3/2) t^2\unit{m/s\up{2}};}
\eqnline{y = - t^2\unit{m/s\up{2}};}
\eqnline{x = - (3/2) y.}
Straight line slope \m{ = - 2/3}.}{m8gr11}

\item [7.] a. \m{\vect{v}_{av} = (\Delta x/\Delta t) \uvec{x} +
(\Delta y/\Delta t) \uvec{y}}.
\begin{one-digit-list}
\item [] The quantities \m{\Delta x} and \m{\Delta y} are the respective areas under the
\m{v(t)} curves for each component.
The area is just the area of a triangle.
\bigskip
\ItemFigure{
 \setlength{\tabcolsep}{2pt}
 \begin{tabular}{l c l}
 \m{\Delta x}: & & \\
 Area & = & \m{\Delta x} \\
      & = & (40\unit{s})(20\unit{m/s})/2 \\
      & = & 400\unit{m}. \\
 \end{tabular}
 \smallskip

 \begin{tabular}{l c l}
 \m{\Delta y}: & & \\
 Area 1 & = & displacement \\
        &   & (0 to 20\unit{s}) \\
        & = & (20\unit{s})(20\unit{m/s})/2 \\
        & = & 200\unit{m}. \\
 Area 2 & = & displacement \\
        &   &  (20\unit{s} to 40\unit{s}) \\
        & = &  (20\unit{s})(\m{- 20\unit{m/s}})/2 \\
        & = & \m{- 200\unit{m}}. \\
 \end{tabular}
}{m8gr12}

Total Area = Area 1 + Area 2 = 0.

\m{\vect{v}_{av} = (400\unit{m}/40\unit{s}) \uvec{x} +
(0\unit{m}/40\unit{s}) \uvec{y} = 10 \uvec{x}\unit{m/s}}.
\item [b.] \m{\vect{a}(20\unit{s}) = (dv_x/dt) \uvec{x} + (dv_y/dt) \uvec{y}}.
\newline
Using the fact that the derivative of a straight line is the slope
of the line,
\m{\vect{a}(20\unit{s}) = \left[- (1/2) \uvec{x} - \uvec{y}\right]\unit{m/s\up{2}}}
\item [c.] \m{\vect{a}_{av} = \Delta v/\Delta t = \left[\vect{v}(60\unit{s}) -
\vect{v}(20\unit{s})\right]/\,\left[60\unit{s} - 20\unit{s}\right]}\newline
\vect{v} at \m{t = 60unit{s}} is (from the graph) \m{(10 \uvec{x} - 20 \uvec{y})\unit{m/s}}.
\newline
\vect{v} at \m{t = 20\unit{s}} is \m{(10 \uvec{x} + 0 \uvec{y})\unit{m/s}}.\newline
\m{\vect{a}_{av} = - (1/2) \uvec{y}\unit{m/s\up{2}}}.
\end{one-digit-list}

\item [8.] Let \m{y} be positive down, \m{x} be positive south.\newline
\begin{tabular}{l l}
\m{a_x = 0} & \m{a_y = 0} \\
\m{v_x = 3\unit{m/s}} & \m{v_y = 8\unit{m/s}} \\
\m{x = 3\unit{m/s} t} & \m{y = 8\unit{m/s} t} \\
\end{tabular}

The time of descent is given by putting \m{y = 300\unit{m}}, and solving
for \m{t}: \m{t = 37.5\unit{s}}.
The displacement is obtained by putting \m{t = 37.5\unit{s}} into
the equation for \m{x}: \m{x = 112.5\unit{m\,south}}.

\item [9.] \m{\vect{v}(t) = \int_0^t \vect{a}(t')\,dt' + \vect{v}_0}

\m{\phantom{\vect{v}(t)} = \uvec{x} \int_0^t A\,dt' + \uvec{y} \int_0^t B\,dt' + 0}

\m{\phantom{\vect{v}(t)} = A t \uvec{x} + B t \uvec{y} = (A \uvec{x} + B \uvec{y}) t}

\m{\vect{v}_{av} = \dfrac{\int_0^t \vect{v}(t')\,dt'}{t - 0}}

\m{\phantom{\vect{v}} = \dfrac{\uvec{x}}{t}\left(\dfrac{1}{2} A t^2 \right) +
\dfrac{\uvec{y}}{t} \left( \dfrac{1}{2} B t^2 \right)}

\m{\phantom{\vect{v}} = \dfrac{1}{2}(A \uvec{x} + B \uvec{y}) t =
\dfrac{1}{2} \vect{v}(t)}

This is a very special result that holds true only for constant acceleration
from rest.
Any other set of conditions must be treated by the above method to find any
special relation between \m{\vect{v}_{av}} and \m{\vect{v}(t)}.

\item [10.] Choose a coordinate system with axes parallel and perpendicular to
the inclined surface.
Choose this coordinate system because then it is easy to
state the mathematical condition corresponding to the ball hitting the
incline: it is when (again) \m{y = 0}.
In such a system the acceleration of the ball, the acceleration of gravity,
is: \m{\vect{a} = \vect{g} = - g \sin30\degrees \uvec{x} - g \cos30\degrees \uvec{y}}.
\help{41}\newline
Using the equations for constant acceleration (the origin is at the boy):
\begin{eqnarray*}
a_x & = & - g \sin30\degrees \\
v_x & = & - g \sin30\degrees t - v_0 \cos30\degrees \\
x   & = & - (g/2) \sin30\degrees t^2 - v_0 \cos30\degrees t \\
a_y & = & - g \cos30\degrees \\
v_y & = & - g \cos30\degrees t + v_0 \sin30\degrees \\
y   & = & - (g/2) \cos30\degrees t^2 + v_0 \sin30\degrees t
\end{eqnarray*}
Putting \m{y = 0} and solving for \m{t} gives two solutions:
either \m{t = 0} or \m{t = 1.18\unit{s}}.
There are two times when the ball is at \m{y = 0}: when it is released and when
it strikes the surface.
The time in question is \m{t = 1.18\unit{s}}.
At this time, \m{x = - 13.62\unit{m}}.

\item [11.] a. \m{\vect{r}(t) = [(t/\unit{s})^2 \uvec{x} +
(t/\unit{s}) \uvec{y}]\unit{m}}
\begin{one-digit-list}
\item [b.] \m{\vect{v}_{av} = (5 \uvec{x} + \uvec{y})\unit{m/s}}                           \,\,
\end{one-digit-list}

\item [12.] a. \m{x = t\unit{m/s}} 
\begin{one-digit-list}
\item [] \m{y = \left[\left(\dfrac{4}{3}\right) \dfrac{t^3}{\unit{s\up{2}}} - 4 t\right]\unit{m/s}} .
\item [] Completing the table:
{\renewcommand{\arraystretch}{1.5}
\begin{tabular}[t]{|r r r|}\hline
  t(s) & x(m) & y(m) \\ \hline
  \m{-2.1} & \m{-2.1} & \m{-3.9} \\ \hline
  \m{-2.0} & \m{-2.0} & \m{-2.7} \\ \hline
  \m{-1.0} & \m{-1.0} & \m{ 2.7} \\ \hline
  \m{ 0.0} & \m{ 0.0} & \m{ 0.0} \\ \hline
  \m{ 1.0} & \m{ 1.0} & \m{-2.7} \\ \hline
  \m{ 2.0} & \m{ 2.0} & \m{ 2.7} \\ \hline
  \m{ 2.1} & \m{ 2.1} & \m{ 3.9} \\ \hline
\end{tabular}
}
\item [] See the graph in part (d).
\item [b.] \m{\vect{v} =
\left[\uvec{x} + \left(4 \dfrac{t^2}{\unit{s\up{2}}} - 4\right) \uvec{y}\right]\unit{m/s}}.
\m{\vect{v}(1\unit{s}) = \uvec{x}\unit{m/s}}.
At \m{t = 1}\,s, the particle is at \m{x = 1\unit{m}}, \m{y = - 2.7\unit{m}}.
The velocity at this time is indicated on the graph.
The point \m{x = 1\unit{m}}, \m{y = - 2.7\unit{m}} is the turn around point and the slope
\m{dy/dx = 0}.
The velocity is tangent to the curve.
\item [c.] \m{\vect{v}(- 2\unit{s}) = ( \uvec{x} + 12 \uvec{y} )\unit{m/s}}.
The particle is at \m{x = - 2\unit{m}}, \m{y = -2.7\unit{m}} and the velocity is as
indicated.
The velocity is tangent to the curve.
\item [d.] \m{\vect{a} = 8 t \uvec{y}\unit{m/s\up{3}}}.
At \m{t = - 1\unit{s}}, \m{\vect{a} = - 8 \uvec{y}\unit{m/s\up{2}}},
\m{\vect{v} = \uvec{x}\unit{m/s}} and \m{\vect{r} = (- \uvec{x} + 2.7 \uvec{y})\unit{m}}.
The velocity is horizontal at this point, and the acceleration is vertical.
Have you noted that \m{v_x} = constant in this problem?
See sketch below.
\CenteredUnframedFixedFigure{m8gr14}

\end{one-digit-list}
\item [13.] a. \m{\vect{v}_0 = v_0 \cos\theta \uvec{x} + v_0 \sin\theta \uvec{y}}
\begin{one-digit-list}
\item [b.] \m{v = - g t \sin\theta \cos\theta \uvec{x} -
g t \sin^2\theta \uvec{y} + v_0 \cos\theta \uvec{x} + v_0 \sin\theta \uvec{y}}

With the origin chosen so that \m{\vect{r}_0 = 0}:

\m{x = - \dfrac{1}{2} g t^2 \sin\theta\,\cos\theta + v_0 t \cos\theta},

\m{y = - \dfrac{1}{2} g t^2 \sin^2\theta + v_0 t \sin\theta},

\m{x/\cos\theta = -\dfrac{1}{2} g t^2 \sin\theta + v_0 t}, and

\m{y/\sin\theta = -\dfrac{1}{2} g t^2 \sin\theta + v_0 t}.

Therefore, \m{\dfrac{x}{\cos\theta} =
\dfrac{y}{\sin\theta}}, \m{y = x \tan\theta}, a straight line.
\end{one-digit-list}

\item [14.] \m{\vect{r}_0 = 0};
            \m{\vect{v}_0 = 0};
            \m{v = [-2 (t/\unit{s})^3 \uvec{x} + \dfrac{(t/\unit{s})^2}{2} \uvec{y}]\unit{m/s}};

\m{x = -\dfrac{(t/\unit{s})^4}{2}\unit{m}};

\m{y = \dfrac{(t/\unit{s})^3}{6}\unit{m}};

\m{x = -\dfrac{(6 y/\unit{m})^{4/3}}{2}\unit{m}}

or:  \m{(x/\unit{m})^3 = - 162 (y/\unit{m})^4}

\item [15.] Choose the origin at the point of release, choose \m{\uvec{y}} upward.

\m{a_x = 0}; \m{v_x = v_{0,x} = 20\,(\unit{m/s}) \cos45\degrees};

\m{x = 20 (\cos45\degrees)\unit{m} (t/\unit{s})}; \m{a_y = - g}; \m{v_y = v_{0,y} - g t}

\m{\phantom{x} = [20 \sin45\degrees - 9.8 (t/\unit{s})] \unit{m/s}};

\m{y = \left[- 4.9 (t/\unit{s})^2 + 20 \sin45\degrees (t/\unit{s})\right]\unit{m}}.

The range for the ball to be caught is \m{y \leq 1\unit{m}}.

Putting \m{y = 1\unit{m}} into the equation gives these two solutions:
\m{t = 2.81\unit{s}} and \m{t = 0.072\unit{s}}. \help{42}

At \m{t = 2.81\unit{s}}, \m{x = 39.7\unit{m}}, but at \m{t = 0.072\unit{s}}, \m{x = 1.02\unit{m}}.
Picture each of these in your mind's eye!
\end{two-digit-list}

}% /Sect

