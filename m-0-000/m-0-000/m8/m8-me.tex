\revhist{12/8/89, pss; 7/31/91, pss; 3/12/93, pss; 6/8/94, pss; 3/28/97, pss}

\Sect{}{}{\SectType{ModelExam}}{

\begin{one-digit-list}
\item [1.] See Output Skills K1-K2 on this module's \textit{ID Sheet}.

\item [2.] The velocity of a particle is given by the expression:
%
\Eqn{}{\vect{v} = [2 (t/\unit{s}) \uvec{x} + \uvec{y}]\unit{m/s}.}
%
\begin{one-digit-list}
\item [a.] At \m{t = 0} the particle is at \m{x = 0}, \m{y = 0}.
What is the position of the particle at a later time \m{t}?
\item [b.] What is the average velocity over the interval \m{t = 2\unit{s}} to
\m{t = 3\unit{s}}?
\end{one-digit-list}

\item [3.] \ItemFigure{A block is projected up an inclined surface which
makes an angle \m{\theta} with the horizontal.
The initial speed is \m{v_0} and the inclined surface is frictionless.
Use a coordinate system defined by \m{x} positive to the right
in a horizontal direction, \m{y} positive up (as illustrated).}{m8gr09}
\begin{one-digit-list}
\item [a.] What is the initial velocity?
\item [b.] The acceleration of the block is observed to be:
\Eqn{}{\vect{a} = - (g \sin\theta\,\cos\theta)\,\uvec{x} - (g \sin^2\theta)\,\uvec{y}.}
Develop expressions for \m{x(t)} and \m{y(t)}.
Divide one by the other to check that they (properly) predict that y(x) is a
straight-line function with slope \m{\tan\theta}.
\end{one-digit-list}

\item [4.] A particle moves with acceleration:
%
\Eqn{}{\vect{a} = [- 6\,(t/\unit{s})^2 \uvec{x} +
(t/\unit{s}) \uvec{y}]\unit{m/s\up{2}}.}
%
At \m{t = 0} the particle is at rest at the origin.
What is the equation of the path (i.e., the trajectory)?

\item [5.] A football is thrown by a quarterback at a speed of
20\unit{m/s} at an angle of {45\degrees} with the horizontal.
A receiver is running such that he will pass under the
ball downfield from the point where it was released.
Assuming the quarterback and the receiver are the same
height, and the receiver can jump or reach one meter higher
than the point of release by the quarterback, what is the
minimum distance from the quarterback the ball can be caught?
\CenteredUnframedFixedFigure{m8gr10}
\end{one-digit-list}

\BriefAns

\begin{one-digit-list}
\item [1.] See this module's \textit{text}.

\item [2.] See this module's \textit{Problem Supplement}, problem~11.

\item [3.] See this module's \textit{Problem Supplement}, problem~13.

\item [4.] See this module's \textit{Problem Supplement}, problem~14.

\item [5.] See this module's \textit{Problem Supplement}, problem~15.
\end{one-digit-list}
}% /Sect
