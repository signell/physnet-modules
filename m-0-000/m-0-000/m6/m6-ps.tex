\revhist{7/23/85, mpm; 3/16/88, pss; 12/8/89, pss; 10/12/91, pss; 9/16/94, pss; 4/25/02, pss;
         6/6/02, pss}

\Sect{}{}{\SectType{ProblemSet}}{

\noindent \textbf{Note 1}: There is a lot of general problem-solving help on the current topic
in this module's \textit{Special Assistance Supplement}.
We suggest you look over that material before trying to solve the problems below.

\noindent Note 2: Problems~2, 10, and 11 also occur in this module's \textit{Model
Exam}.

\begin{two-digit-list}
\item [1.] \CenteredUnframedFixedFigure{m6gr07}
\begin{one-digit-list}
\item [a.] Find the resultant force of this set of parallel forces.
\item [b.] Find the resultant torques produced by this set of forces around
the points (\m{x},\m{y}) = (0\unit{cm},0\unit{cm}), (\m{x},\m{y}) = (1\unit{cm},3\unit{cm}), and
(\m{x},\m{y}) = (1\unit{cm},4\unit{cm}).
\end{one-digit-list}

\item [2.] \CenteredUnframedFixedFigure{m6gr08}
          {Determine the tensions \m{T_1} and \m{T_2} in the ropes if the mass \m{M}
           weighs 10\unit{lb}.}

\item [3.] \CenteredUnframedFixedFigure{m6gr09}{Determine the force exerted on the beam by the wall at point \m{P}
           if the mass \m{M} weighs 20\unit{lb} and the weights of the cable and beam
           are negligible.}
           

\item [4.] \CenteredUnframedFixedFigure{m6gr10}
           A drawbridge weighing 2000\unit{lb} is suspended from the side of a
           castle by a cable, as shown above.
\begin{one-digit-list}
\item [a.] Find the tension in the cable.
\item [b.] Find the force exerted on the bridge by the wall of the building at
           point \m{P}.
           (Hint: you can consider the weight of the drawbridge to be acting on
           its center of gravity at a point halfway along its length.)
           \help{1}
\end{one-digit-list}

\item [5.] \CenteredUnframedFixedFigure{m6gr11}
           A uniform meterstick with two different weights on it is balanced on
           a pivot as shown in the sketch.
           Assume that the weight of the meterstick acts upon its center of
           gravity, which is located at the middle of the meter stick.
\begin{one-digit-list}
\item [a.] Find the weight of the meterstick, \m{W_\text{ms}}, in \unit{newtons}.
\item [b.] Find the force \m{F_p} exerted by the pivot upon the meterstick.
\end{one-digit-list}

\item [6.] \CenteredUnframedFixedFigure{m6gr12}%
           {A load of weight 2000\unit{lb} is suspended by a cable and boom from the
            side of a building, as shown in the sketch.
            The boom weighs 1000\unit{lb}.
            The center of gravity of the boom is halfway along its length.
\begin{one-digit-list}
\item [a.] Draw a one-body force diagram showing all the forces acting on the boom.
\item [b.] Find the tension in the cable.
\item [c.] Find the force exerted by the wall on the boom at the pivot point
           \m{P}.
\end{one-digit-list}
           }

\item [7.] \CenteredUnframedFixedFigure{m6gr13}{Find the center of mass of this right triangle with
           base \m{b} and height \m{h}, assuming it has a constant surface mass
           density \m{\sigma}.  (Hint: \m{dM = \sigma\,dy\,dx}).}
           

\item [8.] \CenteredUnframedFixedFigure{m6gr14}{Find the center of mass of the figure, assuming it has a constant
           surface mass density.}
           

\item [9.] \CenteredUnframedFixedFigure{m6gr15}{Find the center of mass of this figure assuming it has a constant
           surface mass density.}
           

\item [10.] \CenteredUnframedFixedFigure{m6gr16}{A van is loaded so that the load on each pair of
            wheels, front and back, is the same: 2400\unit{pounds}.
            The rear axle is 12\unit{feet} behind the front axle.
            The rear overhang of the van extends two feet behind the rear axle.
            If an additional weight of 600\unit{pounds} is loaded onto the very back
            of the van what is the new load distribution of the front and rear
            tires?}

            (Be sure to draw a correct one-body force diagram showing all of the
            forces acting on the object under consideration.)

\item [11.] The mass of the earth is \m{5.98 \times 10^{24}\unit{kilograms}} while the
            mass of the moon is \m{7.34 \times 10^{22}\unit{kilograms}}.
            The average earth-moon separation (center-to-center) is
            \m{3.84 \times 10^8\unit{meters}}.
            Find the location of the Center of Mass of the earth-moon system.
\end{two-digit-list}

\newpage

\BriefAns

\begin{two-digit-list}
\item [1.] \NullItem
\begin{one-digit-list}
\item [a.] \m{F_R = 5\unit{N}}, to the right
\item [b.] (0\unit{cm},0\unit{cm}): \m{\tau_R = 0\unit{N\,m}}\newline
           (1\unit{cm},3\unit{cm}): \m{\tau_R = 15\unit{N\,cm}}, out of the page\newline
           (1\unit{cm},4\unit{cm}): \m{\tau_R = 20\unit{N\,cm}}, out of the page\newline
           (Torques all different because resultant force is not zero).
\end{one-digit-list}

\item [2.] \m{T_1 = 5\unit{lb}};  \m{T_2 = 8.7\unit{cm}}.

\item [3.] \m{F_\text{wall} = 20\unit{lb}} to the right.

\item [4.] \NullItem
\begin{one-digit-list}
\item [a.] \m{T = 2000\unit{lb}}. (NOT 2000\unit{lb}/\m{\cos 60\degrees}) \help{1}\,
%
\Footnote{PS1}{See sequence [S-1] near the end of this module's \textit{Special
Assistance Supplement}.}
%
\item [b.] \m{F_{\text{wall},x} = 1000\sqrt{3}\unit{lb}}.\m{ = 1.7 \times 10^3\unit{lb}}\newline
           \m{F_{\text{wall},y} = 1000\unit{lb}}.
\item [  ] Magnitude is 2000\unit{lb}, directed at an angle of {30\degrees} above
           horizontal to the right.
\end{one-digit-list}

\item [5.] \NullItem
\begin{one-digit-list}
\item [a.] \m{W_\text{m.s.} = 2\unit{N}}.
\item [b.] \m{F_p = 6\unit{N}}.
\end{one-digit-list}

\item [6.] \NullItem
\begin{one-digit-list}
\item [a.] \CenteredUnframedFixedFigure{m6gr17}
\item [b.] \m{T = 2500\unit{lb}}.
\item [c.] \m{F_{\text{wall},x} = 1250 \sqrt{3}\unit{lb}};
           \m{F_{\text{wall},y} = 1750\unit{lb}};
           Magnitude is 2783.88\unit{lb}, directed at angle {38.95\degrees} above
           horizontal to right.
\end{one-digit-list}

\item [7.] \m{x_\text{CM} = 2/3\,b}; \m{y_\text{CM} = 1/3\,h}.

\item [8.] \m{x_\text{CM} = 19/15\unit{m}}; \m{y_\text{CM} = 16/15\unit{m}}.

\item [9.] \m{x_\text{CM} = 5/6\unit{m}};   \m{y_\text{CM} = 1.0\unit{m}}.

\item [10.] \m{F_f = 2300\unit{lb}}
\item [   ] \m{F_r = 3100\unit{lb}}.

\item [11.] \m{4.66 \times 10^6\unit{m}} from earth's center.
\end{two-digit-list}
}% /Sect
