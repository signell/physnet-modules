\revhist{9/13/89, pss; 4/11/91, pss; 6/29/91, pss; 7/30/92, pss; 3/23/94, pss;
         3/24/95, pss; 9/14/95, pss; 9/6/96, pss; 10/25/96, pss; 3/24/99, pss; 7/15/99, abs;
         2/27/01, pss; 4/9/01, pss}
%
\Sect{1}{Introduction}{\SectType{TextOnePara}}{
%
\par{The \Index{physical phenomena, description of}\Index{description, of
physical phenomena}description of physical phenomena without the results of measurements
and without mathematics is, at best, incomplete and very limited in its
scope.
For example, the observation that the moon orbits the earth is a
\Index{qualitative description}\Index{description, qualitative}qualitative
description of that motion.
The description becomes \Index{quantitative, description}\Index{quantitative
description}quantitative when measurements are made that give
the position of the moon relative to the earth for various times.
With the accumulation of such data, the description begins to enter the realm
of science when, on the basis of these data, the location of the moon can be
predicted for times in the future.
However, the description achieves full scientific status when it can also be
arrived at from a basic principle or physical law which not only fully
predicts all aspects of the observed motion of the moon around the earth,
but also can be applied to problems of the motion of other objects in other
environments.
Relating various observations through physical laws invariably involves the
use of mathematics.
In this module, we will consider some of the basic mathematical skills
necessary for understanding such applications.}% /par
%
\par{The functions whose derivatives are dealt with here constitute most
of the kinds of functions encountered in an introductory physics course.
For ready reference, we tabulate the rules and frequently used derivatives
in Appendix B.}% /par
}% /Sect
%
\Sect[\Index{differentiation}]{2}{Differentiating Common Functions}{\SectType{TextMultiPara}}{
%
\pcap[\Index{derivative, definition of}]{2}{a}{Definition of the Derivative}
%\SubSect[\Index{derivative, definition of}]{2}{a}{Definition of the Derivative}{
\par{The derivative of a function of a single independent variable, \m{y(x)}, with
respect to that independent variable, \m{x}, is defined as:
%
\Eqn{1}{\dfrac{dy(x)}{dx}=\lim_{\Delta x\rightarrow 0}
                            \dfrac{y(x+\Delta x)-y(x)}{\Delta x}}
%
where \m{\Delta x} is a small incremental change in the value of \m{x} that is
required to go to zero as the prescribed ratio is examined.}% /par
%
\par{As an example, consider %\Index{derivative, of (a/x)}\m{y(x)=a/x}
where \m{a} is a constant.
This relationship defines a value for \m{y} for each value of \m{x} except at
\m{x=0}.
The derivative of this function is:
%
\begin{eqnarray*}
\dfrac{dy(x)}{dx}&=&\dfrac{d}{dx}\left(\dfrac{a}{x}\right)\\
&=& \lim_{\Delta x\rightarrow 0}
    \dfrac{\dfrac{a}{x+\Delta x}-
    \dfrac{a}{x}}{\Delta x}\\
&=& \lim_{\Delta x\rightarrow 0}\dfrac{-a}{x(x+\Delta x)}
    \dfrac{\Delta x}{\Delta x}\\
&=&-\dfrac{a}{x^2 }.\\
\end{eqnarray*}
%
This function defines a value of the derivative of \m{y} for each value of \m{x}
except at \m{x=0}.}% /par
%}% /SubSect
%
\pcap{2}{b}{Derivatives of Simple Algebraic Functions}
%\SubSect{2}{b}{Derivatives of Simple Algebraic Functions}{
\par{It is not necessary to go through the cumbersome limiting procedure each
time a derivative of a function needs to be determined.
Instead, some general rules can be used.
For example, if the function is some power of \m{x}, \Index{derivative, of \m{a/x^p}}\m{y(x)=ax^p}, where \m{p} is
any positive or negative number, then:
%
\Eqn{}{\dfrac{dy(x)}{dx}=
   \lim_{\Delta x\rightarrow 0}\dfrac{a(x+\Delta x)^p -ax^p}{\Delta x}}
%
becomes, after applying the \Index{binomial theorem}binomial theorem and the limiting procedure,
\help{2}
%
\Footnote[1]{The [S-2] means that, if you need help, turn to the Special
Assistance Supplement at the end of this module and look at Sequence [S-2].}
%
\Eqn{2}{\dfrac{d(ax^p)}{dx}=pax^{p-1}.}
%
Equation (2) can be applied any time the derivative of any power of the
independent variable is desired.
Because the derivative \Index{derivative, of a sum of terms}of a sum of terms is the sum of the derivatives of
the terms, the above result can be used to produce the derivative of any
\Index{polynomial, derivative of}\Index{derivative, of polynomial}polynomial.
%
\Footnote[2]{A \Index{polynomial, definition of}polynomial is a sum of integer powers of the independent
          variable with constant coefficients.
          A polynomial of degree \m{n} in general is written as:
          %
          \Eqn{}{ y(x) = a_0 x^n + a_1 x^{n-1} + \ldots + a_{n-1}x + a_n.\qquad (a_0 \neq 0)}}
%
For example, if
%
\Eqn{}{y(t)=y_0 + v_0 t - \dfrac{g t^2}{2}}
%
where \m{y_0}, \m{v_0} and \m{g} are constants, and \m{t} is the independent
variable, then \Eqnref{2} produces:
%
\Eqn{}{\dfrac{dy(t)}{dt} = v_0 - g t.}}% /par
%}% /SubSect
%
\pcap[\Index{derivative, second}\Index{second derivatives}]{2}{c}{Second Derivatives}
%\SubSect[\Index{derivative, second}\Index{second derivatives}]{2}{c}{Second Derivatives}{
\par{The derivative of \m{dy(t)/dt}, called the second derivative of \m{y(t)}, is
written:
%
\Eqn{}{\dfrac{d^2 y(t)}{dt^2}.}
%
For example, for the function \m{y(t) = y_0 + v_0 t - gt^2/2}, double
application of rule (2) produces:
%
\Eqn{}{\dfrac{d}{dt}\left(\dfrac{dy(t)}{dt}\right) = - g.}
%
This \m{y(t)} is a polynomial of degree two, its first derivative is a
polynomial of degree one and its second derivative is a polynomial of degree
zero (just a constant).}% /par
%}% /SubSect
%
\pcap[\Index{product, derivative of}\Index{derivative, of product}]{2}{d}{Derivative of a Product}
%\SubSect[\Index{product, derivative of}\Index{derivative, of product}]{2}{d}{Derivative of a Product}{
\par{The derivatives of more complicated \Index{derivative, of algebraic function}\Index{algebraic function, derivative of}algebraic functions, such as the
products or ratios of polynomials, can also be readily found with the aid of
some basic rules.
Consider a function which can be written as the product (gf) of two
functions of the same independent variable; g(x) and f(x).
The derivative of this function, according to the definition, is:
%
\Eqn{}{\dfrac{d(gf)}{dx}=
\lim_{\Delta x\rightarrow 0}\dfrac{g(x+\Delta x)f(x+\Delta x)-g(x)f(x)}
{\Delta x}.}
%
If \m{g(x) f(x + \Delta x)} is added and subtracted in the numerator, the
expression may be rewritten in this way:
%
\Eqn{}{\dfrac{d(gf)}{dx} = \lim_{\Delta x\rightarrow 0}
\left(g(x)h_1(x)
+ h_2(x) f(x+\Delta x)\right)\,,}
%
where
%
\Eqn{}{h_1(x) \equiv \dfrac{f(x+\Delta x)-f(x)}{\Delta x}}
%
and
%
\Eqn{}{h_2(x) \equiv \dfrac{g(x+\Delta x)-g(x)}{\Delta x}\,.}
%
In the limit that \m{\Delta x} goes to zero, the two quantities \m{h_1(x)} and \m{h_2(x)} go to the derivatives of the respective functions while the
\m{f(x + \Delta x)} multiplying \m{h_2(x)} goes simply to \m{f(x)}.
Thus:
%
\Eqn{3}{\dfrac{d(gf)}{dx}=
           \left(\dfrac{dg}{dx}\right) f + g \left(\dfrac{df}{dx}\right) .}
%
Similarly:
\Eqn{4}{\dfrac{d}{dx}\left(\dfrac{g}{f}\right) =
\dfrac{\left(\dfrac{dg}{dx}\right) f -
g \left(\dfrac{df}{dx}\right)}{f^2}.}
%
These rules enable us to find the derivative of rational algebraic functions
(polynomials or ratios of polynomials), but not of irrational algebraic
functions (such as square roots of polynomials).}% /par
%}% /SubSect
%
\pcap{2}{e}{The Chain Rule}
%\SubSect{2}{e}{Chain Rule}{
\par{To enable us to find the derivative of an irrational algebraic function, we
need to use the \Quote{\Index{chain rule}chain rule.}

We are given \m{f} as a function of \m{x} and we suppose we can find its derivative,
%
\Eqn{25}{\dfrac{df}{dx}\,.}
%
Now suppose that \m{x} itself is a function of \m{t} so that \m{f} is also a function of \m{t}.
Then the chain rule says that the derivative of \m{f(x)} with respect to \m{t} is:
%
\Eqn{5}{\dfrac{df(x(t))}{dt}= \left(\dfrac{df}{dx}\right)\left(\dfrac{dx}{dt}\right) .}
%
Using this, the derivative of a function such as
%
\Eqn{}{F(x) = (ax^2 + bx + c)^{1/2} ,}
%
where \m{a}, \m{b}, and \m{c} are constants, can readily be shown to be: \help{5}
%
\Eqn{6}{\dfrac{dF(x)}{dx}=\dfrac{2ax+b}{2(ax^2 +bx+c)^{1/2}}.}}
%
% /SubSect
%
\pcap[\Index{trigonometric functions, derivative of}\Index{derivative, of trigonometric
functions}]{2}{f}{Derivatives of Trigonometric Functions}
%\SubSect[\Index{trigonometric functions, derivative of}\Index{derivative, of trigonometric
%functions}]{2}{f}{Derivatives of Trigonometric Functions}{
%
\par{We can evaluate the derivatives of transcendental functions, such as
\Index{SIN, derivative of}\Index{derivative, of SIN}\m{\sin x}, which cannot be expressed as rational or irrational algebraic
functions, using rules developed from the definition of the derivative.
Consider the function \m{y(x) = \sin (kx + \delta )} where \m{k} and \m{\delta}
are constants.
The derivative of \m{y(x)} is:
%
\Eqn{}{\dfrac{dy(x)}{dx}=\lim_{\Delta x\rightarrow 0}
    \dfrac{\sin(kx + k\Delta x + \delta) - \sin(kx+\delta )}{\Delta x}.}
%
Using a trigonometric identity,
%
\Footnote[3]{See {\em Handbook of Chemistry and Physics}, Chemical Rubber Publishing Co.
          The identity can be proved by recalling the expansion for \m{\sin(x\pm y)}
          and using the substitutions \m{A=x+y} and \m{B=x-y}.}
%
\Eqn{}{\sin A - \sin B = 2 \sin\left(\dfrac{A-B}{2}\right)
             cos\left(\dfrac{A+B}{2}\right) ,}
%
and the limit
%
\Eqn{}{\lim_{\theta \rightarrow 0}\dfrac{sin\theta }{\theta } = 1,}
%
we find that: \help{3}
\Eqn{7}{\dfrac{d}{dx}[\sin(kx + \delta)] = k \cos(kx + \delta).}
%
The \Index{derivative, of COS}\Index{COS, derivative of}derivative of \m{\cos(kx + \delta)} can be obtained directly from the
above result using the relations
%
\Eqn{}{\cos(kx + \delta) = -\sin\left(kx + \delta + \dfrac{\pi}{2}\right)}
%
and
%
\Eqn{8}{\cos\left(kx + \delta - \dfrac{\pi}{2}\right) =sin(kx+\delta),}
%
yielding:
\Eqn{9}{\dfrac{d}{dx}\cos(kx + \delta ) = - k \sin(kx + \delta).}
%
The derivatives of other trigonometric functions may be obtained from the
above results and the appropriate trigonometric identities.
For example, the derivative of \Index{TAN, derivative of}\Index{derivative, of TAN}\m{\tan\theta}  with respect to \m{\theta}  is:
%
\Eqn{10}{\dfrac{d}{d\theta}\tan\theta =
           \dfrac{d}{d\theta}\left(\dfrac{\sin\theta}
           {\cos\theta}\right) = \dfrac{1}{\cos^2\theta} = \sec^2\theta .}
%
Similarly, the derivatives of the \Index{inverse functions}\Index{functions, inverse}inverse trigonometric functions, such a\Index{derivative, of ARCTAN}\Index{ARCTAN, derivative of}s
\m{\theta = \arctan x} (also written as \m{\tan^{-1} x}, which means \m{\theta} is
the angle whose tangent is \m{x}) can also be obtained from the above results
using the rule:
%
\Eqn{11}{\dfrac{dy(x)}{dx}=\dfrac{1}{\dfrac{dx(y)}{dy}};
\;\left(\dfrac{dx}{dy}\neq 0\right) .}
%
Frequently used derivatives of trigonometric functions are listed in
Appendix B.} /par
%}% /SubSect
%
\pcap{2}{g}{Derivatives of Exponentials and Logarithms}
%\SubSect{2}{g}{Derivatives of Exponentials and Logarithms}{
\par{None of our rules, developed from algebraic and trigonometric functions,
apply to \Index{exponential function, derivative of}\Index{derivative, of exponential function}exponential and \Index{logarithm, derivative of}\Index{derivative, of logarithm}logarithmic functions, so here it's necessary to
begin again with the \Index{logarithm, definition of}definition of the derivative.
If \m{x = \log_a F}, then the \Index{derivatives, table of}\Index{table, of derivatives}derivative of \m{x} with respect to \m{F} is:
%
\FourEqns{}{\dfrac{dx}{dF} & = \dfrac{d}{dF}\left(\log_a F\right)}
           {               & = \lim_{\Delta F\rightarrow 0}\dfrac{\log_a(F+\Delta F)- \log_a(F)}{\Delta F}}
           {               & = \lim_{\Delta F\rightarrow 0}\dfrac{\log_a\left(1+\dfrac{\Delta F}{F}\right) }{\Delta F}}
           {               & = \lim_{\Delta F\rightarrow 0}\dfrac{\dfrac{\Delta F}{F}\log_a
                                               \left[\left(1+\dfrac{\Delta F}{F}
                                               \right)^{F/\Delta F}\right]}{\Delta F}}
%
Here we have used \Index{logarithm, properties of}property (3) of logarithms (see Appendix A).
What is the limit of the argument of this logarithm?
%
\Footnote[4]%
{The \Index{limit, of logarithm of function}limit of the logarithm of a \Index{logarithmic function}function is the same as the logarithm of
 the limit of the same function.
}
%
That is, what is:
%
\Eqn{}{y = \lim_{x \rightarrow 0} (1 + x)^{1/x} ?}
%
This limit has a definite value, as can be seen by evaluating it and
plotting it for a few values of \m{x} on both sides of zero.
%
\CaptionedFullFramedFigure{1}{A graph of \m{(1 + x)^{1/x}} in the vicinity of \m{x = 0}.}{m1gr01}
%
The value of the limit is the transcendental number \m{e}, which to 9 decimal
places is
%
\Eqn{}{e = 2.718281828\ldots}
%
As we shall see, \m{e}'s \Index{e, value of}actual value is often of no real interest in physical
problems, but it enters in a natural way as the exponential base appropriate
to the mathematical description of many different kinds of physical
phenomena.} /par
%
\par{Completing the development of the derivative of the logarithm, we get
%
\Eqn{13}{\dfrac{d\left(\log_a F\right)}{dF}=\dfrac{\log_a e}{F}}
%
If we choose the \Index{logarithmic base}\Index{base, logarithmic}base \m{a = e}, then the constant \m{\log_a e} becomes 1 (see
property (4) of logarithms, Appendix A).
Because this makes life simpler, we will restrict the remainder of our
review of derivatives of logarithms and exponentials to that base.
Furthermore, we write \Quote{\m{\log_e}} as simply \Quote{\m{\ell}n.}
Thus, for a variable \m{F}, the derivative of \Quote{\m{\log F} to the base \m{e},}
\m{\ell n\,F}, is:
%
\Eqn{14}{\dfrac{d(\ell n\,F)}{dF}=\dfrac{1}{F}.}
%
We can obtain the derivative of \m{e^x} with respect to \m{x} from the
reciprocal relation between the derivatives when the roles of the dependent
and independent variables are interchanged [See \Eqnref{10}].
We use
%
\begin{eqnarray*}
x(F) & = & \ell n\,F \\
F(x) & = & e^x \\
\dfrac{dx(F)}{dF} & = & \dfrac{1}{dF(x)/dx} \\
\end{eqnarray*}
%
to get:
%
\Eqn{15}{\dfrac{d}{dx}\left(e^x\right) = F = e^x.}
%
Thus the exponential (to the base \m{e}) is its own derivative!
Finally, using the chain rule, it can be shown that:
%
\Eqn{18}{\dfrac{d}{dx}\left(e^{mx}\right) = m e^{mx}.}}% /par
%}% /SubSect
}% /Sect
%
\Sect{3}{Derivatives as Graphical Slopes}{\SectType{TextMultiPara}}{
%
\pcap{3}{a}{Graph of An Equation}
The functional relationship between two variables, often expressed in the
form of an equation, may also be displayed on a graph.
The graphical representation has distinct visual advantages, presenting
all of the properties of the function in one picture.
All of the points that lie on the \Index{curve, of equation}curve are points which satisfy the
equation connecting the variables.

\CaptionedLeftFramedFigure{2}{The graph of \m{F(x) = Ax + b}.}{m1gr02}

As an example, consider the \Index{function, graph of}\Index{graph, of function}functional relationship between \m{F} and \m{x}
given by the equation
%
\Eqn{}{F(x) = A x + B,}
%
where \m{A} and \m{B} are constants.
For every value of \m{x} there is one value
\m{F(x)} and for every value of \m{F(x)} there is one value of \m{x}.
The points for which this correspondence exists can be connected by a
continuous curve.
This \Index{graph, of equation}\Index{equation, graph of}graph of the equation can be plotted on a rectangular coordinate system
in which \m{F(x)} and \m{x} are the perpendicular axes (See \Figref{2}.).
The graph of this equation is a straight line, reflecting the linear
relationship between \m{F} and \m{x}.
Given a value \m{x = x_1}, the value of \m{F(x)} which satisfies the equation
can be obtained from the equation \m{F(x_1) = A x_1 + B} or can be determined
from the graph [point \m{P} on the graph in \Figref{2} has coordinates
\m{x_1}, \m{F(x_1)}].
%
\Footnote[5]%
{On a two-dimensional rectangular plot it is customary to represent
 points by the ordered pair of numbers [\m{x ,y}] which are, respectively, the
 horizontal and vertical coordinates of the point.
}

\CaptionedFullFramedFigure{3}{The slope of a curve \m{G(x)}, at some point
                              \m{P}, equals the slope of the tangent to the curve at that point.}{m1gr03}

\pcap[\Index{slope, of line}\Index{line, slope of}]{3}{b}{Slope of a Straight Line}
A property associated with equations which is of importance in physical
applications and which can most easily be discussed with reference to the
graph of the equation is the rate at which one variable changes due to
changes in the other.
For a graph which is a straight line, the rate is the same as the slope of
the line (the slope being the tangent of the angle that the line makes with
the horizontal axis).
Referring to \Figref{2}, if \m{P_1} and \m{P_2} are two points whose \Index{coordinates}coordinates
are [\m{x_1}, \m{F(x_1)}] and [\m{x_2}, \m{F(x_2)}], then the rate at which \m{F}
changes between those two points relative to the corresponding change in \m{x}
is:
%
\Eqn{}{\dfrac{\Delta F}{\Delta x}=\dfrac{F(x_2)-F(x_1)}{x_2-x_1}.}
%
Let \m{\Delta x = x_2 - x_1} so that \m{F(x_2) = F(x_1 + \Delta x)} and the
slope of the line may be written as:
%
\Eqn{16}{\text{slope } \equiv \dfrac{F(x_1 + \Delta x) - F(x_1)}{\Delta x} \qquad \text{(straight line)}.}
%
For a straight line, the result of \Eqnref{16} is clearly the same no
matter what size we pick for \m{\Delta x}.
That will not be the case for curves that are not straight lines.

\pcap[\Index{slope, of curve}\Index{curve, slope of}]{3}{c}{Slopes of Curves}
If we take the limit of \Eqnref{16} as \m{\Delta x \rightarrow 0}, this
expression for the slope
becomes the derivative of \m{F} with respect to \m{x}, and we can define the
slope at a single point on any curve as:
%
\Eqn{17}{\text{slope at point } x_1 \equiv
                                \left.\dfrac{dF(x)}{dx}\right|_{x = x_1}.}
%
For a function whose graph is not a straight line,
the \Index{slope, of function}\Index{function, slope of}slope does depend upon the point at which it is evaluated.
The slope in such cases is the slope of the straight line that is
tangent to the curve at the point in question (see \Figref{3}).

\pcap{3}{d}{Finding Maximum and Minimum Points}
The collection of all \Index{maximum point}maximum and \Index{minimum point}minimum points of a
function can be found by finding the points of \Index{zero slope}zero \Index{slope, sign of}slope (see \Figref{4}).
From the definition of slope [see \Eqnref{17}] it is apparent that the
slopes at \m{P_1} and \m{P_5} in \Figref{4} are \Index{positive slope}positive.
However, the slope at \m{P_3}, on the decreasing part of the curve, is
\Index{negative slope}negative.
At point \m{P_2}, where the tangent to the curve is horizontal, the slope is
zero.
Similarly, the slope is zero at \m{P_4}.
The two points, \m{P_2} and \m{P_4}, are, respectively, a \Quote{\Index{point, maximum}\Index{maximum point}maximum point} and
a \Quote{\Index{point, minimum}\Index{minimum point}minimum point} of the curve.
At \m{P_2} all points on the curve in the immediate neighborhood of \m{P_2} have
\m{y(x)} less than the corresponding value at \m{P_2}, while at \m{P_4} all points
in the neighborhood of \m{P_4} have \m{y(x)} greater than the value of \m{P_4}.
This illustration indicates the method for determining the maximum and
minimum points of a function:
Find the values of \m{x} which make the derivative of the function equal to
zero.

\tryit Show that the locations of the maximum and minima of the function
%
\Eqn{}{y(x) = 5 x^3 - 2 x^2 - 3 x + 2.}
%
are at:  \m{x = 3/5; x = -1/3}. \help{1}
Show that the first is where \m{y} is a minimum, the \Index{second derivative}second where it is a
maximum.
\help{1}

\pcap{3}{e}{Separating into Maxima and Minima}
How do you determine whether a point of zero slope is a maximum or a
minimum?
You could plot a graph and examine it, or you could evaluate the second
derivative of the function at the point in question.
Referring to \Figref{4}, the slope of the curve is positive to the left of
\m{P_2}, zero at \m{P_2}, and negative to the right of \m{P_2}.
Hence the slope is decreasing.
The derivative of the slope, the second derivative of the function \m{y(x)}
itself, is therefore negative.
Thus the second derivative is negative at a maximum---this is because the
second derivative is the \Quote{bending function} and at a maximum it is bending
the function down, negatively.
Correspondingly, at the location of a minimum, the second derivative of the
function, the \Quote{bending,} is positive.

\tryit See for yourself that the function
%
\Eqn{}{y(x) = 2x^3 - 3x^2 + 4,}
%
has a maximum at point (0,4) and a minimum at point (1,3).

\CaptionedFullFramedFigure{4}{For the curve shown, the slopes at \m{P_1}
          and \m{P_5} are positive numbers while the slope at \m{P_3} is negative.
          At the points \m{P_2} and \m{P_4}, where the tangent line is horizontal, the
          slope is zero.}{m1gr04}
}% /Sect

\Sect{4}{Definite and \Quote{Indefinite} Integrals}{\SectType{TextMultiPara}}{
%\Sect{4}{Definite and Indefinite Integrals}{\SectType{TextMultiPara}}{
%
\pcap[\Index{integral, indefinite}\Index{indefinite integral}]{4}{a}{Indefinite Integrals}
Knowing the derivative of a function with respect to an independent
variable, we often wish to determine the function itself.
The inverse of the derivative is desired here, and the procedure is called
\Quote{\Index{integration}integration.}
The result of this procedure is called \Quote{\Index{integral}the integral.}
For example, because \m{2x} is the derivative of \m{x^2}  with respect to \m{x},
the integral of \m{2x} is \m{x^2}.
However, \m{x^2 + 4} is also the integral of \m{2x}, as is \m{x^2 - 10}.
In fact, because the derivative of a constant is zero, the integral of \m{2x}
is \m{x^2 + C}, where \m{C} is any constant.
Because of the presence of the indefinite constant, \m{x^2 + C} is called the
\Quote{indefinite integral} of \m{2x}.
This inverse of differentiation, the indefinite integral, is customarily
written in the form:
%
\Eqn{}{\int  y(x)\,dx = F(x) + C.}
%
That is, the indefinite integral of \m{y(x)} with respect to \m{x} is:
\m{F(x) + C}.
This will be true if \m{y(x)} is the derivative of \m{F(x) + C} with respect to
\m{x}.
The function being integrated, \m{y(x)}, is called \Quote{the \Index{integrand}integrand.}
Examples:
%
\begin{eqnarray*}
  (i) & \int & \left( 4x^2 +7\right) dx=\dfrac{4}{3}x^3 +7x+C \\
 (ii) & \int & x^n\,dx = \dfrac{x^{n+1}}{n+1} + C \\
(iii) & \int & \dfrac{dx}{x}=\ell n\,x + C \\
 (iv) & \int & \cos\theta\,d\theta = \sin\theta + C \\
  (v) & \int & e^x dx = e^x + C                   \\
 (vi) & \int & \dfrac{dI(x)}{dx}dx = I(x) + C \qquad \text{for any } I(x). \\
\end{eqnarray*}
%
The last result, (vi), can be verified by taking the derivative with respect
to \m{x} of the right side of each equation, thereby producing the integrand
of the integral on the left.
The integrals of other frequently encountered functions, such as those
appearing to the right of the equality sign for entries in Appendix B, may
be determined by identifying the integrands as the derivative of sought-for
answers as illustrated in (vi).

\pcap{4}{b}{Integration by Parts}
The \Index{integrals, techniques for evaluating}technique of \Quote{integration by parts}
is used to integrate a product of two functions, one of which can be written as a derivative
of yet a third function:
%
\Eqn{}{\text{integrand} = f_1(x) f_2(x) = f_1(x) \dfrac{f_3(x)}{dx}\,.}
%
The technique still involves doing an integral, but it has a different integrand which
one hopes will be easier to integrate than was the original integrand.
To derive the formula for integration by parts, we simply integrate and rearrange
\EqnXref{B2} of Appendix B,
%
\Eqn{}{\dfrac{d}{dx}[f(x)g(x)] = \dfrac{df(x)}{dx} g(x) + f(x)\dfrac{dg(x)}{dx}\,,}
%
to get:
%
\Eqn{}{\int g(x)\dfrac{df(x)}{dx}\,dx = f(x)g(x) - \int f(x)\dfrac{dg(x)}{dx}\,dx\,.}
%
This is the well-known formula for \Quote{integration by parts.}
%
\Footnote[6]{The formula is sometimes written with \m{f(x)} replaced by \m{u(x)} and with
\m{g(x)} replaced by \m{v(x)}:
\m{\int v \, \dfrac{du}{dx}\,dx =
                            u \, v - \int u \, \dfrac{dv}{dx}\,dx} or
\m{\int u \, dv = u \, v - \int v \, du}.}
%
Consider, for example, the integral of \m{x\,\cos x}.
The integrand is the product of two functions, \m{x} and \m{\cos x},
We can let \m{g(x) = x} and \m{df(x)/dx = \cos x}, so:
%
\Eqn{}{\dfrac{dg(x)}{dx}=1 \qquad \text{and} \qquad f(x)= \sin x + C.}
%
Substituting into the integral gives:
%
\Eqn{}{\int \,x\, \cos x\,dx = x (\sin x + C) - \int \,(\sin x + C)\,dx,}
%
or:
%
\Eqn{}{\int \,x\, \cos x\, dx = x \sin x + \cos x + C.}
%
\tryit Try making the other choice, \m{g(x) = \cos x} and \m{df(x)/dx = x}
so \m{f(x) = (1/2) x^2}, and see what happens.

\pcap[\Index{limits, of integral}\Index{integral, limits of}]{4}{c}{Other Integration Techniques}
Other difficult-to-perform integrations may be reduced to tractable
form by the introduction of a new variable.
Integrands involving radicals, such as \m{\left(a^2 + x^2 \right)^{1/2}},
where \m{a} is a constant, can be performed by making a trigonometric
substitution (\m{x = a \tan\theta} in the above case).

Tables of integrals of functions of many kinds are available and are the
quickest technique for evaluating the integrals of functions which can be
cast into certain standard forms.
%
\Footnote[7]{For example, {\em A Short Table of Integrals}, Third Revised Edition,
          B.\,O.\,Pierce, Ginn and Co. (1929).
          There is also a table of integrals included in the
          {\em Handbook of Chemistry and Physics}, Chemical Rubber
          Publishing Co., any edition.}
%
There are also computer programs that produce either formal or numerical
integrals.
\Footnote[8]{See \Quote{Numerical Integration} (MISN-0-349) and \Quote{Formal Mathematical
          Solutions by Computer} (MISN-0-361).}
%
\CaptionedFullFramedFigure{5}{The integral of \m{y(x)} from \m{x_0} to \m{x} equals the area that
                              is under the curve and bounded by \m{x_0} and \m{x}.}{m1gr05}

\pcap[\Index{integral, definite}\Index{definite integral}]{4}{d}{Definite Integrals}
While the derivative of a function equals the slope of the curve
representing the function, the integral of a function equals the area under
the function's curve (see \Figref{5}).
To see this, consider the graph of some function \m{y(x)} shown in \Figref{5}.
The area under the curve, between the value \m{x_0} and some other value \m{x},
is the area \m{A} shown in the figure.
If \m{x} is increased by the incremental amount \m{\Delta x}, the area under the
curve increases by some amount we label \m{\Delta A}.
This amount \m{\Delta A} is more than the area of the rectangle of area
\m{y(x)\,\Delta x} (the
smaller dotted rectangle to the right of \m{A}) and less than the area of the
larger rectangle \m{y(x + \Delta x) \Delta x}.
The incremental area may be written as:
%
\Eqn{}{\Delta A = y(x)\,\Delta x + \text{ an amount less than } [y(x + \Delta x) - y(x)]\,\Delta x.}
%
Then:
%
\Eqn{}{\dfrac{\Delta A}{\Delta x} = y(x) + \text{ an amount less than } y(x + \Delta x) - y(x).}
%
In the limit as \m{\Delta x \rightarrow 0,} the second area on the right goes
to zero while the left side of the equation becomes \m{dA/dx}:
%
\Eqn{}{\dfrac{dA}{dx}=y(x).}
%
Integrating this expression, the area under any \m{f(x)} between any \m{x_1} and
\m{x_2} is:
%
\Eqn{}{A(x_1,x_2) = \int_{x_1}^{x_2} y(x) \,dx.}
%
For parts of the curve that are below the \m{x}-axis, the area between the
curve and the \m{x}-axis is given a negative numerical value since the
function is negative there.
Thus for the graph of the sine function, shown in \Figref{6}, in the interval
from \m{x = 0} to \m{x = 2\pi}, there is just as much area below the axis
(negative area) as there is above the axis (positive area), so they cancel
each other and the integral from 0 to 2\m{\pi}  is zero:

\CaptionedFullFramedFigure{6}{Graph of \m{\sin x} illustrating that the
                              integral over the full cycle, 0 to 2\m{\pi}, is zero.
                              The part of the area above the axis, \m{0} to \m{\pi}, is equal to
                              the part of the area below axis, \m{\pi}  to \m{2\pi}.}{m1gr06}
%
\Eqn{}{\int_0^{2 \pi} \sin x dx =
                             \int_0^{\pi} \sin x\; dx + \int_{\pi}^{2\pi}
\sin x\; dx = 2 + (-2) = 0.}
}% /Sect
%
\Sect{}{Acknowledgments}{\SectType{Acknowledgments}}{\NsfAcknowledgment}
%
\Sect{A}{Logarithms}{\SectType{AppendixOnePara}}{
%
The logarithm of a number \m{F} to the base \m{a} is the power to which \m{a} must
be raised to yield the number \m{F}:
%
\begin{center}\begin{tabular}{r l}
if:   & \m{F = a^x}          \\
then: & \m{\log_a F = x} \,. \\
\end{tabular}\end{center}
%
%\Eqn{}{x = \log_a F, \qquad \text{ if: } F = a^x.}
%
For example, to the base \m{a = 10}, the logarithm of \m{F = 100} is
\m{x = \log_{10} F = 2},
while to the base \m{a = 2}, the logarithms of \m{F = 1,024} and \m{F = 10} are
\m{x = \log_2 F = 10} and 3.32, respectively.
The familiar properties of logarithms, listed below, all follow from the
definition.

\noindent
If \m{F = a^x} and \m{G = a^y}, then \m{FG = a^x a^y = a^{x + y}}, and
%
\Eqn{A1}{\log_a(FG) = x + y = \log_a F + \log_a G.}
\Eqn{A2}{\log_a\dfrac{F}{G}=\log_a F-\log_a G}
\Eqn{A3}{\log_a F^n = n\,\log_a F}
\Eqn{A4}{\log_a a = 1}
\Eqn{A5}{\log_a 1 = 0}
%
Each of these can be proved using the definition of the logarithm.
Another useful property expresses the relationship between the logarithms of
the same number \m{F} in two different bases, \m{a} and \m{b}:
%
\Eqn{12}{\log_b F = (\log_a F)\cdot(\log_b a).}
%
The inverse functional relationships,
%
\Eqn{}{x(F) = \log_a F;\qquad \text{and} \qquad F(x) = a^x}
%
define a value of \m{x} for each given value of \m{F}, and vice versa.
}% /Sect
%
\Sect{B}{Frequently Used Derivatives}{\SectType{AppendixOnePara}}{
%
\Eqn{B1}{\dfrac{d}{dx}\left(a x^n\right) = n a x^{n-1}}
\Eqn{B2}{\dfrac{d}{dx}[f(x)g(x)]= g(x)\dfrac{df(x)}{dx}+f(x)\dfrac{dg(x)}{dx}}
\Eqn{B3}{\dfrac{d}{dx}\left[\dfrac{f(x)}{g(x)}\right] = \dfrac{g(x)\dfrac{df(x)}{dx}-f(x)\dfrac{dg(x)}{dx}}{[g(x)]^2}}
\Eqn{B4}{\dfrac{d}{dt}[f(x)]=\dfrac{df}{dx}\cdot \dfrac{dx}{dt}}
\Eqn{B5}{\dfrac{df(x)}{dx}=\left[\dfrac{dx(f)}{df}\right]^{-1}}
\lcr{}{\fbox{Trigonometric}}{}
\Eqn{B6}{\dfrac{d}{d\theta}\sin\theta = \cos\theta}
\Eqn{B6'}{\dfrac{d}{d\theta}\sin m\theta = m \cos m\theta}
\Eqn{B7}{\dfrac{d}{d\theta}\cos\theta = - \sin\theta}
\Eqn{B8}{\dfrac{d}{d\theta}\tan\theta = \sec^2 \theta}
\Eqn{B9}{\dfrac{d}{d\theta}\cot\theta = - \csc^2 \theta}
\Eqn{B10}{\dfrac{d}{d\theta}\sec\theta = \sec\theta \tan\theta}
\Eqn{B11}{\dfrac{d}{d\theta}\csc\theta = - \csc\theta \cot\theta}
\Eqn{B12}{\dfrac{d}{d\theta}\sin^{-1}x(\theta)=\dfrac{\dfrac{dx}{d\theta}}{(1-x^2 )^{1/2}}}
\Eqn{B13}{\dfrac{d}{d\theta}\cos^{-1}x(\theta)=\dfrac{-\dfrac{dx}{d\theta}}{(1-x^2 )^{1/2}}}
\Eqn{B14}{\dfrac{d}{d\theta}\tan^{-1}x(\theta)=\dfrac{\dfrac{dx}{d\theta}}{1+x^2}}
\lcr{}{\fbox{Logarithmic}}{}
\Eqn{B15}{\dfrac{d}{dx}\log_a U(x) =\left[\dfrac{1}{U(x)}\right] \left[\log_a e\right]\left[\dfrac{dU(x)}{dx}\right]}
\Eqn{B16}{\dfrac{d}{dx}\ell n\,x = \dfrac{1}{x}}
\Eqn{B17}{\dfrac{d}{dx}\left(a^x\right) = a^x log_e a}
\Eqn{B18}{\dfrac{d}{dx}\left(e^{y(x)}\right) =e^{y(x)}\left(\dfrac{dy(x)}{dx}\right)}
}% /Sect

