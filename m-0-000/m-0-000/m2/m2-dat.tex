\revhist{2/10/93, pss; 6/29/93, pss; 9/16/93, pss; 2/1/94, pss; 2/25/94, pss;
         9/29/94, pss; 9/15/95, lae; 5/31/96, pss; 10/12/96, pss; 1/31/97, pss;
         3/13/97, pss; 6/5/97, abs; 11/13/97, pss; 2/11/99, pss; 2/22/99, pss;
         9/21/99, pss; 2/24/00, pss; 3/2/00, pss; 12/18/00, pss; 1/5/01, pss; 4/16/02, pss;
         5/2/02, pss; 6/14/02, pss; 9/25/02, pss; 10/3/02, pss; 12/9/02, pss; 12/13/02, pss}
%
\defModTitle{\ph{Sums, Differences and} \ph{Products of Vectors}}
\defCtAuthor{\inits{J.}\inits{S.}Kovacs, Michigan State University}
\defIdAuthor{J.\,S. Kovacs, Department of Physics and Astronomy, Michigan State University,
East Lansing, MI}
%
\defIdItems{
    \IdVersEval{12/13/2002}{0}
    \IdHours{1}
    \begin{InputSkills}
    \item [1.]  Given some combination of angle(s) and leg length(s) for a triangle,
    determine the other angle(s) and leg length(s) \prrqone{0-401}.
    \item [2.] Define the multiplication of a vector by a number and illustrate with a
    drawing \prrqone{0-405}.
    \item [3.] Define the addition and subtraction of two vectors and illustrate with
    drawings \prrqone{0-405}.
    \end{InputSkills}
    \begin{KnowledgeSkills}
    \item [K1.] In two dimensions, add any number of given vectors graphically.
    \item [K2.] Given a vector's components in two dimensions, determine the
    magnitude of the vector and the angle it makes with each of the two coordinate
    axes, and conversely.
    \item [K3.] In three dimensions, add and subtract given vectors expressed as
    (a) a triad of components and (b) in terms of unit vectors along the coordinate
    axes.
    \item [K4.] Evaluate the scalar product of two given vectors, both in terms of
    the vectors' components along a fixed set of axes and in terms of the vectors'
    magnitudes and the angle between them.
    \item [K5.] Evaluate the magnitude of the vector product of two given vectors
    in terms of: their magnitudes and the angle between them.
    Determine the direction by either the right-hand rule or the screw rule.
    \item [K6.] Determine the vector product of two vectors, using their given
    components along a set of Cartesian coordinate axes.
    \end{KnowledgeSkills}
    \begin{OptionalResources}
    \item [1.] Ronald C.\,Davidson and Jerry B.\,Marion, \textit{Mathematical
    Preparation for General Physics and Calculus}, W.\,B.\,Saunders Co. (1973),
    (pp.\,144-151).
    \end{OptionalResources}
}