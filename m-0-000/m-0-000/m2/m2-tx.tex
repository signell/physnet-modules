\revhist{7/5/91, pss; 9/16/93, pss; 2/25/94, pss; 9/29/94, pss; 8/16/1995, pss; 1/31/97, pss;
         11/7/97, lae; 2/24/00, pss; 1/4/01, pss; 5/2/02, pss; 9/25/02, pss}
%
\Sect{1}{Introduction}{\SectType{TextMultiPara}}{
%
\pcap{1}{a}{Scalars, Vectors, and Others}
Most physical quantities can be classified mathematically as either scalars or
\Index{vector}vectors.
Values of \textit{scalar} \Index{scalar quantity}\Index{quantity, scalar}quantities are single numbers, sizes, and this includes such
quantities as temperature, work, distance, speed, musical pitch, and
electric charge.
Values of \textit{vector} quantities have both size and direction, and this includes
such quantities as displacement, velocity, acceleration, force, electric field,
and momentum.
There are other quantities whose values cannot be represented by either scalars
or vectors; examples are color, stress, and inertia.

The rules for adding, subtracting, and multiplying scalars are just the
familiar rules of arithmetic and algebra.
In this module we deal with the rules for adding, subtracting, and
multiplying vectors, rules that are used throughout professional science
and engineering.

\pcap{1}{b}{Indicating Vectors Visually}
Just as scalar quantities are referred to in graphs, drawings, text, and
equations by alphabetic symbols, vector quantities are indicated by the
use of arrows.

As a scalar example, suppose the length of a panel of concrete in a highway
increases as its temperature rises.
While describing the situation mathematically, we represent the length by the
symbol \m{L} and the temperature by the symbol \m{T}.
To indicate that the length is a function of temperature, we write \m{L(T)}.

As a vector example, suppose a particular force is changing with time.
We write \vect{F} to indicate that we are treating the force as a vector and
\m{t} to show that we are treating time as a scalar.
To indicate that the force is a function of time we write \m{\vect{F}(t)}.
If we are only interested in the size of the force, not its direction, we
write: \m{F(t)}.

When we want to indicate the size and \Index{direction, of vector}direction of a vector quantity on a
graph or a sketch of some physical phenomenon, we draw it as an
arrow, a line with an arrowhead on one end (see \Figref{1}).
The length of the arrow shows the vector quantity's \Index{magnitude, of vector}\textit{magnitude} (its size)
and the arrowhead shows its direction.

\CaptionedLeftFramedFigure{1}{Force indicated on a diagram.}{m2gr01}

\tryit In \Figref{1} the force scale is 10\unit{lb per inch}.
Determine the magnitude of the force. \help{1}
%
\Footnote{1}{This notation indicates that help is available in sequence [S-1]
of this module's \textit{Special Assistance Supplement}.}
%
}% /Sect
%
\Sect{2}{Addition, Subtraction of Vectors}{\SectType{TextMultiPara}}{
%
\pcap{2}{a}{General Features}
When the rules of \Index{addition, of vectors}vector addition are properly used to add two vectors,
the result is a third vector which usually has a different magnitude and
direction than those of the two vectors being added.
We here present two methods for adding two vectors: both produce the same
result.
In the \textit{graphical} method you draw the vectors to scale and then
construct the sum on the drawing and measure its magnitude and direction.
The result only has the precision of your drawing and does not produce
symbolic answers.
In the \textit{trigonometric} method you sketch the vectors and apply simple rules
of trigonometry.

\pcap{2}{b}{Addition: Graphical Method}
Here are the rules for \Index{vector addition}adding two vectors graphically:
\begin{itemize}
\item [1.] Take either of the vectors to be added and draw it on graph paper
to scale, so it is an arrow of appropriate \Index{magnitude, of vector sum}magnitude and direction.
\item [2.] Draw the arrow representing the second vector on the same graph,
to the same scale, with the correct magnitude and direction, but with its tail
at the head of the first arrow.
\item [3.] Draw a third arrow on the graph, going from the tail of the first
arrow to the head of the second arrow.
This third arrow is the sum of the other two: it is called the \Quote{\Index{resultant vector}resultant,}
which is the vector word for \Quote{\Index{vector sum}\Index{sum, of two vectors}sum.}
\end{itemize}
These three steps are illustrated in \Figref{2}, where each of the
vectors being added is taken in turn as the first vector.
Note that the resultant is the same in both cases.
%
\Footnote{2}{Mathematically, one says that vector addition is commutative.}
%
\CaptionedLeftFramedFigure{2}{Two vectors are added two ways:
(a) \vect{A} is drawn first and then \vect{B} is added to it;
(b) \vect{B} is drawn first and then \vect{A} is added to it.
The resultants in the two methods are the same: they have identical
magnitudes and directions.}{m2gr02}

\tryit Show, graphically, that the result of \m{\vect{A} + \vect{B}} is the same
as the result of \m{\vect{B} + \vect{A}}. \help{2}

\pcap[\Index{trigonometric determination of vector sum}]{2}{c}{Addition: Trigonometric Method}
The trigonometric method of \Index{vector addition}adding two vectors makes use of the law of cosines.
This is illustrated in \Figref{3}, where \vect{R} is the \Quote{resultant} from
adding vectors \vect{A} and \vect{B}.
The \Index{sum, of two vectors}\Index{vector sum}\Index{vector sum, magnitude of}\Index{magnitude, of vector sum}magnitude of \vect{R} is determined from the right triangle whose
sides are \m{A + B \cos\phi} and \m{B \sin\phi} and whose hypotenuse is \m{R}.
Here \m{\phi} is the angle \m{B} makes with \m{A}.
Then:
%
\Eqn{}{R^2 = (A + B \cos\phi)^2 + B^2 \sin^2\phi\,,}
%
and the magnitude of \m{R} is:
%
\Eqn{1}{R = \left( A^2 + B^2 + 2AB \cos\phi \right)^{1/2}\,.}
%
Most scientists and engineers either remember this equation or become
proficient at quickly deriving it.
%
\Footnote{3}{The derivation is in Sect.\,4d of this module.}
%
The angle that vector \vect{R} makes with the vector \vect{A} can also be
expressed in terms of the magnitudes of \vect{A} and \vect{B} and the angle
\m{\phi} by (see \Figref{3}):
%
\Eqn{2}{\theta =\tan^{-1}\left[\dfrac{B\sin\phi}{A+B\cos\phi}\right].}
%

%
\tryit Check \Eqnsref{1} and \Eqnssref{2} by measuring and
%
then calculating with the relevant quantities in \Figref{3}. \help{3}
%
\TwoCaptionedFramedFigures{3}{Using trigonometry to add two vectors.}{m2gr03}%
                          {4}{Illustration of vector subtraction.}{m2gr04}

\pcap[\Index{subtraction, of vectors}]{2}{d}{Subtraction}
Subtraction is performed by using the rules for vector addition.
That is, the subtraction of one vector from another is accomplished by adding
the negative of the vector being \Quote{subtracted.}
The negative of a vector is most easily seen from its \Index{vector product, geometric definition of}geometrical
representation where the head and the tail of the arrow are interchanged.
The magnitude of the vector is unchanged but its direction reverses.
The subtraction \m{\vect{A} - \vect{B}} is illustrated in \Figref{4}.

Note also in \Figref{3} the difference \m{\vect{R} - \vect{B}} must be \vect{A}
since \m{\vect{A} + \vect{B} = \vect{R}}.
If the head and tail of \vect{B} are interchanged, the result is \m{-\vect{B}}
and that vector can then be added to \vect{R}: \m{\vect{R} + (-\vect{B}) =
\vect{A}}.

\tryit Try it geometrically and see if you get \vect{A}!
}% /Sect
%
\Sect{3}{The Product of Two Vectors}{\SectType{TextMultiPara}}{
%
\pcap{3}{a}{Two Kinds of Products}
There are two kinds of multiplication possible for two vectors: one produces a
scalar result while the other produces a vector result.
Important physical quantities are usually obtained by such multiplication.
For example, work is often obtained as a \textit{scalar} product of force and
displacement vectors, while angular momentum is often obtained as a \textit{vector}
product of radius and momentum vectors.
Each of these kinds of multiplication has its own rules.

\CaptionedFullFramedFigure{5}{Two rules for
finding the direction of \m{\vect{C} = \vect{A} \times \vect{B}} by:
(a) the \Quote{right hand} rule; (b) the \Quote{screw} rule.}{m2gr05}

\pcap[\Index{product, scalar}\Index{scalar product}]{3}{b}{The Scalar Product}
The scalar product of two vectors \vect{A} and \vect{B} is the
magnitude of \vect{A} times the magnitude of \vect{B} times the cosine of
angle between them (this angle is usually denoted \m{\theta}).
The scalar product is sometimes called the \Quote{\Index{dot product}dot product} because of the
way it is indicated visually:
%
\Eqn{3}{\text{scalar product } \equiv \vect{A} \cdot \vect{B} = A B \cos\theta.}
%
\tryit Show that the scalar product of two vectors is zero if the two are
perpendicular.

\tryit Under what other circumstances will the scalar product be zero?
\help{4}

\tryit Let \m{\vect{R} = \vect{A} + \vect{B}} and let \m{\theta} be the angle between
\vect{A} and \vect{B}.
Use \m{R = \sqrt{R^2} = \sqrt{\vect{R} \cdot \vect{R}}} to show that
\m{R = \sqrt{A^2 + B^2 + 2AB\cos\theta}}. \help{5}
Note: Some professionals use this method to produce the \Quote{adjacent sides and
included angle} trig formula.

\pcap[\Index{product, vector}\Index{vector product}]{3}{c}{The Vector Product}
The vector product of, say, \vect{A} and \vect{B}, is itself a
vector and we will show separately how to find its \Index{magnitude, of vector product}magnitude and direction.
The vector product is sometimes called the \Quote{\Index{cross product}cross product} because of the way
it is indicated visually:
%
\Eqn{}{\vect{C} = \vect{A} \times \vect{B}.}
%
The magnitude of \vect{C} is defined as magnitude of \vect{A} times the
magnitude of \vect{B} times the sine of the angle between them, as measured
counterclockwise from the direction of \vect{A} to the direction of \vect{B}:
%
\Eqn{4}{|\vect{C}| = |\vect{A} \times \vect{B}| = A B \sin\theta.}
%
The direction of the vector \vect{C} is perpendicular to the plane formed by
\vect{A} and \vect{B} and that narrows it to one of two directions.
We will here describe two rules for obtaining the exact direction.

\noindent The \Quote{\Index{right-hand rule}right-hand rule}:
\begin{itemize}
\item [1.] Extend the index finger of your right hand in the direction of
the first vector (\vect{A} in the example) with the rest of the fingers closed.
\item [2.] Rotate the index finger and hand until the index finger aligns with
the second vector (\vect{B} in the above example).
That is, the first of the two vectors denoted in the cross product is rotated
toward the second through the smaller angle between them (\vect{A} rotated
toward \vect{B} such that \m{\theta} in \Figref{5} decreases) and the curled
fingers of the right hand follow this rotation.
\item [3.] The direction of the extended thumb gives you the direction of
\vect{C} (see \Figref{5}).
\end{itemize}

\noindent The \Quote{screw} rule:
\begin{itemize}
\item [1.] Imagine placing vectors \vect{A} and \vect{B} tail-to-tail,
as in \Figref{5}.
\item [2.] Imagine a screw with right hand threads placed where the tails of
the two vectors come together, with the axis of the screw perpendicular to the
plane formed by the two vectors (see \Figref{5}).
\item [3.] Imagine a screwdriver placed in the screw with the axis of the
screwdriver being along the axis of the screw.
\item [4.] Imagine a spot on the screw, next to vector \vect{A}.
\item [5.] Turn the screwdriver through the shortest angle so the spot is
now next to vector \vect{B}.
The direction the screw went (in or out) is the direction of the product vector
\vect{C} (see \Figref{5}).
\end{itemize}

\tryit Show that the vector product of two vectors is \Index{zero vector product}zero if the two are
parallel.

\tryit Show that the vector product of a vector with itself is zero.

}% /Sect
%
\Sect{4}{Using Cartesian Components}{\SectType{TextMultiPara}}{
%
\CaptionedFullFramedFigure{6}{Cartesian components of a vector.}{m2gr06}

\enlargethispage{.5pc}

\pcap{4}{a}{The Components}
You can use the \Index{cartesian components, of vector}\Index{vector, cartesian components of}Cartesian components of vectors to add and subtract the vectors
and to take their scalar and vector products.
Here is the process for obtaining the components of vectors so the
\Index{components, vector}\Index{vector components}components can be combined according to the rules for addition, subtraction,
and scalar and vector multiplication.
For two vectors, say \vect{A} and \vect{B} (refer to \Figref{6} in each step):
\begin{itemize}
\item [1.] Draw a set of Cartesian (mutually perpendicular) \Index{axes, coordinate}\Index{coordinate axes}coordinate axes
and label them \m{x}, \m{y}, and \m{z}, as usual. \help{20}
Choose the origin and orientation of the axes for convenience in the problem
being tackled (this will become automatic with experience).
The results will be independent of your choice of axes.
\item [2.] Draw lines from each end of both \vect{A} and \vect{B} to the
axes, such that these lines are perpendicular to the axes.
Thus for the two vectors \vect{A} and \vect{B} there will be four such lines
to each of the three axes (12 lines in all).
\item [3.] Label \m{A_x} the distance along the \m{x}-axis between the two
perpendiculars from the ends of \vect{A}.
Label \m{A_y} and \m{A_z} the similar distances along the \m{y}- and \m{z}-axes for
\vect{A}.
Do the same for \vect{B}, labeling the projections along the axes
\m{B_x}, \m{B_y}, and \m{B_z}.
\item [4.] Call \m{A_x}, \m{A_y}, and \m{A_z} the \textit{components} of \vect{A} and \m{B_x},
\m{B_y}, and \m{B_z} the \textit{components} of \vect{B}.
\end{itemize}
The components of a vector are frequently presented as a \Index{triad}triad of numbers, in
the specific order: \m{x}-component, \m{y}-component, \m{z}-component.
Thus for \vect{A} we write:
%
\Eqn{5}{\vect{A} = (A_x,\,A_y,\,A_z).}
%

\tryit Find the numerical values of the components of the vector in
\Figref{6}. \help{6}

\pcap[\Index{vector addition}\Index{vector components, addition of}\Index{addition, of vector components}]{4}{b}{Addition}
If the two vectors to be added are \vect{A} and \vect{B}, with components
\m{(A_x,\,A_y,\,A_z)} and \m{(B_x,\,B_y,\,B_z)}, then the resultant vector
\vect{R} has components \m{(A_x + B_x, A_y + B_y, A_z + B_z)}.
That is,
%
\ThreeEqns{6}{R_x & = A_x + B_x\,,}
             {R_y & = A_y + B_y\,,}
             {R_z & = A_z + B_z\,.}
%
\Index{vector sum}\Index{sum, of two vectors}Thus if the Cartesian components of two vectors are known, the Cartesian
components of the resultant are easy to compute.
The verification of this equation is left to an Appendix.

\tryit Vector \m{\vect{A} = (1,0,1)} while \m{\vect{B} = (0,1,-1)}.
Determine the components of \m{R = \vect{A} + \vect{B}}. \help{7}

\pcap[\Index{subtraction, of vector components}]{4}{c}{Subtraction}
If \vect{B} is to be subtracted from \vect{A} to produce \vect{C},
%
\Eqn{}{\vect{C} = \vect{A} - \vect{B}}
%
we just use the rules of addition but with the sign of the components of \vect{B} made negative:
%
\ThreeEqns{7}{C_x & = A_x - B_x\,,}
             {C_y & = A_y - B_y\,,}
             {C_z & = A_z - B_z\,.}

\tryit Example: Vector \m{\vect{A} = (1,0,1)} while \m{\vect{B} = (0,1,-1)}.
Determine the components of \m{R = \vect{A} - \vect{B}}. \help{8}

\pcap[\Index{product, scalar}\Index{scalar product}]{4}{d}{Scalar Product}
In terms of Cartesian components, the scalar or \Quote{dot} product of \m{A} and \m{B}
to form \m{C} is:
%
\Eqn{8}{C = \vect{A} \cdot \vect{B} = A B \cos\theta =
       A_x B_x + A_y B_y + A_z B_z.}
%
\tryit \Index{zero scalar product}Vector \m{\vect{A} = (1,0,1)} while \m{\vect{B} = (0,1,-1)}.
Use Cartesian components to determine: \m{R = \vect{A} \cdot \vect{B}}. \help{9}

\pcap[\Index{vector product}\Index{product, vector}\Index{cross product}]{4}{e}{Vector Product}
The six remaining products of the \Index{vector product, in terms of components}components of two
vectors may be combined so that they form the three components of another vector which is used
in almost all areas of physics and technology.
This new vector is called the \textit{vector product} of the two combining vectors and is written with
a multiplication sign.
Here is how the vector \vect{C} is written as the vector product of \vect{A} and \vect{B}:
%
\Eqn{20}{\vect{C} = \vect{A} \times \vect{B}\,.}
%
In terms of Cartesian components:
%
\ThreeEqns{9}{C_x & = A_y B_z - A_z B_y}
             {C_y & = A_z B_x - A_x B_z}
             {C_z & = A_x B_y - A_y B_x}
%
The mnemonic for remembering the order of the subscripts on these components
is to note that, starting from left to right, the first three subscripts
are always cyclic permutations of \m{xyz} (\m{xyz}, \m{yzx}, \m{zxy}).
%
\Footnote{4}{Note that the combination \m{A_y B_z - A_z B_y} is the \m{x}-component of this
vector, expressed with respect to the same coordinate system.
Two other pairs of products complete the groupings (and, incidentally,
together with the combinations that form the scalar product, exhaust the
possibilities of pairs of products of components of \m{A} and
\vect{B}.}
%

\tryit Vector \m{\vect{A} = (1,0,1)} while \m{\vect{B} = (0,1,-1)}.
Determine the components of \m{\vect{R} = \vect{A} \times \vect{B}}. \help{10}
}% /Sect
%
\Sect{5}{Using Unit Vectors}{\SectType{TextMultiPara}}{
%
\pcap{5}{a}{Cartesian Unit Vectors}
A vector can be written as the sum of its components through the use of
\Quote{\Index{vector, unit}\Index{unit vector}unit vectors,} which are three vectors
of unit length that each point outward
along one of the three Cartesian coordinate axes you are using.
We write these unit vectors as \m{\uvec{x}}, \m{\uvec{y}} and \m{\uvec{z}},
where \m{\uvec{x}} is a vector of unit length that points in the direction of
increasing values along the \m{x}-axis, \m{\uvec{y}} similarly along the \m{y}-axis,
and \m{\uvec{z}} along the \m{z}-axis.
%
\Footnote{5}{Different authors use different notations for unit vectors along
the axes of a Cartesian coordinate system.
We use \m{\uvec{x}}, \m{\uvec{y}} and \m{\uvec{z}} for unit vectors along the
\m{x}-, \m{y}-, and \m{z}-axes, which is a commonly used notation.
Other authors have used such notations as:
(\m{\uvec{i}}, \m{\uvec{j}}, \m{\uvec{k}}),
(\textbf{i}, \textbf{j}, \textbf{k}),
(\m{\uvec{e}_1}, \m{\uvec{e}_2}, \m{\uvec{e}_3}),
and (\m{\uvec{u}_x}, \m{\uvec{u}_y}, \m{\uvec{u}_z}).
All such sets are equivalent.}
%
Since the unit vectors are of unit length, a vector \vect{A} with components
\m{(A_x,A_y,A_z)} can be written as the sum of three vectors:
%
\Eqn{10}{\vect{A} = A_x \uvec{x} + A_y \uvec{y} + A_z \uvec{z}.}
%
This addition is illustrated for two dimensions in \Figref{7}.

\pcap{5}{b}{Products Using Cartesian Unit Vectors and Components}

\tryit Use the definitions of products of vectors, \Eqnsref{8} and
\Eqnssref{20}, to show that: (\help{19})
%
\FourEqns{12}%
{\uvec{x} \cdot \uvec{x}  & = 1, \qquad \uvec{y} \cdot \uvec{y} = 1, \qquad \uvec{z} \cdot \uvec{z} = 1,}
{\uvec{x} \cdot \uvec{y}  & = 0, \qquad \uvec{x} \cdot \uvec{z} = 0, \qquad \uvec{y} \cdot \uvec{z} = 0,}
{\uvec{x} \times \uvec{y} & = \uvec{z}, \qquad \uvec{y} \times \uvec{z} = \uvec{x}, \qquad \uvec{z} \times \uvec{x} = \uvec{y},}
{\uvec{x} \times \uvec{z} & = - \uvec{y}, \qquad \uvec{x} \times \uvec{x} = 0, \text{ etc.}}
%
Note that, in the vector product, the order (\m{xyz}), in any cyclic
permutation, has a plus sign on the right hand side.
Any other order has a minus sign.
%
\Footnote{6}{This provides a test for whether a Cartesian coordinate system
is \textit{right-handed}, which is what we use throughout science and engineering.
An improperly defined, \textit{left-handed}, system would produce
\m{\uvec{x} \times \uvec{y} = - \uvec{z}}.}
%
\help{11}

\CaptionedFullFramedFigure{7}{Illustration of combining unit vectors and a
vector's components.}{m2gr07}

\pcap{5}{c}{The Length of a Vector in Terms of its Components}
To find the length of a vector \vect{A} write:
 %
%
 \Eqn{}{\vect{A} = A_x \uvec{x} + A_y \uvec{y} + A_z \uvec{z}}
%
 %
 and take the scalar product of \vect{A} with itself:
 %
\TwoEqns{}%
     {\vect{A} \cdot \vect{A} & = (A_x \uvec{x} + A_y \uvec{y} + A_z \uvec{z})
                                 \cdot (A_x \uvec{x} + A_y \uvec{y} + A_z \uvec{z})}
     {                      & = A_x^2 + A_y^2 + A_z^2\,.}
%
 where we have used the products of unit vectors, \Eqnref{12}.
 
 Now from the definition of the scalar product, \Eqnref{8}, we
%
 know that \m{\vect{A} \cdot \vect{A} = A^2}, which is the square of the length
 of vector \vect{A}.
 Then the length of \m{A} is given in terms of its components by the square
 root of the product of \m{A} with itself, \m{\vect{A} \cdot \vect{A}}:
 %
%
 \Eqn{13}{A = \sqrt{A_x^2 + A_y^2 + A_z^2}\,.}
%
\tryit Show that: \m{|\vect{A} \times \vect{B}| = |\vect{B} \times \vect{A}|}. \help{17}

\tryit Show that: \m{\vect{A} \times \vect{B} = - \vect{B} \times \vect{A}}. \help{18}

\pcap{5}{d}{Determinant Form of the Vector Product}
Unit vectors can be used to write the vector product in determinant form.
Some people find it easier to remember these forms:
%
\Eqn{11}{\vect{C} = \vect{A} \times \vect{B} =
\left|
\begin{array}{c c c}
\uvec{x} &  \uvec{y} & \uvec{z} \\
A_x   &  A_y   & A_z   \\
B_x  &  B_y   & B_z   \\
\end{array}
\right|}
%
The minors of the unit vectors as co-factors are the appropriate components.
\help{16}

\tryit Use the determinant method to show that \m{\uvec{x} \times \uvec{y} =
\uvec{z}}. \help{12}

\tryit Given \m{\vect{A} = 5 \uvec{x} - 2 \uvec{y}} and \m{\vect{B} = \uvec{x} +
\uvec{y} + 3\uvec{z}}), show that \m{\vect{A} \times \vect{B} = -6 \uvec{x} -
15 \uvec{y} + 7 \uvec{z}}. \help{13}

\pcap{5}{e}{Representing a Vector by Magnitude and Direction}
It is often convenient to write a vector as the product of a single magnitude and
a single direction rather than as the sum of its Cartesian Components.
First we define the magnitude of a vector:
%
\Eqn{14}{A \equiv |\vect{A}| = \sqrt{A_x^2 + A_y^2 + A_z^2}\,.}
%
Note that the magnitude it always positive.
Next we define a unit vector that is in the direction of the vector we are representing:
%
\Eqn{15}{\uvec{A} \equiv \dfrac{\vect{A}}{|\vect{A}|}\,.}
%
Note that \m{\uvec{A}} has unit length because we have divided \vect{A} by its own length.
Then we can rewrite \Eqnref{15} to show the vector as the product of a
magnitude and a direction, which is entirely equivalent to the representation of the
vector in Cartesian components:
%
\Eqn{16}{\vect{A} = A \uvec{A} = A_x \uvec{x} + A_y \uvec{y} + A_z \uvec{z}\,.}
%
Finally, we write the magnitude and direction of the vector product \vect{C} of two vectors
\vect{A} and \vect{B} in terms of the magnitudes and directions of \vect{A} and \vect{B}.
First, the magnitude is:
%
\TwoEqns{17}{C & = \left[(A_y B_z - A_z B_y)^2 + (A_z B_x - A_x B_z)^2 + (A_x B_y - A_y B_x)^2\right]^{1/2}}
            {  & = A B \sin{\theta}\,,}
%
where \m{\theta} is the counter-clockwise angle between the directions of \vect{A} and \vect{B}.
The direction of the vector product is:
%
\Eqn{18}{\uvec{C} = \dfrac{\vect{C}}{C} = \dfrac{\vect{A} \times \vect{B}}{|\vect{A} \times \vect{B}|}
                  = \dfrac{(A_y B_z - A_z B_y) \uvec{x} + \ldots}{\left[(A_y B_z - A_z B_y)^2 + \ldots \right]^{1/2}}\,.}
}% /Sect
%
\Sect{}{Acknowledgments}{\SectType{Acknowledgments}}{
Professor James Linneman made several helpful suggestions. Ashlee McFarland pointed out a typo.
\NsfAcknowledgment
}% /Sect

