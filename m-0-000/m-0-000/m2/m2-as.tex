\revhist{9/13/89, pss; 7/30/91, pss; 9/16/93, pss; 10/19/94, pss; 8/16/95, pss; 5/2/02, pss;
         12/11/02, pss; 12/13/02, pss}

\Sect{}{}{\SectType{SpecialAssistance}}{

\AsItem{1}{TX-1b}
{Using a ruler, we find the \vect{F} vector to be 0.5\unit{in} long so
\m{F = 10\unit{lb/in} \times 0.5\unit{in} = 5\unit{lb}}.}

\AsItem{2}{TX-2b}
{The easiest way is to put a sheet of paper over \Figref{2} and put both sheets
together against a window pane that is illuminated mainly from the other side.
Trace the first resultant onto the clean sheet, then slide it over to the
other resultant and see if the magnitude and direction are identical (up to
your ability to trace and compare visually).
}

\AsItem{3}{TX-2c}
{In \Figref{3}, measure \m{A}, \m{B}, and \m{R} with a ruler and measure the angles
\m{\theta} and \m{\phi} with a protractor.
Then use your calculator to finish the job, making sure you have set it for
the proper choice of radians or degrees.
}

\AsItem{4}{TX-3b}
{The magnitude of one or both is zero.
}

\AsItem{5}{TX-3b}
{\Eqn{}{R = \left[(\vect{A} + \vect{B}) \cdot (\vect{A} + \vect{B})\right]^{1/2} =
\left(A^2 + B^2 + 2\vect{A} \cdot \vect{B}\right)^{1/2}}
}

\AsItem{6}{TX-4a}
{We find: \m{A_x = 1.00\unit{inches}}, \m{A_y = 0.75\unit{inches}}.
}

\AsItem{7}{TX-4b}
{\Eqn{}{(1,\,0,\,1) + (0,\,1,-1) = (1,\,1,\,0)}
}

\AsItem{8}{TX-4c}
{\Eqn{}{(1,\,0,\,1) - (0,\,1,-1) = (1,-1,\,2)}
}

\AsItem{9}{TX-4d}
{\Eqn{}{(1,\,0,\,1) \cdot (0,\,1,-1) = 1 \cdot 0 + 0 \cdot 1 + 1 \cdot (-1) = -1}
}

\AsItem{10}{TX-4e}
{\Eqn{}{(1,\,0,\,1) \times (0,\,1,-1) = (-1,\,1,\,1)}
}

\AsItem{11}{TX-5a}
{Use the \Quote{right hand} rule or the \Quote{screw} rule.
You could also work with components: \m{\uvec{x} = (1,0,0)}, \m{\uvec{y} = (0,1,0)},
and \m{\uvec{z} = (0,0,1)}.
Then, for example, the \m{x}-component of \m{\uvec{x} \times \uvec{y}} is:
\m{(\uvec{x} \times \uvec{y})_x = (\uvec{x})_y (\uvec{y})_z -
(\uvec{x})_z (\uvec{y})_y = 0 \cdot 0 - 0 \cdot 1 = 0}.
}

\AsItem{12}{TX-5b}
{\Eqn{}{\uvec{x} \times \uvec{y} =
\left|
\begin{array}{c c c}
\uvec{x} &  \uvec{y} & \uvec{z} \\
(\uvec{x})_x & (\uvec{x})_y & (\uvec{x})_z \\
(\uvec{y})_x & (\uvec{y})_y & (\uvec{y})_z \\
\end{array}
\right|
 =
\left|
\begin{array}{c c c}
\uvec{x} &  \uvec{y} & \uvec{z} \\
   1    &     0    &    0    \\
   0    &     1    &    0    \\
\end{array}
\right|
}
}

\AsItem{13}{TX-5b}
{\Eqn{}{
\left|
\begin{array}{c c c}
\uvec{x} &  \uvec{y} & \uvec{z} \\
   5    &    -2    &    0    \\
   1    &     1    &    3    \\
\end{array}
\right|
}
}

\AsItem{14}{PS, problem~12b}
{If you get
 %
 \Eqn{}{C^2 = A^2 + B^2 + 2 A B \cos\gamma}
 %
 then you did not pay  attention to the definition of \m{\gamma} in the
 statement of the problem.
 The definition of \m{\gamma} is taken from the usual statement of the
 relationship between two sides and the included angle in a triangle.
 \smallskip

 If you do not know how to obtain \m{\cos(\pi - \gamma)} from
 \m{\cos(\gamma)}, see \Quote{trigonometric functions} in this volume's
 \textit{Index}.
}

\AsItem{15}{PS, problem~7a}
{Look at the figure in the problem statement.
 To find \m{C_x}, drop a vertical line from the tip of \vect{C} to
 the \m{x}-axis, so the line is perpendicular to the \m{x}-axis.
 Label this line \m{C_y} and label the \m{x}-axis from the origin out to this
 line \m{C_x}.
 Notice the right-angle triangle formed by the three lines \m{C}, \m{C_x}, and
 \m{C_y} (the vertical line down from the tip of \vect{C}).
 You know the values of \m{C} and \m{\phi} so simply apply trigonometry to
 find \m{C_x}.
 If you don't remember how, see \Quote{trigonometric functions} in this
 volume's \textit{Index}.
}

\AsItem{16}{TX-5d}
{Answer (which also gives you the way a \m{3 \times 3} determinant is evaluated):
 \Eqn{}{\vec{A} \times \vec{B} = \uvec{x}(A_yB_z - A_zB_y) + \uvec{y}(A_zB_x - A_xB_z)
            + \uvec{z}(A_xB_y - A_yB_x)}
}

\AsItem{17}{TX-5c}
{You can show it any of three ways.
We suggest that you do all three and then choose the method that suits you best.
The ways are:
(1) by expanding \m{|\vect{A} \times \vect{B}|} and \m{|\vect{B} \times \vect{A}|}
in Cartesian components and showing that the two are equal (the absolute magnitude of
a vector is given in \Eqnref{13}); or
(2) by showing that the answer is unchanged if you interchange \vect{A} and \vect{B} in \Eqnref{4}; or
(3) by taking the absolute magnitude of each side of
\m{\vect{A} \times \vect{B} = - \vect{B} \times \vect{A}}, in which case the
absolute magnitude of the right side deletes the over-all minus sign and the proof is complete.}

\AsItem{18}{TX-5c}
{You can show it either of two ways.
We suggest that you do both and then choose the method that suits you best.
The ways are:
(1) by expanding the cross-products in Cartesian components; or
(2) by using the fact that \m{\vect{A} \times \vect{B}} has the same magnitude as
\m{\vect{B} \times \vect{A}} (see \textit{S-17})), in which case the two can differ only by direction,
and then using the \Quote{tight-hand rule} or the \Quote{screw} rule to show that
the direction of either one is opposite to that of the other.}

\AsItem{19}{TX-5b}
{As an example, we treat the case \m{\uvec{x}\cdot\uvec{x}=1}:
\smallskip

Method 1. Apply the second method of \Eqnref{8} to the case at hand so \m{\vec{A} = \vec{B} = \uvec{x}}.  Since \uvec{x} is a unit vector in the \m{x}-direction, \m{A_x = B_x = 1} and \m{A_y = B_y = A_z = B_z = 0}.  Then, by \Eqnref{8}, \m{\uvec{x}\cdot\uvec{x}= (1\cdot1) + (0\cdot0) + (0\cdot0) = 1}.
\smallskip

Method 2. Apply the first method of \Eqnref{8} to the case at hand so \m{A = B = 1} and \m{\cos{\theta} = 1} so \m{\uvec{x}\cdot\uvec{x} = 1\cdot1\cdot1 = 1}.

}

\AsItem{20}{TX-4a}
{The three mutually perpendicular axes are usually set up with the position of the origin and the
directions of the axes chosen to make the mathematics of the problem as simple as possible.  For
example, the motion in a problem may be entirely in a plane, whereupon one would normally choose
the \uvec{x} and \uvec{y} axes to be in that plane.  Then there would be no motion in the \uvec{z}
direction and the \m{z}-coordinate could be ignored (in that problem).
\smallskip

Any two non-colinear (not parallel) vectors define a plane and the vector (\Quote{cross}) product of
two such vectors is perpendicular to that plane.
It takes two dimensions to define points in the plane of the two vectors and a third dimension to define
points out of that plane.
Thus all vector products need to use three-dimensional coordinate systems.
\smallskip

In the top center sketch in \Figref{5}, imagine placing the \m{x}-axis in the direction of \vect{A}.
Then imagine the plane that goes through both vectors \vect{A} and \vect{B}, indicated by the loop in the
sketch.
Imagine drawing the \m{y}-axis in that plane, perpendicular to the \m{x}-axis (approximately where
\vect{B} is shown in the sketch).
Finally, imagine drawing the \m{z}-axis perpendicular to the \m{x}- and \m{y}-axes, along the upward
vertical labeled \vect{C} in the sketch.
In this Cartesian coordinate system, any point in space can be uniquely described by its
\m{x}-, \m{y}-, and \m{z}-coordinates.
}

}% /Sect
