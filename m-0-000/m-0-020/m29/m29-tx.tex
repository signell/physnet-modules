\revhist{4/1/91, pss; 10/11/94, pss; 2/5/99 bds; 10/3/02, pss; 10/21/02, pss}
%
\Sect{1}{Introduction}{\SectType{TextMultiPara}}{
%
\CaptionedFullFramedFigure{1}{A simple damped oscillator.}{m29gr01}
%
\pcap{1}{a}{Damping is Universal}
In this module we add damping, or kinetic energy dissipation, to the case of
simple harmonic motion.
This is important because such dissipation is always present in real mechanical
systems.
In addition, we often wish to design the rate of dissipation in order to damp
out unwanted oscillatory motions.
An example of this is the up and down motion of a car wheel after it goes
over a bump; without damping the wheel would continue to oscillate up and
down indefinitely.

\pcap{1}{b}{An Example}
In \Figref{1} we show a simple damped oscillatory system consisting of a spring,
a mass, and a damping device (a device that dissipates the kinetic energy of the
oscillator).
Without the damping device, the mass will oscillate indefinitely.
With the device, its motion gradually dies out.

Inside the cylindrical damping device there is a piston immersed in a fluid.
This fluid resists the motion of the piston through it.
This resistance can be controlled by varying the viscosity of the fluid or by
otherwise varying its ability to bypass the piston.

\TextAndFigurePage{%
\pcap{1}{c}{Varying the Resistance to Motion}
If we gradually increase the ability of the damping device in \Figref{1} to resist
motion, we get the series of curves shown in \Figref{2}.
Each curve shows the displacement of the oscillator from its position of static
equilibrium as a function of time.

The first curve is with no damping.
Succeeding curves are for increasing resistance of the damping device to
motion, such as by making the fluid go from water to light oil to heavy oil to
thick molasses (the quantity \m{\gamma} is a measure of this increasing
resistance).
Note that a point is reached beyond which there are {\em no} oscillations.
This transition point, called the {\em critical damping} point, occurs at
\m{\gamma = 2 \pi f_0}.

\pcap{1}{d}{Prospectus}
In the rest of this module we will first treat each forces acting on the
mass separately, then combine the forces.
This means we will review the restoring force that produces simple harmonic
motion, then introduce the damping force and see what it does to the motion.
In the process we will make quantitative the ideas we have discussed.
This will provide an understanding of a wide variety of cases and devices.
It will also lay the ground work for considering damped {\em driven} oscillatory
motion, wherein energy is continually fed into the system as well as being
dissipated from it.
%
The addition of the driving force is treated elsewhere.%
%
\Footnote{1}{See \Quote{Damped Driven Oscillatory Motion,} (MISN-0-30).}%
}{2}{\Figref{2}. (see text).}{m29gr02}%
%
}% /Sect
%
\Sect{2}{The Forces}{\SectType{TextMultiPara}}{
%
\pcap{2}{a}{The Restoring Force}
The restoring force is the force that, at all times, accelerates the oscillator
back toward the position of static equilibrium.
If the displacement is sufficiently small for the case at hand, the restoring
force will be linear to a good approximation so we write it:
%
\Eqn{1}{F = -\,kx\,.}
%
A force that obeys \Eqnref{1}, with an interpretation of \m{k} appropriate for that
force, is called a \Quote{Hooke's law force.}
If it is the only force acting on a mass, the mass will undergo Simple Harmonic
Motion (hereafter referred to as \Quote{SHM}):
%
\Eqn{2}{x(t) = x(0)\,\cos{(\omega_0 t + \alpha)}\,.}
%
\tryit Substitute \Eqnsref{2} and \Eqnssref{1} into
\m{F=ma} and show that the equation is satisfied if the constant \m{\omega_0}
is equal to the correct combination of \m{m} and \m{k}.
\help{1}\,
%
\Footnote{2}{If you need help in this, see sequence [S-1] in this module's
\textit{Special Assistance Supplement}.}

\pcap{2}{b}{The Damping Force}
A \Quote{damping force} is one that, acting by itself, (smoothly) stops motion.
For example, friction acting on a horizontally coasting bicycle or car is a
damping force because it gradually brings the vehicle to a halt.

A damping force must oppose the velocity in order that the resultant
acceleration is a deceleration.
This means that, mathematically, the damping force has the opposite sign
to the velocity.
The damping force cannot be a constant since that would decrease the velocity
through zero and then cause it to increase in the direction of the force,
whereas the damping force (by itself) is to make the motion stop and then
stay stopped.
If the damping is sufficiently small for the case at hand, it is likely to
be linear to a good approximation:
%
\Eqn{3}{F = -\lambda v\,.}
%
If this is the only force acting on a mass, the velocity will decrease
to zero exponentially in time:
%
\Eqn{7}{v(t) = v(0)\,\exp{(-\gamma t)}\,.}
%

\tryit Substitute \Eqnsref{7} and \Eqnssref{3} into
\m{F=ma} and show that the equation is satisfied if the constant \m{\gamma}
is equal to the correct combination of \m{\lambda} and \m{m}. \help{2}
Since the differential equation is second order and homogeneous, this
solution is the unique solution (apart from transformations to mathematically
equivalent functions).

\tryit Visualize \Eqnref{7} as a graph and think of how the
curve on it changes as the damping constant \m{\lambda} and mass \m{m} are
increased and decreased. \help{3}

\tryit Integrate \Eqnref{3} to give:
\m{x(t) = x(0) - [v(0)/\gamma]\,\exp{(-\gamma t)}}.

\pcap{2}{c}{The Two Forces Together}
A linearly-damped linearly-restored oscillator is one in which the force
acting on the mass is the sum of \Eqnsref{1} and \Eqnssref{2}:
%
\Eqn{5}{F = -\,kx\,-\lambda v\,.}
%
In this case we will write the solution differently depending on whether
\m{\lambda} is smaller, equal to, or larger than the combination \m{2\sqrt{mk}}.
These three cases are referred to as \Quote{underdamped,} \Quote{critically damped,}
and \Quote{overdamped.}
We could write a single solution for all three cases but it would involve
complex variables.
To avoid that, we separate the three cases:
\renewcommand{\tabcolsep}{2pt}
\begin{center}\begin{tabular}{r c c l}
underdamped:       \hspace{4pt} & \m{\lambda} & \m{<} & \m{2\sqrt{mk}} \\
critically damped: \hspace{4pt} & \m{\lambda} & \m{=} & \m{2\sqrt{mk}} \\
overdamped:        \hspace{4pt} & \m{\lambda} & \m{>} & \m{2\sqrt{mk}} \\
\end{tabular}\end{center}

\tryit Note that the progression in the names matches the progression in the
size of the damping constant \m{\lambda}.
}% /Sect
%
\Sect{3}{The Damped Oscillator Solutions}{\SectType{TextMultiPara}}{
%
\pcap{3}{a}{The Underdamped Solution}
For the underdamped case, the solution to \m{F=ma} can be written in the form:
%
\Eqn{6}{x(t) = A\,e^{-\gamma t}\,\cos{(\omega t + \alpha)}\,.}
%
\noindent Here \m{A} and \m{\alpha} must be fixed by two known values of \m{x}, or
its derivatives, at one or two specific times.
For example, \m{A} and \m{\alpha} could be fixed by knowing \m{x} at two different
times or by knowing \m{x} at a specific time and the velocity \m{v} at a specific
time.

\tryit Substitute \Eqnref{6} and \Eqnref{5} into
\m{F=ma} and show that the equation is satisfied if the constants \m{\gamma}
and \m{\omega} are equal to the correct combinations of \m{k}, \m{\lambda}, and \m{m}:
\help{4}
%
\Eqn{8}{\gamma = \lambda/(2 m)\,,}
%
\Eqn{9}{\omega = \sqrt{\omega_0^2 - \gamma^2}\,,}
%
where we define the undamped frequency, \m{\omega_0}, as usual:
%
\Eqn{10}{\omega_0 = \sqrt{k/m}\,.}
%
Note that the condition for the underdamped case can be written:
\m{\lambda < \omega_0}.

\tryit Sketch a graph of \Eqnref{6}.
Note that the frequency of the oscillations, \m{\omega}, remains constant in time
but that the amplitude of the oscillations dies out exponentially.
To make the sketch, first construct positive and negative decreasing
exponential functions on the graph using dashed lines.
The result should be a graph that is symmetric under reflections about the time
axis.
Now draw the constant-frequency exponentially-dying oscillations inside the
dashed-lines \Quote{envelope.} \help{5}

\pcap{3}{b}{The Overdamped Solution}

For the overdamped case, \m{\gamma > \omega_0}, the solution can be written in
the form:
%
\Eqn{11}{x(t) = A'\,e^{-(\gamma + \omega ')t} + B'\,e^{-(\gamma - \omega ')t}\,.}
%
where \m{\gamma} and \m{\omega_0} are as before and:
%
\Eqn{19}{\omega ' = \sqrt{\gamma^2-\,\omega_0^2}\,.}
%
As in the underdamped case, \m{A'} and \m{B'} must be fixed by two known values of
\m{x(t)} and/or its derivatives.
\Eqnref{11} can be shown to be mathematically equivalent to
%
\Eqnref{6}.
%
\Footnote{3}{If you are interested, see this module's \textit{Appendix}.}

\tryit  Sketch a graph of \Eqnref{11}.
Note that the curve is just the sum of two decaying exponentials.
To the inexperienced eye, the sketch will look more or less like a single
decaying exponential.

\tryit Make sketches of what happens to the curve as you increase the damping
constant to make it farther and farther from \Quote{critical} damping.

\pcap{3}{c}{The Critically Damped Solution}

For the critically damped case, \m{\omega = \omega ' = 0}.
Using this, we can create a simpler form of \Eqnref{6} or
\Eqnref{11} by letting \m{\omega} or \m{\omega '} approach zero in either
%
of those equations.
%
\Footnote{4}{If you are interested in the details, see this module's
\textit{Appendix}.}
%
This gives:
%
\Eqn{12}{x(t) = (A'' + B'' t)\,e^{-\gamma t}\,.}
%
Again, \m{A''} and \m{B''} must be fixed by two known values of \m{x(t)} and/or its
derivatives.

\tryit  Sketch a graph of \Eqnref{12}.

\pcap{3}{d}{The Transition}
In \Figref{2} we have plotted \m{x(t)}, \Eqnsref{6} and \Eqnssref{11}, for increasing
values of the damping constant \m{\lambda}.
The critically damped case is not shown but corresponds to:
\m{\gamma = \omega_0 = 2 \pi f_0}.
We have chosen the constants such that \m{\alpha = 0}.

Think of the system shown in the figure as the wheel of a car that
has just gone over a sharp bump.
If the shock absorber is worn out and hence inoperative, the damping is zero
and the wheel oscillates up and down as in the first figure.
If the shock absorber is almost worn out, so the damping is small, the wheel
responds as in the next figure.
If the shock absorber produces too much damping, the wheel takes a long time
to come back to the equilibrium position, as shown in the last figure.
Obviously, critical damping is close to what one wants.
It is also obvious that the shock absorbers must be matched to the weight of
the car and to the stiffness of its springs.
}% /Sect
%
\Sect{}{Acknowledgment}{\SectType{Acknowledgments}}{
%
The author wishes to thank Jules Kovacs and Julie Junttila for helping to
put together a study guide for a previous version of this module.
Production was supported in part by grants from the National Science
Foundation and from IBM.
}% /Sect
%
\Sect{A}{Notes for Those Interested}{\SectType{AppendixMultiPara}}{
%
\xpcap{}{a}{The Equivalence of \Eqnsref{11} and \Eqnssref{6}}
To see that the two equations are equivalent, note that \m{\omega^2=-\omega^2}
so \m{\omega ' = i \omega} where \m{i = \sqrt{-1}}.
Make this substitution and finish the conversion using
\m{e^{ix} = \cos{x} + i \sin{x}}. \help{6}

\xpcap{}{b}{Derivation of \Eqnref{12}}
To show that \Eqnref{12} is the limit of
\Eqnref{6} as \m{\omega} approaches zero, use the
approximation: \m{e^x \approx 1 + x} for small \m{x}. \help{7}

\xpcap{}{c}{Uniqueness of the solution}
A fundamental existence theorem in mathematics guarantees that there is only
one solution to \m{F=ma} for a force that is linear in \m{x}, \m{v}, and/or \m{a},
providing that force contains no higher derivatives and providing the solution
contains two independently adjustable constants.
Thus any solution we find that satisfies \m{F=ma}, with the restrictions noted,
is the only solution.
That is why we we can show that the three forms are equivalent.

Although the solution is unique in a mathematical sense, it can also be thought
of as a two-parameter family of solutions, the two parameters being the two
adjustable constants.
For example, consider the plotted curve for an underdamped case, showing
oscillations inside a decaying-exponential envelope.
Assume the value zero for the phase \m{\alpha} so the curve is at its highest
value at the origin.
Now start continuously increasing \m{\alpha} and watch as the curve changes
shape, going through complete oscillational cycles down and up the vertical
axis at the origin as \m{\alpha} continues to increase.
Of course the constant \m{A} can also be varied and together the two variations
produce the two-parameter family of solutions for fixed \m{\lambda}, \m{k}, and
\m{m}.
}% /Sect
