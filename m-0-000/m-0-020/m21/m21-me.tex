\revhist{8/29/91, pss; 3/11/93, pss; 4/20/94, pss; 10/29/96, pss; 11/14/96, pss}

\Sect{}{}{\SectType{ModelExam}}{

\begin{one-digit-list}
\item [1.] See Output Skill K1 in this module's \textit{ID Sheet}.

\item [2.] A nonconstant force, \m{F}, directed vertically upward (the force
varies in magnitude as some unknown function of altitude) is exerted on an
object of mass \m{m}.
Under the action of this force, the object moves straight up from ground
level to a distance \m{h}, above the ground.
It starts at rest and when it gets to height \m{h} it is moving with speed \m{v}
upward.
\begin{one-digit-list}
\item [a.] Draw the free-body diagram of forces that act on this mass during
the course of its motion upward.
\item [b.] Write down the Work-Energy relation (in its original form without
potential energies) for this system, expressing the work of each of the
forces separately.
\item [c.] Evaluate the integrals that can be evaluated.
\item [d.] Draw the appropriate picture(s) for solving this problem using
conservation of energy.
\item [e.] Write the conservation of energy relation for this object.
\item [f.] How much work was done by the force \m{F} in lifting the mass to the
height \m{h}?
\end{one-digit-list}

\item [3.] A nail partially driven into a board, is driven in further by
dropping a mass onto it.
The mass, \m{m}, is released a distance \m{y_0} above the head of the nail and
drives the nail an additional distance \m{s} into the board.
Neglect the mass of the nail.
\begin{one-digit-list}
\item [a.] Defining the zero of potential at the top of the nail, find
the potential energy of the system (take the mass \m{m} as the system) at the
instant \m{m} is released.
\item [b.] What is the kinetic energy at that instant?
\item [c.] Find the potential energy after the mass has driven the nail the
additional distance \m{s} into the board.
\item [d.] What is the kinetic energy at that instant?
\item [e.] Write the conservation of energy relation for this problem.
\item [f.] How much work is done on the mass by the nail?
\item [g.] How much work is done on the nail by the mass?
\item [h.] Assuming that the nail exerts a constant force on the mass during
the interval that the nail is slowing the mass to a stop, determine the
magnitude of this force, \m{F_N}.
\end{one-digit-list}

\item [4.] \noindent \CenteredUnframedFixedFigure{m21gr02}\newline
An artillery shell fired from a gun at ground level leaves the muzzle of the
gun with a speed of 50\,m/s.
The shell strikes a direct hit at a site on the top of a hill which is at an
altitude of 45\,m above ground level.
\begin{one-digit-list}
\item [a.] Assuming no friction, use energy conservation to find the speed with
which the shell hits at the site.
\item [b.] If the shell strikes the ground at the site with a speed of 38
meters per second, what percentage of its original energy was lost (to
friction) by the shell in its flight?
\end{one-digit-list}
\end{one-digit-list}

\BriefAns

\begin{one-digit-list}
\item [1.] See this module's \textit{text}.

\item [2.] See Problem~20 in this module's {Problem Supplement}.

\item [3.] See Problem~21 in this module's {Problem Supplement}.

\item [4.] See Problem~22 in this module's {Problem Supplement}.
\end{one-digit-list}

}% /Sect
