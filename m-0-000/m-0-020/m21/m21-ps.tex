\revhist{12/20/88, ejdk; 2/27/91, pss; 8/29/91, pss; 2/26/93, pss; 4/20/94, pss
         10/2/95, pss; 11/14/96, pss; 11/13/97, pss; 1/5/01, pss}

\Sect{}{}{\SectType{ProblemSet}}{

\noindent Use: \m{g = 9.8\unit{m/s\up{2}}}

\noindent Use Conservation of Energy in each problem marked with an asterisk (*).

\noindent If no Reference Point is given, use the standard one for that
type of problem.

-\noindent Problems~20-22 are also in this module's \textit{Model Exam}.

\begin{two-digit-list}
\item [1.] In the example of Sect.\,5b in the text, we found that 175\unit{J} of
energy was converted to heat and breaking of chemical bonds, primarily in
your toe.
If it requires an average of \m{10^{-18}\unit{J}} to break one molecular bond and
4/7 of the \Quote{lost} energy went into breaking molecular bonds, how many
bonds were broken?* \answer{3}

\item [2.] The useful output power from a certain water pump is 80.0\% of
the electrical power supplied to it.
This pump is used to raise 4.0\unit{kg/s} of water from a well.
If the water leaves the system at a speed of 8.0\unit{m/s} at a height of 12.0\unit{m}
above the water level in the well, find the electrical power being supplied
to the pump.* \answer{7} \help{24}

\item [3.] The high-dive at an Olympic swimming pool is 10.0\unit{m} (32.8\unit{ft})
above water level.
Neglecting air resistance, with what speed will a diver hit the water if he
or she starts with a speed of 5.0\unit{m/s} at an angle of {36.87\degrees} above
the horizontal?* \answer{1} \help{25}

\item [4.] A girl of mass 49.0\unit{kg} is on a swing which has a mass of
1.0\unit{kg}.
Suppose you pull her back until her center of mass is 2.0\unit{m} above the
ground.
You let her go and she swings out and returns to the same point.
Are all forces acting on the system conservative?
(Caution: this question deserves a carefully qualified answer.) \answer{2}

\item [5.] Assume that all forces in the previous problem are conservative,
and that the center of mass of the girl-plus-swing system is 0.75\unit{m} off the
ground at its lowest point.
Find the girl's maximum speed.* \answer{8}

\item [6.] Suppose you push the girl in problem~5 and add 490\unit{J} of energy
to the system.
Find her new maximum height above the ground.* \answer{4}

\item [7.] A certain toboggan with its riders has a mass of 305\unit{kg} and goes
down a hill which is 25\unit{m} high.
The toboggan starts from rest and attains a final speed at the bottom of the
hill of 12\unit{m/s}.
Find the energy lost to friction.* \answer{10}

\item [8.] One bit of advice that bowlers often hear is \Quote{let the weight of
the ball do the work.
Don't force it.}
For each of the following two cases, find the percentage of the ball's final
energy which is supplied by the bowler:*\help{26}
\begin{one-digit-list}
\item [a.] Good bowler, \m{\text{average } = 200}, \m{\text{backswing height } = 1.10\unit{m}}, release
                        \m{\text{speed } = 6.76\unit{m/s}}.  \answer{5}
\item [b.] Average bowler, \m{\text{average} = 120}, \m{\text{backswing height } = 0.70\unit{m}},
                           \m{\text{release speed } = 7.70\unit{m/s}}.  \answer{5}
           (Data from Murose, \Quote{Biomechanics of Bowling,} Nelson, ed.,
           {\em Biomechanics IV}, 1974.
\end{one-digit-list}

\item [9.] A rubber ball is dropped from a height of 1.5\unit{m} onto a concrete
           floor and rebounds to a height of 1.2\unit{m}.
           What fraction of its original energy was lost to friction?* \answer{11}

\item [10.] Given a conservative force defined by \m{\vect{F} = - k x^2 \uvec{x}}
find the corresponding potential energy function.
State the reference point. \answer{6}

\item [11.] Given a conservative force, \m{\vect{F} = k/r^3 \uvec{r}}, find the
associated potential energy function.
State the reference point. \answer{9}

\item [12.] Given the conservative force \m{\vect{F} = x y_0 \uvec{x}} find the
associated potential energy function.
State the reference point. \answer{13}

\item [13.] Given the conservative force, \m{\vect{F} = A \cos x\,\uvec{x}},
find the potential energy function.
State the reference point. \answer{14}

\item [14.] A man at the scene of an accident claims that he was traveling
on a level road at 25\unit{m/s} (56\unit{mph}) when he saw the car pull out in front
of him.
He says he skidded for about 30\unit{m} and hit the other car with a
speed of about 5\unit{m/s}.
If he is telling the truth, what was the coefficient of friction, assuming
it to be independent of speed? \answer{3} \help{12}

\item [15.] A car with a mass of 1500\unit{kg} accelerates from rest to a speed
of 21\unit{m/s} in a time of 8.0\unit{s}.
Neglecting losses due to friction and assuming a level road, what horsepower
is required?
(\unit{hp} = 746\unit{W}) \answer{15}

\item [16.] Re-do the previous problem for the car on a hilly road if its
increase in height is 8\unit{m}.* \answer{17}

\item [17.] A 40.0\unit{kg} mass is traveling horizontally with velocity
2.0\,\m{\uvec{x}}\unit{m/s} as it crosses the coordinate origin and encounters the
force \m{\vect{F} = - k x \uvec{x}} due to a spring.
If \m{k = 250}\,N/m, at what value of \m{x} will the mass stop?* \answer{3} \help{18}

\item [18.] A system consists of particles \m{A} and \m{B} with masses 11\unit{kg}
and 21\unit{kg} respectively.
At time zero: \m{A} has a velocity of (5.0\unit{m/s}, east) at height \m{h = 0.0}.
\m{B} has a velocity of (4.0\unit{m/s} at {23\degrees} north of east) and
\m{h = 2.0\unit{m}}.
One hour later: \m{A} is at rest at \m{h = 3.0\unit{m}}.
\m{B} has a velocity of (8.0\unit{m/s}, north) at \m{h = 0.0}.
Find the net work done on the system during the hour.* \answer{16} \help{29}

\item [19.] How many times could a 72\unit{kg} person climb the Empire State
Building while using up \m{2.0 \times 10^3} food calories of energy, if her or
his body were 100\% efficient in turning food energy into useful mechanical
energy?
The Empire State Building is 381\,m high, and a \m{\text{food calorie } = 4186\unit{J}}. \answer{19}
(It should be clear from the answer to this problem that very little of the
body's energy is converted into useful mechanical energy.
Most of it goes into heat.)

\item [20.] A nonconstant force, \m{F}, directed vertically upward (the force
varies in magnitude as some unknown function of altitude) is exerted on an
object of mass \m{m}.
Under the action of this force, the object moves straight up from ground
level to a distance \m{h}, above the ground.
It starts at rest and when it gets to height \m{h} it is moving with speed \m{v}
upward.
\begin{one-digit-list}
\item [a.] Draw the free-body diagram of forces that act on this mass during
the course of its motion upward.
Label the unknown force vector with a symbol but point it in the proper
direction. \answer{35}
\item [b.] Write down the Work-Force integral for this system, expressing
the work of each of the forces separately. \answer{22}
\item [c.] Evaluate the integrals that can be evaluated. \answer{28}
\item [d.] Draw the appropriate picture(s) for solving this problem using
conservation of energy. \answer{20}
\item [e.] Write the conservation of energy relation for this object. \answer{32}
\item [f.] How much work was done by the force \m{F}? \answer{34}
\end{one-digit-list}

\item [21.] A nail partially driven into a board, is driven in further by
dropping a mass onto it.
The mass, \m{m}, is released a distance \m{y_0} above the head of the nail and
drives the nail an additional distance \m{s} into the board.
Neglect the mass of the nail.
\begin{one-digit-list}
\item [a.] Defining the zero of potential at the top of the nail, find
the potential energy of the mass \m{m} at the instant \m{m} is released.
Assume the nail's mass is negligible. \answer{23}
\item [b.] What is the kinetic energy at that instant? \answer{33}
\item [c.] Find the potential energy after the mass has driven the nail the
additional distance \m{s} into the board. \answer{29}
\item [d.] What is the kinetic energy at that instant? \answer{21}
\item [e.] Write the conservation of energy relation for this problem. \answer{25}
\item [f.] How much work is done on the mass \m{m} by the
nail-plus-board-plus-earth mechanically resistive system? \answer{30}
\item [g.] How much work is done on the nail-plus-board-plus-earth
mechanically resistive system by the mass-plus-earth gravitational system? \answer{31}
\item [h.] Assuming that the nail exerts a constant force on the mass during
the interval that the nail is slowing the mass to a stop, determine the
magnitude of this force, \m{F_N}.* \answer{24}
\end{one-digit-list}

\item [22.] \noindent \CenteredUnframedFixedFigure{m21gr02}\newline
An artillery shell fired from a gun at ground level leaves the muzzle of the
gun with a speed of \m{5.0 \times 10^1\unit{m/s}}.
The shell strikes a direct hit at a site on the top of a hill which is at an
altitude of 45\unit{m} above ground level.
\begin{one-digit-list}
\item [a.] Assuming no friction, use energy conservation to find the speed
with which the shell hits at the site. \answer{27}
\item [b.] If the shell strikes the ground at the site with a speed of
38\unit{m/s}, what percentage of its original energy was lost (to friction with
the air) by the shell in its flight? \answer{26}
\end{one-digit-list}
\end{two-digit-list}

\newpage

\BriefAns

\begin{two-digit-list}
\item [1.] 14.9\unit{m/s}
\item [2.] Possibly yes, if there is no friction.
But in the more likely real case, there will be friction, which is a
non-conservative force.
Thus if she does indeed return to the same point, there must have been a
non-conservative force acting on the system, e.g., the girl \Quote{pumping.}
\item [3.] \m{10^{20}}.  Ouch!
\item [4.] 3.0\unit{m}
\item [5.] a) 53\%;  b) 77\%
\item [6.] \m{k x^3/3}; \m{x = 0}
\item [7.] \m{7.5 \times 10^2\unit{W}}
\item [8.] 5.0\unit{m/s}
\item [9.] \m{k/(2 r^2)}; \m{x = \infty}
\item [10.] 53\unit{kJ}
\item [11.] 1/5
\item [12.] 1.0
\item [13.] \m{- x^2 y_0/2}; \m{x = 0}
\item [14.] \m{- A \sin x}; \m{x = 0}
\item [15.] 55\unit{hp}
\item [16.] \m{2.8 \times 10^2\unit{J}}
\item [17.] 75\unit{hp}
\item [18.] 0.80\unit{m}
\item [19.] 31 times
\item [20.] \noindent \CenteredUnframedFixedFigure{m21gr03}\newline
\item [21.] Zero
\item [22.] \m{\int_\text{ground}^h \vect{F} \cdot d\vect{r} +
           \int_\text{ground}^h (- m g) ds = \dfrac{1}{2} m v^2}
\item [23.] \m{ m g y_0}
\item [24.] \m{m g \left(1+\dfrac{y_0}{s}\right)}
\item [25.] \m{m g y_0 = - m g s + } energy dissipated in heat and ruptured
bonds
\item [26.] \m{\approx} 7\% (6\% or 7\%)
\item [27.] \m{4.0\times10^1\unit{m/s}}
\item [28.] Second integral is \m{- m g h}
\item [29.] \m{- m g s}
\item [30.] \m{- m g (y_0 + s)}
\item [31.] \m{m g (y_0 + s)}
\item [32.] \m{\int_\text{ground}^h \vect{F} \cdot d\vect{r} =
\dfrac{1}{2} m v^2 + m g h}.

Note that \m{E_k\text{(ground)} = 0} and \m{E_p\text{(ground)} = 0}.
\item [33.] Zero
\item [34.] \m{\dfrac{1}{2} m v^2 + m g h}
\item [35.] See diagram: \newline\CenteredUnframedFixedFigure{m21gr04}
\end{two-digit-list}

}% /Sect
