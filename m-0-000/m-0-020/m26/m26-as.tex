\revhist{9/9/91, pss; 11/18/93, pss}

\Sect{}{}{\SectType{SpecialAssistance}}{

\AsItem{1}{TX-1b}
{\noindent\CenteredUnframedFixedFigure{m26gr05}\newline

 The force curve is related to the potential energy curve by \m{F = - dE_p/dx}:

 \CenteredUnframedFixedFigure{m26gr06}\newline
 Now draw the force curve for: \m{E_p = \dfrac{1}{2} k (x - x_0)^2.} \help{6}
}

\AsItem{2}{TX-3a}
{\m{\cos(\theta + \phi) = \cos\phi \cos\theta - \sin\phi \sin\theta}\smallskip \newline
 \m{X = A \cos(\omega t + \delta_0) =
 A [\cos\delta_0 \cos\omega t - \sin\delta_0 \sin\omega t]}\smallskip \newline
 Let: \m{a \equiv A \cos\delta_0}; \m{b \equiv - A \sin\delta_0}.\smallskip \newline
 Then: \m{X = a \cos\omega t + b \sin\omega t}.
}

\AsItem{3}{TX-3b}
{\m{\omega = \left( \dfrac{k}{m} \right)^{1/2} =
 \left( \dfrac{48\unit{N/m}}{3\unit{kg}} \right)^{1/2} = 4\unit{rad/s}}
 \smallskip\newline
 \m{T = \dfrac{2 \pi}{\omega} =
 \dfrac{2 \pi}{4\unit{s}} = \dfrac{\pi}{2}}\,s; 
 \m{\nu = \dfrac{1}{T} = \dfrac{1}{\pi/2\unit{s}} = \dfrac{2}{\pi}\unit{Hz}}
}

\AsItem{4}{TX-3c}
{ \m{\omega = \left(\dfrac{k}{m} \right)^{1/2} =
 \left( \dfrac{9.8\unit{N/m}}{0.1\unit{kg}} \right)^{1/2} = 9.9\unit{rad/s}}
}

\AsItem{5}{TX-3c}
{\m{x = A \cos(\omega t + \delta_0)}\smallskip \newline
 \m{v = dx/dt = -A \omega \sin(\omega t + \delta_0)}\smallskip \newline
 at \m{t = 0}; \m{v = - A \omega \sin\delta_0}\smallskip \newline
 if \m{v = 0} then \m{\delta_0 = 0} (or \m{n \pi}); if \m{v \neq 0} then \m{\delta_0 \neq 0} \smallskip \newline
 for \m{\delta_0 \neq 0}; \m{X_i = A \cos\delta_0} \m{(t = 0)}\smallskip \newline
 \m{| \cos\delta_0 | < 1}: \m{X_i < A}\smallskip \newline
 i.e. amplitude is larger than initial displacement \m{X_i}.
}

\AsItem{6}{[S-1]}
 {\m{F = - dE_p/dx = - k (x - x_0)}

 \CenteredUnframedFixedFigure{m26gr07}
}

}% /Sect
