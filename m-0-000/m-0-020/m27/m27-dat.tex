\revhist{2/28/85 mpm; 1/31/91, pss; 10/11/94, pss; 2/22/99, pss; 5/26/99, pss; 10/19/99, pss;
         4/25/02, pss}
%
\defModTitle{\ph{Phasors}}
\defCtAuthor{Peter Signell, Michigan State University}
\defIdAuthor{Peter Signell, Michigan State University}
%
\defLG{True}
\defRD{3}
%
\defIdItems{
    \IdVersEval{4/25/2002}{4}
    \IdHours{1}
    \begin{InputSkills}
    \item [1.] Vocabulary: acceleration, amplitude, angular frequency, displacement,
    restoring force, frequency, initial phase, phase \prrqtwo{0-430}{0-25}.
    \end{InputSkills}
    %
    \begin{KnowledgeSkills}
    \item [K1.] Vocabulary: phasor, phasor diagram.
    \item [K2.] Describe the relative positions and the motions of phasors on a phasor
    diagram representing the displacement, velocity and acceleration of a
    harmonic oscillator, and the restoring force acting on it.
    \item [K3.] State how the actual displacement, velocity and acceleration of a
    harmonic oscillator and the actual restoring force can be determined from a
    phasor diagram.
    \end{KnowledgeSkills}
    %
    \begin{ProblemSolvingSkills}
    \item [S1.] Given a harmonic oscillator with specified amplitude, frequency, and
    initial phase, draw a phasor diagram illustrating the displacement, velocity,
    and acceleration of the oscillator and the restoring force acting on it at a
    specific time.
    Describe how the diagram changes if any one of the given parameters is
    changed continuously from its initial value.
    \end{ProblemSolvingSkills}
    %
    \begin{RequiredResources}
    \item [1.] M.\,Alonso and E.\,J.\,Finn, \textit{Physics}, Addison-Wesley (1970).
    For availability, see this module's \textit{Local Guide}.
    \end{RequiredResources}
}