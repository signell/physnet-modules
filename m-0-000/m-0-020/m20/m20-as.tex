\revhist{8/14/91, pss; 11/5/91, pss; 11/24/92, pss; 3/9/93, pss; 11/19/93, pss
         4/13/94, pss; 9/7/94, pss; 12/8/95, pss; 8/23/97, pss; 11/13/97, pss
         12/4/97, pss; 4/1/02, pss; 4/23/02, pss}

\Sect{}{}{\SectType{SpecialAssistance}}{

\xpcap{}{}{PURPOSE}
If you have trouble with the Problem Supplement, carry out the activities in
this Supplement.\smallskip

\xpcap{}{}{CONTENTS}\vspace*{-7pt}
\begin{itemize}
\item [1.] Introduction
\item [2.] Understanding the Definitions
\item [3.] Advice and Problems With Hints Available
\item [4.] Answers for the Assistance Supplement
\item [5.] Hints for the \Emph{Problem and Assistance Supplements}
\end{itemize}

\xpcap{}{}{Sect.\,1. Introduction}
We apply these general relationships:
%
\Eqn{}{W_{A\rightarrow B} = \int_A^B \vect{F} \cdot \,d\vect{s} = \Delta E_k}
%
and
%
\Eqn{}{P = \dfrac{dW}{dt} = \vect{F} \cdot \,\vect{v}\,.}
%
You should fully understand those relationships so you can recognize their
applications to special cases.


\xpcap{}{}{Sect.\,2. Understanding the Definitions}

\begin{one-digit-list}
\item [a.] Complete the following table, which summarizes the basic
characteristics of the quantities in the first column:

\begin{center}\begin{tabular}{|l|c|c|c|c|c|}                         \hline
               & Vector & Possible          & SI   & SI     & Unit in\\
               &  or    &   Sign            & Unit & Symbol & MKS    \\
               & Scalar & \m{+},\,0,\,or\,\m{-} &      &        &        \\ \hline
Force          & & & & & \\ \hline
Displacement   & & & & & \\ \hline
Work           & & & & & \\ \hline
Power          & & & & & \\ \hline
\end{tabular}\end{center} \answer{49}

\item [b.] Symbol Recognition: write out in words the meaning of each of the
following symbols:
\begin{center}\begin{tabular}{l l l}
(a) \m{W_{A \rightarrow B}} & (d) \m{d\vect{s}} & (g) \m{{\Delta}E_k} \\
(b) \m{W_T}                 & (e) \m{E_k}      & (h) \m{P_{av}} \\
(c) \m{\vect{F}_R}           & (f) dW         & (i) \m{\vect{F}(s)} \\
\end{tabular}\end{center} \answer{50}

\item [c.] Describe under what conditions each of the following is a valid
expression for \m{\int_A^B \vect{F}(s) \cdot\, d\vect{s}}:
\begin{center}\begin{tabular}{l l}
(a) \m{ = \vect{F} \cdot \vect{s}} \qquad & (d) \m{ = F \cos\theta \int \,ds} \\
(b) \m{ = F s}                   \qquad & (e) \m{ = \int F \cos\theta\,ds} \\
(c) \m{ = F \int \,ds} \\
\end{tabular}\end{center} \answer{51}

\item [d.] Indicate whether each of the following expressions is either
correctly or incorrectly written:
\begin{center}\begin{tabular}{l l}
(a) \m{\int \,dW = \int F \cos\theta\,ds} \qquad     &
(f) \m{W = \int \vect{F}\,d\vect{s}} \\
(b) \m{W = \int F s}                              &
(g) \m{W = F s} \\
(c) \m{\int \,dW = \int \vect{F} \cdot \,d\vect{s}} &
(h) \m{\int \vect{F} \cdot \,d\vect{s} = \vect{F} \cdot \vect{s}} \\
(d) \m{dW = \vect{F} \cdot \vect{s}}                &
(i) \m{W = \int P dt} \\
(e) \m{W = F d s} & (j) \m{P = \vect{F} \vect{v}} \\
\end{tabular}\end{center} \answer{52}

\item [e.] How is it possible for an object which is moving to have:
           \m{\int \,d\vect{s} = 0}? \answer{53}
\item [f.] State the three cases for which: \m{W =
           \int \vect{F} \cdot \,d\vect{s} = 0}. \answer{54}
\item [g.] Show that: \m{\left( F_x \uvec{x} + F_y \uvec{y} +
           F_z \uvec{z} \right) \cdot \left( dx \uvec{x} +
           dy \uvec{y} + dz \uvec{z} \right)} reduces to \m{F_x\,dx + F_y\,dy +
           F_z\,dz}. \answer{55}
\item [h.] A 90\unit{kg} ice hockey player skating during warmup has a kinetic
           energy of 400\unit{J}.
           (a) If during the game he skates at three times his warmup speed
           what is his new kinetic energy?
           (b) If during the game he collides with another player, of the
           same mass, velocity, and slide along the ice together, what is
           their combined kinetic energy? \answer{56}
\item [i.] Compute the average kinetic energy of a 70\unit{kg} sprinter who
           covers 100\unit{m} in 9.8\,\unit{seconds}. \answer{57}
\item [j.] When we are moving at a constant velocity what can we say about
           \m{\Delta E_k}? \answer{58}
\item [k.] If an object with forces acting on it is moving with a velocity
           \m{v} and \m{\Delta E_k = 0}, then why does
           \m{W = \int \vect{F} \cdot \,d\vect{s} = 0}? \answer{59}
\item [l.] What do we mean by: (a) \Quote{\m{+}} work? (b) \Quote{\m{-}} work?
           (c) \Quote{\m{+}} power? (d) \Quote{\m{-}} power? \answer{60}
\item [m.] Why is kinetic energy either \m{0} or \m{+} (that is, it is never
           negative)? \answer{61}
\item [n.] Calculate both analytically and graphically the work done by the
           force shown below in moving a particle from \m{x = 0} to \m{x = 6\unit{meters}}.
           \newline\CenteredUnframedFixedFigure{m20gr14} \answer{62}
\item [o.] If a force \m{\vect{F} = (2x + 3 x^3) \uvec{x}\unit{N}} acts on an object,
           what is the work done by this force for a displacement from \m{x = 2}
           to \m{x = 5\unit{meters}}? \answer{63}
\item [p.] Starting with \m{W = \int_A^B \vect{F} \cdot\,d\vect{s}}, derive the
           expression:
           %
           \Eqn{}{W = \int_{t_1}^{t_2} P\,dt.}
           %
\item [q.] Show that: \m{\int_{v_i}^{v_f} \vect{v} \cdot\, d\vect{v} = \dfrac{1}{2}(v_f^2 - v_i^2)}.
\item []   Hint -- write the vectors in component form.
\item [r.] (a) What power is developed by a 70\unit{kg} man when climbing, in 20\,\unit{seconds},
               a flight of stairs that rises 12\,\unit{meters}?
\item []   (b) What power must be provided to move a 10\unit{kg} block
           horizontally on a surface with friction, if a force of
           \m{1.0\times10^2\unit{N}} gives it a constant velocity of
           \m{2.0\times10^1\unit{m/s}}? \answer{64}
\item [s.] Why is the definition of work a valuable concept? \answer{65}
\item [t.] Why is the work-energy principle a useful relationship? \answer{66}
\end{one-digit-list}

\xpcap{}{}{Sect.\,3. Advice and Problems With Hints Available}

\noindent In working out the solution to a problem:

\begin{one-digit-list}
\item [a.] List the properties or data given.
\item [b.] Clearly indicate which quantities are to be found.
\item [c.] Determine which conceptual relationship(s) is to be applied.
\item [d.] Simplify the general relationship for the specific case.
\end{one-digit-list}

\noindent Attempt to work these problems without assistance before consulting
the answers or hints provided in Sections 5 and 6.

\begin{one-digit-list}
\item [1.] \ItemFigure{An object which has a mass of 20\unit{kg} sits on a horizontal surface.
           It is found that the frictional force between the object and the
           surface when the object moves across the surface is a constant 15\unit{N}.
           Suppose that you move this object with a horizontal force a distance
           of 50\unit{m} at a constant speed of 3\unit{m/s}.}{m20gr15}

\item []   How much work is done by the resultant of all the forces acting on
           this object?  How much work do you do?  \answer{39}

\item [2.] A quarterback throws a long pass and the football
          (\m{\text{ mass } = 4.0\times10^2\unit{gm}}) leaves his hand at an angle of
           \m{4.0\times10^1\,\unit{degrees}} to the horizontal ground.
           The ball hits the tight end, who lets it slip through his hands
           so that it falls to the ground.
\begin{one-digit-list}
\item [a.] What was the work done on the ball during the first 0.10\unit{m} of
           travel just after it left the quarterback's hand?
           This displacement may be considered infinitesimal compared to the
           total trajectory.  \answer{43}
\item [b.] What is the work done on the ball during an 0.10\unit{m} displacement
           at the peak of its trajectory?  \answer{38}
\item [c.] What is the increment of work done on the ball during an 0.10\unit{m}
           displacement as it falls vertically to the ground just after
           hitting the tight end?  \answer{41}
\end{one-digit-list}

\item [3.] A \m{1.00\times10^2\unit{kg}} fullback has a velocity of 9.0\unit{m/s} as he
           hits the line.
           If he is to be held to a 2.0\unit{m} gain, what average resultant
           force must be exerted on him by the opposing team?  \answer{44}

\item [4.] A baseball (\m{\text{mass } = 0.15\unit{kg}}) has a kinetic energy of
           \m{5.0\times10^1\unit{J}} at the peak of its trajectory, which is 34\unit{m} above ground.
           What was the velocity of the ball just after it was hit by the bat?  \answer{37}

\item [5.] A \m{1.00\times10^3\unit{kg}} elevator starts from rest and experiences
           a constant upward acceleration of 2.0\unit{m/s\up{2}}.
           Determine the power required to maintain this acceleration for
           4.0\unit{s}.  \answer{38}
\end{one-digit-list}

\newpage

\xpcap{}{}{Answers for the Assistance Supplement}
\begin{two-digit-list}
\item [37.] 36.5\unit{m/s}. The work done by gravity is \m{mgh = \Delta E_k}.
\item [38.] Zero, since \m{m\,g\,\uvec{y} \perp d\vect{s}}.
\item [39.] When you draw your one-body force diagram you should have four
           forces acting on this object: gravity, the normal force upward
           exerted by the surface, the force you apply, and the frictional
           force in the opposite direction.
           The key phrase in this problem description is \Quote{constant speed,}
           which implies \m{\vect{a} = 0} so \m{\vect{F}_R = 0}, hence \m{W_R = 0}.
\item []   What force do you exert? (J)
           How much work do you do? (K)
\item [40.] 8000\,W.
           Use \m{s = v t + a t^2/2} to find the distance travelled and
           evaluate \m{W = \int \vect{F} \cdot \,d\vect{s} = m a s}, then solve
           for power.
\item [41.] \m{+0.39\unit{J}}.
            \m{W = \int\,m\,g(+1)\,ds}.
\item [42.] It does not change.
           This is consistent with \m{W = \Delta E_k = 0}, since the speed is
           constant.
\item [43.] \m{-0.25\unit{J}}, \m{W = \int\,m\,g(\cos\,130\degrees)\,ds}. \help{39}
\item [44.] 2025\unit{N}.
           Use the work-kinetic energy relation, assuming the opposing force
           starts acting at the line of scrimmage.
\item [45.] 6 meters.
           What is the kinetic energy of this object?
           (See L).
           Now apply the work-kinetic energy principle, where the only force
           acting on the block is the resistive frictional force.
\item [46.] Constant \m{v} so \m{\vect{F} = 0} so you exert 15\unit{N}.
\item [47.] given \m{v} and \m{d}, \m{t = d/v} so \m{W = P t = (F v) \times (d/v) = 750\unit{J}}.
            Also: \m{W = F \times d = 750\unit{J}}.
\item [48.] 90\unit{J}.
\item [49.] \noindent\newline
\begin{tabular}{|l|c|c|c|c|c|}                            \hline
               & Vector & Possible          & SI   & SI     & Unit in\\
               &  or    &   Sign            & Unit & Symbol & MKS    \\
               & Scalar &                   &      &        &  \\ \hline
Force          & Vector & \m{+,0,-} & newton& N & kg\,m\,/s\up{2} \\ \hline
Displacement   & Vector & \m{+,0,-} & meter & m & m \\ \hline
Work           & Scalar & \m{+,0,-} & joule & J & kg\,m\up{2}/s\up{2} \\ \hline
Power          & Scalar & \m{+,0,-} & watt  & W & kg\,m\up{2}/s\up{3} \\ \hline
\end{tabular}

\item [50.] (a) Work done on a particle undergoing a displacement from position
           \m{A} to position \m{B}.
\item []   (b) The scalar summation of the work done by each force acting on a
           particle - superposition of work principle.
\item []   (c) The resultant force which acts upon a particle.
           It is the same as \m{\sum_i \vect{F}_i}.
\item []   (d) An infinitesimal displacement of a particle.
           It is a vector.
           Contrast it to \vect{s}, which is a finite displacement.
\item []   (e) The kinetic energy of a particle.
           For a particle of mass m moving with a velocity \m{v}, the kinetic
           energy is \m{mv^2/2}.
\item []   (f) An element of work done on a particle by a force during an
           infinitesimal displacement \m{d\vect{s}}.
\item []   (g) The change in kinetic energy of a particle;
           \m{mv_f^2/2 - mv_i^2/2}, where \m{v_f} is the velocity at the final
           position (\m{f}) and \m{v_i} is the velocity at the initial position
           (\m{i}).
\item []   (h) The average power, (\m{W/t}) exerted during a time interval \m{t},
           where the total work done during this interval is \m{W}.
\item []   (i) A force that varies in magnitude and direction as a function
           of position.
\item [51.] (a) Constant force, not necessarily parallel to finite
           displacement.
\item []   (b) Constant force, parallel to finite displacement.
           Although W = Fs may be the easiest form of the work definition to
           recall, remember that it is just a special case of a more general
           relationship.
\item []   (c) Constant force, parallel to an infinitesimal displacement.
\item []   (d) Constant force, not parallel to an infinitesimal displacement.
\item []   (e) Variable force which depends on position.
\item [52.] (a) ok
\item []   (b) NO - must have a ds under integral sign.
\item []   (c) ok
\item []   (d) No - left side implies increment, right side a finite amount.
\item []   (e) No - must have integral sign on right side.
\item []   (f) No, must have vector multiplication sign (dot) between
           \vect{F} and d\vect{s}.
\item []   (g) ok
\item []   (h) ok
\item []   (i) ok
\item []   (j) No, just have vector multiplication sign (dot) between
           \vect{F} and \vect{v}.
\item [53.] Consider the work done by gravity on a round trip, when you throw
           a ball up into the air vertically and it returns to your hand.
\item []   \m{W = \int \vect{F} \cdot \,d\vect{s}}
\item []   \m{W = W_\text{up} + W_\text{down} =
           \int mg(-1)\,ds + \int mg(+1)\,ds = 0}
           because net \m{\vect{s} = \int \,d\vect{s} = 0}.
\item []   For any round trip displacement net \m{\vect{s} = 0}.
\item [54.] \m{\vect{F} = 0}, \m{d\vect{s} = 0} and \m{\vect{F} \perp \,d\vect{s}}.
\item []   \ItemFigure{If a constant force \vect{F} acts perpendicular to
                          the displacement \vect{s} of a particle then
                          \m{\theta = 90\degrees} so that \m{W = 0} and no work is
                          done by this force.
                          Note that work may not always be done on a particle
                          which is undergoing a displacement even though a
                          force is applied.}{m20gr16}
\item []   Here are three forces doing zero work (see figures below):
\item []   (a) the force of gravity (\m{m g \uvec{y}}) on a particle moving
           horizontally.
\item []   (b) the normal force (\vect{N}) as the particle moves along the
           surface.
\item []   (c) the tension (\vect{T}) in the cord of a simple pendulum,
           since this force is always \m{\perp} to the displacement.

           \CenteredUnframedFixedFigure{m20gr17}
\item [55.] Remember that \m{\uvec{x} \cdot \uvec{x} = 1},
           \m{\uvec{x} \cdot \uvec{y} = \uvec{x} \cdot \uvec{z} = 0}, etc.

\item [56.] a) 3600\unit{J}
\item [ ]   b) 7200\unit{J}

\item [57.] \m{3.6 \times 10^3\unit{J}}.  Hint: \m{\vect{v} = d\vect{s}/dt}.

\item [58.] \m{\Delta E_k = 0}, since \m{\Delta E_k = m \Delta (v^2)/2 =
           m [(v + \Delta v)^2 - v^2]/2}, and \m{\Delta v = 0}.

\item [59.] \m{\vect{F}_R = 0}, that is there is no net force parallel to the
           displacement.

\item [60.] (a) The resultant force acts on the particle in the direction of
           the displacement, so particle increases in energy (kinetic).
\item []   (b) The resultant force acting on the particle is opposite to the
           direction of the displacement, so that there is a decrease in
           kinetic energy.
\item []   (c) An increase in the energy of the system.
\item []   (d) A decrease in the energy of the system.
\item [61.] \m{4E_k = m v^2/2}, since \m{v^2} is always positive the kinetic
            energy is positive.
            You can never have less than zero kinetic energy.
            Zero kinetic energy corresponds to an object at rest.
            Since kinetic energy is a scalar it does not tell you anything
            about the direction of the velocity of a particle.
\item [62.] 19\unit{J}
\item [63.] 480\unit{J}
\item [64.] a) 412\unit{W}; b) 2000\unit{W} \help{38}
\item [65.] It provides a method for calculating the change in energy of a
           particle when the force acting on it is not constant but is a
           function of the position of the particle.
\item [66.] Work can be found from the \m{\Delta E_k} without having to know
           anything about the nature of the forces involved.
\end{two-digit-list}

\xpcap{}{}{Sect.\,6: Hints for the \textit{Problem Supplement}}

\AsItem{1}{PS-problem~1}
{(a) At constant magnitude of the frictional force will equal the
           magnitude applied force.
           Resolve the applied force in its vertical and horizontal components.
           Remember that \m{N = m g - F \sin\theta}.

 (b) The coefficient of friction is the frictional force divided by
           the normal force, so that
           \m{\mu = F \cos\theta/(m g - F \sin\theta)}.
}

\AsItem{2}{PS-problem~2}
{Determine \m{\vect{F} \cdot \,d\vect{s}}, applying the fact that
           \m{\uvec{x} \cdot \uvec{x} = 1} and \m{\uvec{x} \cdot \uvec{y} =
           \uvec{x} \cdot \uvec{z} = 0}.
}

\AsItem{3}{PS-problem~3}
{The kinetic energy is given, so use \m{mv^2/2 = E_k} and solve to find that
 \m{v = (2E_k/m)^{1/^2}}.
}

\AsItem{4}{PS-problem~4}
{Recall that \m{|\vect{v}|^2 = v^2 = v_x^2 + v_y^2}.
}

\AsItem{5}{PS-problem~5}{%
           (a) Calculate the work done by the escalator against gravity by
           resolving the force it exerts into vertical and horizontal
           components.
           Only the vertical component will do any work.
           Why?

           (b) Use the definition of power involving work and time.}

\AsItem{6}{PS-problem~6}
{Use the definition of power involving force and velocity.
}

\AsItem{7}{PS-problem~7}{%
           Draw a one-body diagram for the car.
           Resolve the forces into their components parallel and
           perpendicular to the street.
           Here we write the street-to-horizontal angle as \m{\phi}, the
           cable-to-street angle as \m{\theta}, the mass of the car as \m{m},
           the tension in the cable as \m{T}, the distance along the street as
           \m{s}, the coefficient of friction as \m{\mu}, and the work done by
           the \textit{truck} on the \textit{car} as \m{W}.
           Then for the work done by the truck (via the cable) on the
           car-earth system we get:\newline
           \m{W = s T \cos\theta = m g s \cos\theta (\sin\phi + \mu \cos\phi)/(\cos\theta +
           \mu \sin\theta)}.\newline
           If you still have trouble, see \help{24}.}

\AsItem{8}{PS-problem~8}{%
           Since the particle is in equilibrium the sum of the horizontal
           components of force are zero, as is the sum of the vertical components.
           You should be able to show for the horizontal components that:
           \m{F - T \sin\phi = 0}, and for the vertical components, \m{T \cos\phi - m g = 0}.
           Eliminate \m{T} between these two equations to obtain \m{F} in terms of \m{\phi}.
           Since \m{\phi = s/\ell}, then \m{ds = \ell\,d\phi}.
           Now evaluate \m{\int \vect{F} \cdot \,d\vect{s}}, remembering that
           there is an angle between \vect{F} and \m{d\vect{s}}.}

\AsItem{9}{PS-problem~9}{Apply the definitions of \m{\vect{v} = d\vect{s}/dt} and \m{E_k}.}

\AsItem{10}{PS-problem~10}{Show that the work done is equal to
           \m{\int (m g \uvec{z}) \cdot (d\vect{s})} where \vect{s} turns out to
           be the net vertical displacement.}

\AsItem{11}{PS-problem~11}{For your analytical solution write the equation for the
           straight line shown which will represent the effective force.
           This is \m{F = 10 - 2x}.}

\AsItem{13}{PS-problem~13}{Apply the work-kinetic energy relation, assuming that
 the retarding force is constant during the displacement, to find the distance the
 bumper moves from knowing the total work done using \m{\Delta E_k}.}

\AsItem{14}{PS-problem~14}{Apply the work-kinetic energy relation for motion in a
 vertical direction:  \m{v_f = [2gh + v_0^2]^{1/2}}.
 This method and the laws of linear motion should lead you to the same
 results.}

\AsItem{15}{PS-problem~15}{First apply the work-kinetic energy relation to find the
 average retarding force per unit thickness of metal plate, since the \m{\Delta E_k} is known.
 Then apply the principle a second time with the new \m{\Delta E_k} to find the distance required.
 Note that \m{v_f = 0} for the second case.}

\AsItem{16}{PS-problem~16}{Use the work-kinetic energy relation, with
 \m{F_\text{friction } = \mu N = \mu m\,g} so that \m{s = v_0^2/(2\mu g)}.}

\AsItem{17}{PS-problem~17}{%
 (a) Since \m{F = kx}, find \m{k} from the given values of \m{F} and \m{x}.
     Then integrate \m{\vect{F} \cdot d\vect{s}} from 0 to 0.3\unit{m} to find the work done.
     Note that only 80\% of this work is converted into \m{E_k}.

 (b) \m{h = v_i^2/(2g)}

 (c) \m{F = [(mv_i^2/2) - (mgh')]/h'} where \m{h' = 0.9\,h}.}

\AsItem{18}{PS-problem~18}{Here \m{x = y} so that \m{dx = dy}.
           Use \m{\int F_x\,dx + \int F_y\,dy} and simplify before integrating.}

\AsItem{19}{PS-problem~19}{Use \m{\int \vect{F} \cdot \,d\vect{s}} with
           \m{\vect{F} = - k x^2 \uvec{x}} and \m{d\vect{s} = dx \uvec{x}}.}

\AsItem{20}{PS-problem~20}{Use the work-kinetic energy relation, including the drag
 energy in the \m{W} expression, so that the power will equal
 \m{[W_\text{drag} + (M v_f^2/2)]/t} where \m{M} is the total mass.}

\AsItem{21}{PS-problem~21}{In the expression \m{W = \int_i^f \vect{F}\cdot d\vect{s}}:

 (1) What is \m{d\vect{s}}? \help{26}

 (2) What is \m{\vect{F}\cdot d\vect{s}}? \help{32}

 (3) Use a \Emph{given} property of \vect{F} to simplify the integral
 before integrating. \help{28}

 Part (a): The other forces besides \vect{F} acting of the subject are the
 force due to the earth's gravity, and the force exerted on the object by the
 surface of the incline.
 Draw your vector diagram of the forces and resolve them into their components
 parallel and perpendicular to the inclined surface.
 
 (4) Repeat questions (1) and (2) above for the force due to gravity. \help{27} and \help{35}

 (5) Repeat questions (1) and (2) above for the force exerted by the surface. \help{29} and \help{1}

 (6) What are the resulting force components parallel to the displacement?  \help{36}

 Part (b): Now your initial velocity vector is \textit{down} the incline, but
 \m{d\vect{s}} is still up the incline. Why? \help{34}}

\AsItem{22}{PS-problem~22}{%
 (a) You should find \m{r = (K/g)^{1/2}}.

 (b) This is a variable force: it depends upon position \vect{s}. So:

     \m{W_{R_e \rightarrow r} = \int_{R_e}^{r} \vect{F}\cdot d\vect{s}}

     Why must \m{\vect{F}\cdot d\vect{s}} be evaluated under the integral sign? \help{33}

 (c) Here \m{r \rightarrow \infty} so that the work done becomes \m{K\,m/R_e}. Why? \help{37}}

\AsItem{23}{PS-problem~23}{\m{F = \dfrac{1.60\times 10^2 \times 550\unit{ft}\unit{lb/s}}
            {(56\unit{mi/hr})(\unit{hr}/3600\unit{s})(5280\unit{ft/mi})}}}

\AsItem{24}{[S-7]}{The work done by the cable is done against car-earth
 frictional (contact) and gravitational (non-contact) forces.
 \m{W_{cable} = \vect{T}\cdot\vect{s} = Ts\cos\theta} and
 \m{T=mg(\mu\cos\phi+\sin\phi)/(\mu\sin\theta+\cos\theta)}.
 \newline
 \textbf{Only for those interested}:  The work done against the
 car-earth gravitational force is \m{mgs\sin\phi=1.24\times10^6\unit{J}}.
 The work done against the car-earth frictional force is
 \m{\mu Ns=0.23\times10^6\unit{J}} where:\newline
 \m{N=mg(\cos\theta\cos\phi-\sin\theta\sin\phi)/%
 (\mu\sin\theta+\cos\theta)}.}

\AsItem{26}{[S-21]}{An infinitesimal displacement vector whose magnitude is the
 element of path length along the incline and whose direction is up the incline parallel to its surface.}

\AsItem{27}{[S-21]}{Same as [S-26].}

\AsItem{28}{[S-21]}{\m{\int \vect{F}\cdot d\vect{s} = \vect{F}\cdot \int d\vect{s}} because \m{F}
 is constant.  \m{\int d\vect{s}} is just a vector along the incline whose magnitude is the
 total distance between the starting point and the ending point.}

\AsItem{29}{[S-21]}{Same as [S-26].}

\AsItem{31}{[S-21]}{Zero (be sure you know why).}

\AsItem{32}{[S-21]}{\m{(F\cos\theta) ds} where \m{ds} is element of path length up the incline.}

\AsItem{33}{[S-22]}{The unit vector \m{\uvec{r}} is radially outward, by definition from mathematics.
 The force you exert is in the same direction so:
 %
 \Eqn{}{\vect{F} = \dfrac{K m}{r^2} \left(\uvec{r}\right)\,.}
 %
 The displacement was described as being in the direction away from the
 center of the earth so: \m{d\vect{s} = \uvec{r} dr}\,.
 Recall that the scalar product of any unit vector with itself is unity
 so:
 %
 \Eqn{}{\vect{F} \cdot d\vect{s} = \dfrac{K m}{r^2}\,dr\,.}
 %
}

\AsItem{34}{[S-21]}{Since it is only the final displacement \m{D} which is of interest.}

\AsItem{35}{[S-21]}{\m{-(Mg\sin\theta)ds}, where \m{ds} is element of path length up the incline.}

\AsItem{36}{[S-21]}{\m{(F\cos\theta - mg \sin\theta)}.}

\AsItem{37}{[S-22]}{As \m{r \rightarrow \infty}, \m{\dfrac{1}{r} \rightarrow 0}.}
        
\AsItem{38}{AS-problem (r)}{velocity of block was said to be constant, so acceleration
 of block is zero, so net force on block is zero.}
        
\AsItem{39}{AS-Answer (43)}{During this short time interval, the angle between the path
 of the football and the force of gravity is \ldots}
        
}% /Sect
