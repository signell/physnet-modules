\revhist{12/20/88, ejdk; 8/10/91, pss; 5/6/92, pss; 3/8/93, pss; 9/7/94, pss; 3/27/95, pss;
         11/14/96, pss; 11/7/97, pss; 11/7/97, lae; 12/13/2000, pss; 3/22/01, pss; 3/29/01, kag;
         4/23/02, pss; 10/3/02, pss}

\begin{center}\Quote{It's love that makes the world go round.}\\
                            \hspace*{2in}  - Ancient Ditty \end{center}

\begin{center}\Quote{Energy makes the world go round. \\
   ENERGY explains EVERYTHING.}\\
            \hspace*{2.4in} \parbox[t]{2in}{- Modern Ditty to be\newline
                                          found in science books}\end{center}
%
\Sect{1}{Introduction}{\SectType{TextOnePara}}{
%
The fundamental problem of particle dynamics is to determine the external
forces that act on an object, then use them to find the position of the
object as a function of time.
In more detail, once we know the forces we add them to get the resultant
force \m{\vect{F}_R}, which we put into Newton's second law in order to get the
acceleration \vect{a}.
We can then find the final velocity by integrating the time-varying vector
equation, inserting the initial velocity as an integration constant.
We can repeat the integration to find the time-varying position.
If the resultant force is constant in time these integrations produce
\m{\vect{v} = \vect{v}_0 + \vect{a}\,t} and \m{\vect{s} = \vect{v}_0\,t + (1/2)
           \vect{a}\,t^2}.
However, there is an important class of problems in physics in which the force
is not constant but varies as a function of the position of the particle.
The gravitational force and the force exerted by a stretched spring are
examples.

With the introduction of work, power, and energy, we have alternative
methods for the solution of dynamics problems, methods that involve
scalar equations rather than the vector equations required in the direct
application of Newton's laws.

More important than the alternative methods themselves is the
concept of energy\Index{energy} and the conservation law associated with it.
The principle of conservation of energy is universal: it holds in
all cases if all energy is carefully accounted for.
It is true even for areas of physics where Newton's laws are not valid,
as in the atomic-molecular-nuclear world.
It is of major interest in an energy-conscious world.
}% /Sect
%
\Sect{2}{Work}{\SectType{TextMultiPara}}
{
%
\pcap{2}{a}{Meanings Associated with Work}
Ever since your childhood you have heard and used the term \Quote{work.}
Family members went to work.
You were encouraged to work hard in school.
You worked on your car, or it was difficult work riding a bike up the hill.
All of us have an intuitive feeling about what is meant by work.
However, it is necessary for the scientist/technologist to have a precise
definition for meaningful communication with other professionals.

Technically, \Quote{work}\Index{work} is the amount of energy transferred into or out
of a definite mechanical system through the action of a mechanical force acting
on that system along a finite trajectory.
By conservation of energy, work done on a system enhances its energy while
work done by a system depletes it.
If we can calculate this energy, we can often use it to then calculate
what happens to various properties of the system.

\pcap{2}{b}{Definition for Constant Effective Force}
We start with the special case where the effective force acting on an object is
constant:
%
\textbox{{\bf Work} (for a constant effective force): the product of the
\textit{effective} force acting on an object times the path-wise displacement of
the point of application of the force.}
%
By \Quote{effective force}\Index{effective force}\Index{force| effective} we mean
the component of the force in the direction of
the displacement: it is that portion of the force that is effective in doing
work on the object.
The \Quote{pathwise displacement} is the distance along its path that the point of
application of the force moves while the force is being applied.

In \Figref{1}, \m{\theta} is the angle between the actual applied force \vect{F}
and the horizontal displacement vector \vect{s} (not shown).
The effective force is \m{|\vect{F}| \cos\theta = F \cos\theta}.
This implies that in order for work to be done: (a) a force must
act upon an object: (b) the point of application of the force must move through
a displacement; and (c) the force must not be perpendicular to the displacement.
Unless these conditions are fulfilled, no \Emph{technical} work has been done.
Just thinking about this definition, or holding this page to read it, is not
scientific \Quote{work.}
However, if you take your pencil and copy the definition, you will be doing
\Quote{work} in the technical sense of the word.

\CaptionedFullFramedFigure{1}{The \textit{effective force} for horizontal displacement.}{m20gr01}

\pcap{2}{c}{Work Done by a Constant Force}
By our definition, the work done by a constant force\Index{constant force, work done by}\Index{work| done by constant force} \vect{F} that makes a
constant angle \m{\theta} with the direction of the displacement \vect{s} (see
\Figref{1}) is:
%
\Footnote{1}{See \Quote{Vectors I: Products of Vectors} (MISN-0-2).}
%
\Eqn{1}{W = |\vect{F}|\; \cos\theta\; |\vect{s}| = F\,s\,\cos\theta\,.}
%
We recognize that the right hand side of this equation has the same form as
the scalar (or dot) product of the two vectors \vect{F} and \vect{s}, so we
can express the work done as:
%
\Eqn{2}{W = |\vect{F}|\,|\vect{s}|\,\cos\theta = \vect{F}\,\cdot\,\vect{s}\,.}
%
Work is a scalar quantity, although the force and displacement involved in
its definition are vector quantities.

Notice that we can write \Eqnref{1} either as
(\m{F \cos\theta) \times s} or as \m{F \times (s \cos\theta)}.
This suggests that the work can be calculated in two different ways: either
we multiply the magnitude of the displacement by the component of the force
in the direction of the displacement or we multiply the magnitude of the
force by the component of the displacement in the direction of the force.
These two ways are entirely equivalent.

Work\Index{work| sign of} can be positive or negative since \m{\cos\theta} can take on
positive or negative values (\m{-1 \leq \cos\theta \leq + 1}).
If the force acts in the displacement direction, the work is positive.
If the force acts in the opposite direction to the displacement, the work is
negative.
For example, consider a person lowering an object to the floor.
In this case \vect{F} points up and \vect{s} points down.
While lowering the object, negative work is done by the upward
force of the person's hand.

\pcap{2}{d}{Units of Work}
The units of work\Index{SI units| of work}\Index{units| of work}\Index{work| units of}
are products of units of force and units of distance.
In SI units, work is expressed in joules\Index{joule}, abbreviated J.
A joule is a newton-meter: one joule is the work done by a force of one newton
when it moves a particle one\,\unit{meter} in the same direction as the force.
Recalling that \m{\unit{N} = \unit{kg\,m/s\up{2}}} we have that:
%
\Eqn{}{\unit{J} = \unit{N\,m} = \unit{kg\,m\up{2}/s\up{2}}\,.}
%
The name joule was chosen to honor James Joule (1816-1889), a British
scientist famous for his research on the concepts of heat and energy.

\CaptionedFullFramedFigure{2}{One-body force diagram for a pulled
car (see text).}{m20gr02}

\pcap{2}{e}{Illustration of the Work Concept} An Example/Problem:
A Toyota (mass equal to \m{1.0 \times 10^3\unit{kg}}) that had run out of gas was
pulled down a level street by 3 people, each exerting \m{7.0 \times 10^2\unit{N}} of
force on a rope inclined at {30.0\degrees} to the horizontal.
The motion was at constant velocity because of friction, mostly between the
tires and the street.
The people pulled the car for one block (150\,\unit{meters}) before becoming tired
and quitting.
%
\begin{itemize}
\item How much work was done by the people?
\begin{eqnarray*}
F_\text{people} & = & \sum_{i=1}^3 F_i = 3(700\unit{N}) = 2100\unit{N} \\
W_\text{people} & = & \vect{F} \cdot \vect{s} = F (\cos\theta)\,s =
                      (2100\unit{N})(\cos30\degrees) (150\unit{m}) \\
                & = & 2.7 \times 10^5\unit{J}. \quad \text{(2 digits of accuracy)}
\end{eqnarray*}
\item How much work was done by the normal force \vect{N}?

Since \m{\sum \vect{F}_\text{vertical} = 0}, then \m{\vect{N} = (mg - F \sin\theta)\,\uvec{y}}.

\item[] \m{W_\text{normal} = \vect{N} \cdot \vect{s} = 0}, since \m{\vect{N} \perp \vect{s}}.
\end{itemize}
\begin{itemize}
\item How much work was done by the gravitational force, \m{\vect{F}_g}?\newline
The answer:
\m{W_\text{gravity} = \vect{F} \cdot \vect{s}} which is also zero, since \m{\vect{F}_g \perp \vect{s}}.

These last two cases emphasize that, whenever \m{\theta = 90\degrees}, the work done will be zero.

\item Where did the energy go?
We found that \m{2.7 \times 10^5\unit{joules}} of work were performed, by the people,
on the car-earth system.
This means that the people lost, and the car-earth system gained,
\m{2.7 \times 10^5\unit{joules}} of energy.
What happened to that energy?
Some might have gone into energy of motion, kinetic energy of the car, but
the car wound up not having any motion.
In fact, the energy went into heating the pavement, the tires, the axles,
the wheel bearings, the wheel bearing grease, and eventually the air
surrounding these items as they cooled off.
Finally, the atmosphere radiated some of the energy out into space and it
became lost to the earth.
\end{itemize}

\pcap{2}{f}{Graphical Interpretation of Work}
For a graphical interpretation of the work\Index{work| graphical interpretation of}
concept we plot the effective
force (\m{F \cos\theta}) versus the displacement during the interval from the
initial position \m{s_i} to the final position \m{s_f}.
The work done is \m{(F \cos\theta)(s)}, the area under the curve shown in
\Figref{3}.

\CaptionedFullFramedFigure{3}{The effective force versus the displacement.}{m20gr03}
}% /Sect
%
\Sect{3}{Work Done by Variable Forces}{\SectType{TextMultiPara}}{
%
\pcap{3}{a}{Work During Infinitesimal Displacement}
Let us now consider the more usual case where the work is done by a force whose
value will depend on the position of the point of application.
For a force that is changing only in magnitude, we can represent the
situation graphically as in \Figref{4}.

In order to find the work done for some displacement, we imagine dividing the
displacement into a very large number of infinitesimal intervals.
The work done by a force \m{\vect{F}(s)} during any one infinitesimal
displacement \m{d\vect{s}} is given by:
%
\Eqn{4}{dW = \vect{F}(s) \cdot d\vect{s}\,.}
%
\CaptionedLeftFramedFigure{4}{A variable force versus displacement.}{m20gr04}

In order to obtain the total work done, which is a finite measurable
parameter, we must sum up (integrate) those (infinitesimal) increments of
work.

\pcap{3}{b}{One Dimensional Motion: An Integral}
As a first stab at integrating \Eqnref{4}, let us investigate the
situation where the force and the displacement are along the same line of
action (say the \m{x}-axis) and the force is a known function of the position
\m{x}.
That is, \m{\vect{F} = F(x) \uvec{x}} where \m{F(x)} is known.
During a small displacement \m{dx}, so \m{d\vect{s} = dx \uvec{x}}, the force
does an amount of work \m{dW} given by:
%
\Eqn{}{dW = \vect{F} \cdot d\vect{s} = F(x)\,dx\,.}
%
\CaptionedLeftFramedFigure{5}{The area under the curve represents Work.}{m20gr05}

To obtain the work over a finite interval we sum these infinitesimal
contributions by integrating.
As the force moves the particle from \m{a} to \m{b} the work varies from zero to
its final value:
%
\Eqn{}{\int_0^W dW = \int_a^b F(x)\,dx}
%
or
%
\Eqn{5}{W_{a\rightarrow b} = \int_a^b F(x)\,dx\,.}
%
Graphically, the work is the area under the curve of \m{F(x)} versus \m{x} (see \Figref{5}).
In order to calculate it we did not need to know the actual details of the
motion, such as velocity as a function of time.
Note that the graphical representation illustrates that work requires a
displacement (if we are to have an area under the curve) and the notation
\m{W_{a\rightarrow b}} also serves to remind us of this fact.
Work is not%

\CaptionedFullFramedFigure{6}{Force characteristics of a
spring.
(a) The spring force \m{F} as a function of its displacement \m{x}.
(b) The spring in its equilibrium state.
(c) The spring stretched by a displacement \m{x} to the right and with a
spring force \m{F} to the left.
(d) The spring compressed with a displacement \m{x} to the spring force \m{F} to
the right.}{m20gr06}

\CaptionedLeftFramedFigure{7}{The work done by a force compressing a
spring from 0 to \m{x} is the area \m{(1/2) k x^2} under the \Quote{force versus
displacement} curve.}{m20gr07}

\noindent
a function of a single position in space like \m{F(x)}.
Work at a point has no meaning; only over a displacement is it meaningful.

\pcap{3}{c}{Example: a Stretched Spring}
As a helical spring is stretched or compressed, away from its equilibrium
position, the spring resists with a force that is fairly
accurately linear (unless it is poorly made or becomes deformed).
That is, if a \Quote{linear} spring is stretched or compressed a distance \m{x},
it resists with a force \m{F = -k x}.
Here the (\m{-}) sign indicates that the spring's force is
opposite to the direction of the displacement from equilibrium.
We call this a \Quote{restoring} force since it tends to restore the spring
to its equilibrium position.
The quantity \m{k} is called the \Quote{spring constant}: it is a measure of the
\Quote{stiffness} of the spring.
By the way, saying the spring is in its \Quote{equilibrium position} merely
means that it is neither compressed nor stretched.

Suppose one finds that a 36\unit{N} force compressed a particular spring by
6.0\unit{cm}.
How much work is done by the force if it compresses the spring by 5.0\unit{cm}?
First, note that the value of the spring constant is:
%
\Eqn{}{k = -\dfrac{F}{x} = \dfrac{-36\unit{N}}{-6.0 \times 10^{-2}\unit{m}} =
           6.0 \times 10^2\unit{N/m}\,.}
%
Then the work done by the compressing force is:
%
\Eqn{}{W = \int \vect{F}_R \cdot d\vect{s} = \int_{-x}^0 (-k x)\,dx\,,}
%
since \vect{F} is parallel to \m{d\vect{s}} and is in the same direction.
Thus:
%
\Eqn{}{W = - k \int_{-x}^0 x\,dx = \dfrac{k x^2}{2} = 0.75\unit{J}\,.}
%
This transfer of energy depleted the energy of the system applying the force
and increased the (internal) energy of the spring.

\CaptionedLeftFramedFigure{8}{A force whose point of application
follows a curved path.}{m20gr08}

We can arrive at the same result graphically by calculating the area under
the \Quote{\m{F} versus \m{x}} curve.
Since the area of a triangle is half its height times its base, we have:
%
\Eqn{}{\dfrac{1}{2}(-k x)(-x) = \dfrac{1}{2}k x^2\,.}
%

\pcap{3}{d}{General Motion: A Line Integral}
\Index{integral| line}\Index{line integral}The force \vect{F} doing work may
vary in direction as well as in
magnitude, and the point of application may move along a curved path.
To compute the work done in this general case we again divide the path up
into a large number of small displacements \m{d\vect{s}}, each pointing along
the path in the direction of motion.
At each point, \m{d\vect{s}} is in the direction of motion.

The amount of work done during a displacement \m{d\vect{s}} is:
%
\Eqn{}{dW = \vect{F}(s) \cdot d\vect{s} = F(s) (\cos\theta)\,ds\,,}
%
where \m{F(s) \cos\theta} is the component of the force along the tangent to
the trajectory at \m{d\vect{s}}.
The total work done in moving from point \m{s_i} to point \m{s_f} is the
sum of all the work done during successive infinitesimal displacements:
%
\Eqn{}{W = \vect{F}_1 \cdot d\vect{s}_1 + \vect{F}_2 \cdot d\vect{s}_2 +
\vect{F}_3 \cdot d\vect{s}_3 + \ldots}
%
Replacing the sum over the line segments by an integral, the work\Index{work| done by variable force}
is found to be:
%
\Eqn{6}{W_{A\rightarrow B} = \int_A^B \vect{F}(s) \cdot d\vect{s} =
\int_A^B F(s) (\cos\theta)\,ds\,,}
%
where \m{\theta} is a function of \m{s}, the position along the trajectory.
This is the most general definition of the work done by a force \m{\vect{F}(s)}.
We cannot evaluate this integral until we know how both \m{F} and \m{\theta}
vary from point to point along the path.

\CaptionedLeftFramedFigure{9}{The total work is the sum over successive
infinitesimal displacements.}{m20gr09}

For any vector \vect{V} which is a function of position, an integral of the
form
%
\Eqn{}{\int_i^f \vect{V} \cdot d\vect{s}\,,}
%
along some path joining points \m{i} and \m{f}, is called \Quote{the line integral of
\vect{V}.}
\Equationref{6} is of this nature because it is evaluated along the
actual path in space followed by the particle as it moves from \m{i} to \m{f}.

\Equationref{6} is the \Quote{line integral definition of work.}\Index{work| line integral definition of}
For each increment of displacement \m{d\vect{s}} along the path, the
corresponding increment of work \m{dW = \vect{F} \cdot d\vect{s}} is calculated
and then these scalar quantities are simply summed to give the work along
the total path.

We can obtain an equivalent general expression for \Eqnref{6} by
expressing \vect{F} and \m{d\vect{s}} in scalar component form.
With \m{\vect{F} = F_x \uvec{x} + F_y \uvec{y} + F_z \uvec{z}} and
\m{d\vect{s} = dx \uvec{x} + dy \uvec{y} + dz \uvec{z}}, the resulting work done in
going from position \m{A = (x_A, y_A, z_A)} to position \m{B = (x_B, y_B, z_B)} can
be expressed as:
%
\Eqn{7}{W_{i\rightarrow f} =
 \int_{x_i}^{x_f} F_x\,dx + \int_{y_i}^{y_f} F_y\,dy +
\int_{z_i}^{z_f} F_z\,dz\,,}
%
where each integral must be evaluated along a projection of the path.
%
\SubSubSect{}{Example:}{
At Space Mountain in Disney World the space rocket ride is simulated by a
cart which slides along a roller coaster track (see cover) which we will
consider to be frictionless.
Starting at an initial position \#1 at height \m{H} above the ground, find the
work done on this rocket when you ride it to the final position \m{f} at the
bottom of the track (see \Figref{10}).

The forces that act on this \Quote{rocket} are: the force of gravity,
\m{\vect{F}_g = - m g \uvec{y}} and the \Quote{normal} surface reaction force
\vect{N}.
The total work done by the resultant force \m{\vect{F}_R = \vect{F}_g + \vect{N}}
on the rocket between the two end points is:
%
\Eqn{}{W_{i\rightarrow f} = \int_i^f \vect{F}_R \cdot d\vect{s} =
                            \int(\vect{F}_g + \vect{N}) \cdot d\vect{s} =
                            \int\vect{F}_g \cdot d\vect{s} + \int \vect{N} \cdot d\vect{s}.}
%

\CaptionedFullFramedFigure{10}{Force diagram for rocket cart.}{m20gr10}

Since the surface force is always perpendicular to the path, it does no
work.
That is, \m{\int \vect{N} \cdot d\vect{s} = \int N \cos\phi\,ds = 0}, since
the angle between \vect{N} and \m{d\vect{s}} is always {90\degrees}.
This is true despite the fact that the angle between \m{\vect{F}_g} and
\m{d\vect{s}} changes continuously as the rocket goes down the track.

Now we can write \m{d\vect{s} = dx \,\uvec{x} + dy \,\uvec{y}} so that:
\begin{eqnarray*}
W_{i\rightarrow f} & = & \int \vect{F}_g \cdot d\vect{s} =
\int (-m g \uvec{y}) \cdot (dx \uvec{x} + dy \uvec{y}) \\
                   & = & - \int_H^0 m g\,dy = - m g \int_H^0\,dy = m g H
\end{eqnarray*}
Here we see that the line integral reduces to a simple summation of the
elements \m{dy}, which are the projections of \m{d\vect{s}} on the constant
direction \m{\vect{F}_g}.
The answer, \m{m g H}, is an important one to remember.
It is the energy of the earth-plus-rocket system that was transferred from
gravitational energy to mechanical energy.
That answer holds for any earth-plus-object system.
}% /SubSubSect
}% /Sect
%
\Sect{4}{Power}{\SectType{TextMultiPara}}{
%
\pcap{4}{a}{Definition of Power}
\Index{power}Let us now consider the time involved in doing work.
The same amount of work is done in raising a given body through a given
height whether it takes one second or one year to do so.
However, the rate at which work is done is often as interesting to us as is
the total work performed.
When an engineer designs a machine it is usually the time rate at which the
machine can do work that matters.

Instantaneous power\Index{power| instantaneous} is defined as the time rate at which work is being done
at some instant of time.
That is, it is the limit, as the time interval approaches zero, of the amount
of work done during the interval divided by the interval.
Since this is the definition of the time derivative, we have:
%
\Eqn{14}{P=\dfrac{dW}{dt}\,.}
%
For constant force or velocity, using \Eqnref{4} and \m{\vect{v} =
d\vect{s}/dt} we get:
%
\Eqn{15}{P = \vect{F}_\text{const}\, \cdot \vect{v}             \qquad ; \qquad
         P = \vect{F}               \cdot \vect{v}_\text{const} \,.}
%
The average power\Index{power| average} \m{P_{av}} during a time interval \m{\Delta t} is:
%
\Eqn{}{P_{av} =\dfrac{W}{\Delta t}\,.}
%
If the power is constant in time, then \m{P_\text{const} = P_{av}} and
%
\Eqn{16}{W = P_\text{const}\,\Delta t\,.}
%

\pcap{4}{b}{Units of Power}
\Index{units| of power}According to the definition of power, its units are units of work divided by
units of time.
In the MKS system, the unit of power\Index{power| units of} is called a watt\Index{watt},
abbreviated W, which is equivalent to a joule per second.
One watt is the power of a machine that does work at the rate of one joule
every second.
Recalling that \m{\unit{J} = \unit{m\up{2}\,kg/s\up{2}}}, we have that:
%
\Eqn{}{\unit{W} = \unit{J/s} = \unit{m\up{2}\,kg/s\up{3}}\,.}
%
The name watt was chosen in honor of the British engineer James Watt
(1736-1819) who improved the steam engine with his inventions.

Work can be expressed in units of \m{\text{power } \times \text{ time}}.
This is the origin of the term kilowatt-hour (kWh).
One kilowatt-hour is the work done in 1\,\unit{hour} by an agent working at a
constant rate of 1\unit{kW} (1000\unit{W}).
Electricity is sold \Quote{per \unit{kWh}.}
%
\SubSubSect{}{Example:}{
Under very intense physical activity the total power output of the heart
may be 15\,unit{watts}.
How much work does the heart do in one minute at this rate?
%
\Eqn{}{W = P_\text{const} \, \Delta t = (15\unit{W})\,(60\unit{s}) =
900\unit{J}\,.}
%
}% /SubSubSect
%
}% /Sect
%
\Sect{5}{Kinetic Energy}{\SectType{TextMultiPara}}{
%
\pcap{5}{a}{Definition of Kinetic Energy}
A particle's kinetic energy is defined as the amount of energy a particle
has, due solely to its velocity.
Using Newton's second law and the definition of total work, one can show that
the kinetic energy, \m{E_k}, of a particle of mass \m{m} traveling at velocity
\m{v} is:
%
\Eqn{9}{\text{KINETIC ENERGY } = E_k =\dfrac{1}{2}m v^2\,.}
%
This is valid whenever Newtonian mechanics is valid.
We can easily illustrate this derivation for the special case of a resultant
force, \m{\vect{F}_R}, that acts on a particle of mass \m{m} along the direction
of its displacement.
For this case the total work done on the particle is:
%
\Eqn{10}{W_{T,i\rightarrow f} = \int_{s_i}^{s_f} \vect{F}_R \cdot \,d\vect{s} =
                              \int_{x_i}^{x_f} F_R \,dx\,.}
%
Since the force, hence the acceleration, is along the direction of
displacement, we can use \Eqnref{10} and Newton's second law
%
\Footnote{2}{See \Quote{Free-Body Force Diagrams, Frictional Forces, Newton's Second
Law} (MISN-0-15).}
%
to write:
%
\TwoEqns{11}{W_{i\rightarrow f} & = \int_{x_i}^{x_f}  \,m \,a \,dx =
                                    \int_{x_i}^{x_f}  \,m\,\dfrac{dv}{dt}dx =
                                    \int_{x_i}^{x_f}  \,m\,dv\,\dfrac{dx}{dt}}
            {                   & = \int_{x_i}^{x_f}  \,m\,v\,dv =
                                    m \int_{x_i}^{x_f} \,v\,dv =
                                    \dfrac{m v_f^2}{2}-\dfrac{m v_i^2}{2}\,.}
%
Thus the work that went into accelerating the particle, from \m{v_i} to \m{v_f},
is exactly equal to the change in the kinetic energy for that particle.
Note that kinetic energy is a scalar quantity.
It is, as we shall see, just as significant as the vector quantity, momentum,
that is also a quantity used to describe a particle in motion.

\Index{kinetic energy| units of}From its definition, kinetic energy has dimensions
\m{M L^2/T^2}, either mass
multiplied by the square of speed, or, since it is equivalent to work [as
shown in \Eqnref{11}], force multiplied by distance.
Thus, the kinetic energy of a particle may be expressed in \textit{joules}.
It follows that a 2\unit{kg} particle moving at 1\unit{m/s} has a kinetic energy of
1\unit{J}.

The kinetic energy of a particle can be expressed in terms of the
magnitude of its linear momentum, \m{m \vect{v} = \vect{p}}:
%
\Eqn{12}{E_k=\dfrac{1}{2}mv^2=\dfrac{(mv)^2}{2m}=\dfrac{p^2}{2m}=
                \dfrac{|\vect{p}|^2}{2m}\,}
%
\SubSubSect{}{Example:}{
A particle of mass \m{m} starts from rest and falls a vertical distance \m{h}.
What is the work done on this particle and what is its final kinetic energy?

The particle experiences only a single constant force,
\m{\vect{F} = - m g\,\uvec{y}}, in the downward direction.
The displacement, \m{d\vect{s} = -\,dy\,\uvec{y}}, is also downward for a
distance \m{h}.
Therefore the work done by the gravitational force is:
%
\Eqn{}{W = \int\,\vect{F} \cdot d\vect{s} =
   \int_0^h (-m g\,\uvec{y})\cdot(-dy\,\uvec{y}) =
   \int_0^h \,m\,g\,dy = m\,g\,h\,.}
%
Since this is the only work done on the particle, it is equal to the final
kinetic energy:
%
\Eqn{}{m g h = \dfrac{1}{2}m v^2\,.}
%
Solving for velocity:
%
\Eqn{}{v = \sqrt{2\,g\,h\,}\,,}
%
which is the same result we obtain from kinematics for an object falling
with constant acceleration \m{g} (a \Quote{freely-falling} object).
}% /SubSubSect
%
\SubSubSect{}{Example:}{
It is possible for a person with a mass of \m{7.0 \times 10^1\unit{kg}} to fall to
the ground from a height of 10.0\,\unit{meters} without sustaining an injury.
What is the kinetic energy of such a person just before hitting the ground?
The velocity of the person is:
%
\Eqn{}{v = \sqrt{2\,g\,h} = \sqrt{2(9.8\unit{m/s\up{2}})(10\unit{m})} = 14\unit{m/s}\,.}
%
Then the kinetic energy of the person is:
%
\Eqn{}{E_k = \dfrac{1}{2}m v^2 =
             \dfrac{1}{2}(70\unit{kg})(14\unit{m/s})^2 = 6.9 \times 10^3\unit{J}\,.}

\pcap{5}{b}{The Energy Concept}
Of all the concepts of physics, that of energy is perhaps the most
far-reaching.
Everyone, whether a scientist or not, has an awareness of energy and what it
means.
Energy\Index{energy} is what we have to pay for in order to get things done.
The work itself may remain in the background, but we recognize that each
gallon of gasoline, each Btu of heating gas, each kilowatt-hour of
electricity, each calorie of food value, represents, in
one way or another, the wherewithal for doing something.
We do not think in terms of paying for force, or acceleration, or momentum.
Energy is the universal currency that exists in apparently countless
denominations, and physical processes represent a conversion from one
denomination to another.

The above remarks do not really define energy.
No matter.
It is worth recalling the opinion of H.\,A.\,Kramers:
\Quote{The most important and most fruitful concepts are those to which it is
impossible to attach a well-defined meaning.}
The clue to the immense value of energy as a concept lies in its
transformation.
We often find energy defined in general as the ability to do work.
A system possesses energy; it can do work.
At any instant a system has a certain energy content.
Part or all of this energy can be transformed into the activity of work.
Work is only an active measure of energy and not a form of energy itself.
Work is best regarded as a mode of transfer of energy from one form to
another.
It is a medium of exchange.
In this unit we are dealing with only one category of energy--the kinetic
energy associated with the motion of an object.
If energy should be transferred from this form into chemical energy,
radiation, or the random molecular and atomic motion we call heat, then from
the standpoint of mechanics it is gone.
This is a very important feature, because it means that, if we restrict our
attention purely to mechanics, conservation of energy does not hold.
Nevertheless, as we shall see, there are many physical situations in which
total mechanical energy is conserved, and in such contexts it is of
enormous value in the analysis of real problems.
}% /Sect
%
\Sect{6}{The Work-Kinetic Energy Relation}{\SectType{TextMultiPara}}{
%
\pcap{6}{a}{Derivation of the Relation}
\Index{work-energy relation}\Equationref{11} shows that the work done on a particle, when the
resultant force acting on the particle is in the direction of the
displacement, is equal to the change in the particle's kinetic energy.
For the more general case of a force that is not in the direction of motion,
we can still derive this valuable relation between the work done and the
change in kinetic energy.

Write \m{\vect{F}_R} for the resultant force acting upon a particle of mass \m{m},
moving along a path between the two positions \m{s_i} and \m{s_f}, then:
\begin{eqnarray*}
\sum W_{i\rightarrow f} & = &
        \int_{s_i}^{s_f} \vect{F}_R \cdot \,d\vect{s} =
        \int m \vect{a}\,\cdot\,d\vect{s} =
        m \int \dfrac{d\vect{v}}{dt}\,\cdot\,d\vect{s} \\
                        & = &
        m \int \,d\vect{v} \cdot \dfrac{d\vect{s}}{dt} =
        m \int\,d\vect{v}\cdot \vect{v}\,. \\
\end{eqnarray*}
\noindent
Thus:
%
\Eqn{}{\sum W_{i\rightarrow f} = m \int_{v_i}^{v_f}\,d\vect{v} \cdot \vect{v} =
   \dfrac{mv_f^2}{2}-\dfrac{mv_i^2}{2} = E_{k,f} - E_{k,i} =\Delta E_k\,.}
%
The evaluation of the integral in this last step can be easily seen if you
write \vect{v} and \m{d\vect{v}} in scalar component notation.
%
\Footnote{3}{For additional discussion of the mathematical steps presented in
this derivation see \Emph{Newtonian Mechanics}, A.\,P.\,French, W.\,W.\,Norton
\& Co. (1971), pp.\,368-72, or \Emph{Physics for Scientists and Engineers,
Volume 1}, Melissinos and Lobkowicz, W.\,B.\,Saunders Company (1975),
p.\,170.}
%
In summary,
%
\Eqn{13}{W_{i\rightarrow f} = \int_i^f \,\vect{F}_R\,\cdot\,d\vect{s} =
                    \Delta E_k\,.}
%
\Equationref{13} is known as the Work-Kinetic Energy Relation and it
is valid no matter what the nature of the force:
\begin{itemize}
\item [] \textit{The total work done on a particle (by the resultant force
acting on it), between some starting point \m{s_i} and ending point \m{s_f},
is the change in the particle's kinetic energy between those two end points.}
\end{itemize}

\noindent
This relation applies quite apart from the particular path followed, so long
as the total work done on the particle is properly computed from the
resultant force.

\pcap{6}{b}{Significance of the Relation}
The work-kinetic energy relation is not a new independent relationship of
classical physics.
We have derived it directly from Newton's second law, utilizing the
definitions of work and kinetic energy.
This relation is helpful in solving problems where the work done by the
resultant force is easily computed, or where we are interested in finding the
speed of a particle at a particular position.
However, we should recognize that the work-kinetic energy principle is the
starting point for formulating some sweeping generalizations in physics.
We have stressed that this principle can be applied when \m{\Sigma W} is
interpreted as the work done by the resultant force acting on a particle.
In many cases however, it is more useful to compute separately the work done
by each of certain types of forces which may be acting and to give these
special designations.
This leads us to the identification of different types of energy, and the
principle of the conservation of energy.
%
\Footnote{4}{See \Quote{Potential Energy, Conservative Forces, the Law of
Conservation of Energy} (MISN-0-21).}
%
}% /SubSubSect
%
\SubSubSect{}{Example:}{
\noindent
(a) How much work is required to stop a \m{1.0 \times 10^3\unit{kg}} car that is
moving at a speed of 20.0\unit{m/s}?
\begin{eqnarray*}
W_{T,i\rightarrow f} & = & E_{k,f} - E_{k,i} = 0 -\dfrac{1}{2}m v_i^2 =
                           0 - \dfrac{1}{2}(100\,0\unit{kg})(20\unit{m/s})^2 \\
                     & = & - 2.0 \times 10^5\unit{J}. \\
\end{eqnarray*}
The negative (\m{-}) sign indicates that there was a decrease in the kinetic energy of
the car; that is, work was done on the car by a braking force opposing
its motion.

\noindent
(b) If this car requires 100 meters to come to rest, what is the total
resultant braking force acting on the car?
%
\Eqn{}{W_{i\rightarrow f} = \int_i^f\vect{F}_R \,\cdot\, d\vect{s} = -f \int_{s_i}^{s_f} \,ds\,,}
%
since the total resultant force can be represented by a single constant
braking force f which directly opposes the motion.
Thus:
%
\Eqn{}{-2 \times 10^5\unit{J} = - f(100\unit{m}) \qquad \text{ or } \qquad f = 2 \times 10^3\unit{N}\,.}
%
}% /SubSubSect
%
\SubSubSect{}{Example:}{
In a previous example we found that a stretched spring did an amount of work
\m{k x^2/2} on an object when the spring returned to its equilibrium position
from a displacement \m{x}.

We can now use the Work-Kinetic Energy relation to easily find the velocity
of the block as it passes the equilibrium position:
%
\Eqn{}{W_{i\rightarrow f} = \Delta E_k}
%
\Eqn{}{\dfrac{k x^2}{2} = \dfrac{m v_f^2}{2} - \dfrac{m v_i^2}{2}\,.}
%
Since it initially started from rest we find:
%
\Eqn{}{v_f = x \, \sqrt{k/m}\,.}
}% /SubSubSect
%
}% /Sect
%
\Sect{}{Acknowledgments}{\SectType{Acknowledgments}}%
{Professor James Linneman made several helpful suggestions. \NsfAcknowledgment
}% /Sect

