\revhist{7/2/85, mpm; 12/20/88 ejdk; 8/30/91, pss; 4/20/94, pss; 12/18/2000, pss}

\Sect{}{}{\SectType{ProblemSet}}{

\noindent All four problems also occur in this module's \textit{Model Exam}.

\begin{one-digit-list}
\item [1.] Consider the potential energy function \m{E_p(x) = 7 x^2 - x^3}.
Assume the particle interacting with this potential has mass \m{M = 60}.
For this problem let us neglect units.
Draw a rough sketch of \m{E_p(x)}. \answer{18}
At a certain instant, the particle is at \m{x = +1} with total energy \m{E = 36}
and is moving to the right.
\begin{one-digit-list}
\item [a.] What is the kinetic energy of the particle at this point? \answer{11}
\item [b.] What is its velocity? \answer{14}
\item [c.] What is the instantaneous force this particle feels at this point? \answer{5}
\item [d.] What is the acceleration (magnitude, direction) this particle
undergoes? \answer{9}
\item [e.] Therefore, is this particle's speed increasing or decreasing at this
point? \answer{16}
Explain. \answer{19}
\item [f.] With this particle moving to the right, and from the shape of the
\m{E_p} curve in the vicinity of this point (\m{x = 1}) determine whether the
kinetic energy of the particle will increase or decrease. \answer{3}
Is this consistent with your answer of (e)? \answer{12}
\item [g.] For what value of \m{x} will the speed of this particle (moving to the
right) be zero?
(You may not actually be able to find the numerical value of \m{x} for which this
occurs; however, if you do it right you'll have the equation satisfied by \m{x}
at this point on the graph you've drawn.) \answer{1}
\item [h.] At this point where \m{v = 0} what is the direction of the
acceleration of this particle?
(Get this from determining the sign of the force on the particle as
determined from the slope of the curve.) \answer{10}
With this direction for the acceleration what is the direction of \m{v} an
instant after the instant when \m{v = 0}, hence what is the direction of the
subsequent motion of the particle? \answer{15}
\item [i.] On your \m{E_p(x)} plot, show the region of those values of \m{x}
forbidden to the particle when it has total energy \m{E = 36}. \answer{7}
\item [j.] What are the turning points of the motion of the particle when its
total energy is 36? \answer{25}
\end{one-digit-list}

\item [2.] Consider the same situation as in Problem~1 except that at point
\m{x = - 3.0} the particle of mass \m{M = 60} starts out with speed \m{v = 0}.
\begin{one-digit-list}
\item [a.] What is its kinetic energy? \answer{4}
\item [b.] What is its potential energy? \answer{20}
\item [c.] What is its total energy? \answer{24}
\item [d.] What is the force (magnitude and direction) on the particle? \answer{13}
\item [e.] What is the magnitude of its acceleration? \answer{22}
\item [f.] What is the direction of its acceleration? \answer{2}
\item [g.] When it gets to \m{x = 0}, what is its speed? \answer{8}
\item [h.] When it gets to \m{x = 14/3}, what is its speed? \answer{21}
\item [i.] What is its acceleration (magnitude and direction) at \m{x = 14/3}? \answer{17}
\item [j.] How far will it go before it turns around? \answer{23}
\item [k.] Explain your answer to (j). \answer{6}
\end{one-digit-list}

\item [3.] A potential energy function is given by \m{E_p(r) = - 1/2\,k r^2}
where \m{r} is the distance from the origin of the coordinate system to a point
in space (\m{k} is a constant).
\begin{one-digit-list}
\item [a.] Find the magnitude of the force on a particle placed at a point
which is at a distance \m{r} from the origin. \answer{26}
\item [b.] Find the direction of this force. \answer{27}
\end{one-digit-list}

\item [4.] \NullItem
\begin{one-digit-list}
\item [a.] If the potential energy of a particle is \m{E_p = k/r}, \m{k} being
a constant equal to 10\unit{N\,m\up{2}}, find the force on the particle when it is
at the point \m{x = 0\unit{m}}, \m{y = 1\unit{m}}, \m{z = 2\unit{m}}. \answer{28}
\item [b.] What is the direction of the force? \answer{29}
\end{one-digit-list}
\end{one-digit-list}

\newpage

\BriefAns

\begin{two-digit-list}
\item [1.] One of the solutions of \m{x^3 - 7 x^2 + 36 = 0}.
There are 3 solutions, call them \m{x_1}, \m{x_2}, \m{x_3} with
\m{x_1 < x_2 < x_3} (refer to graph, notice \m{x_1} is negative).
\m{x_2} is the answer to this question.
The actual values are \m{x_1 = -2}, \m{x_2 = 3}, \m{x_3 = 6}.
\item [2.] To right.
\item [3.] Decreases.
\item [4.] Zero.
\item [5.] \m{F_x = -11} (to left).
\item [6.] After point \m{x = 14/3}, the potential energy continues to decrease
to infinity, total energy remains constant, kinetic energy increases, never
 goes to zero, so \m{v} never changes direction.
\item [7.] Forbidden regions: \m{x < x_1}, \m{x_2 < x < x_3}.
All other regions are allowed.
The quoted regions are forbidden because the particle would have a negative
kinetic energy, hence an imaginary speed in these regions.
Verify this from the energy diagram you've sketched.
\item [8.] \m{\sqrt{3}}.
\item [9.] \m{a_x = - 11/60} (to left).
\item [10.] To left.
\item [11.] \m{E_k = 30}.
\item [12.] Yes, kinetic energy and speed both are decreasing.
\item [13.] 69 to right.
\item [14.] \m{v_x = +1}.
\item [15.] Because \m{a_x} is to left, an instant after \m{v_x = 0}, \m{v_x} is to
left.
\item [16.] Decreasing.
\item [17.] Zero.
\item [18.] A rough sketch of \m{E_p(x)} can be drawn by considering the
following.
For very large positive or negative values of \m{x} the \m{x^2} term may be
neglected compared to the \m{x^3} term.
Hence, for very large negative values the graph tends to plus infinity while
for very large positive values the graph goes to negative infinity.
The graph crosses the \m{x}-axis whenever \m{E_p(x)} is zero.
There are three roots to a cubic equation.
In this case the roots are \m{x = 0} (twice) and \m{x = 7}.
You can also determine that the graph has a minimum at \m{x = 0} and a maximum at
\m{x = 14/3}.
With this information a rough graph can be sketched.
\item [19.] \m{v_x} is positive and \m{a_x} is negative so \m{v_x} decreases.
\item [20.] 90.
\item [21.] \m{1.14 = (529/405)^{1/2}}.
\item [22.] 1.15.
\item [23.] To infinity.
\item [24.] 90.
\item [25.] \m{x_1} and \m{x_2}: using results of Answer~1 the particle moves
between points \m{x = -2} and \m{x = +3}.
\item [26.] \m{+ k r};
\item [27.] Outward radially.
\item [28.] 2\unit{newtons};
\item [29.] Outward, along the radial line going through (0,1,2).
\end{two-digit-list}

}% /Sect
