\revhist{7/2/85 mpm, 8/30/91, pss; 2/26/93, pss; 4/20/94, pss; 12/18/2000, pss}

\Sect{}{}{\SectType{ModelExam}}{

\begin{one-digit-list}
\item [1.] See Output Skill K1 in this module's \textit{ID Sheet}.

\item [2.] Assume a particle has mass \m{M = 60} and potential energy function
\m{E_p(x) = 7 x^2 - x^3}.
For this problem let us neglect units.
Draw a rough sketch of \m{E_p(x)}.
At a certain instant, the particle is at \m{x = +1} with total energy \m{E = 36}
and is moving to the right.
\begin{one-digit-list}
\item [a.] What is the kinetic energy of the particle at this point?
\item [b.] What is its velocity?
\item [c.] What is the instantaneous force this particle feels at this point?
\item [d.] What is the acceleration (magnitude, direction) this particle
undergoes?
\item [e.] Therefore, is this particle's speed increasing or decreasing at this
point?
Explain.
\item [f.] With this particle moving to the right, and from the shape of the
\m{E_p} curve in the vicinity of this point (\m{x = 1}) determine whether the
kinetic energy of the particle will increase or decrease.
Is this consistent with your answer to (e)?
\item [g.] For what value of \m{x} will the speed of this particle (moving to the
right) be zero?
(You may not actually be able to find the numerical value of \m{x} for which this
occurs: however, if you do it right you'll have the equation satisfied by \m{x}
at this point and you'll be able to identify the point on the graph you've
drawn.)
\item [h.] At this point where \m{v = 0} what is the direction of the
acceleration of this particle?
(Get this from determining the sign of the force on the particle as
determined from the slope of the curve.)
With this direction for the acceleration what is the direction of \m{v} an
instant after the instant when \m{v = 0}, hence what is the direction of the
subsequent motion of the particle?
\item [i.] On your \m{E_p(x)} plot show the region of those values of \m{x}
forbidden to the particle when it has this total energy, \m{E = 36}.
\item [j.] What are the turning points of the motion of the particle when its
total energy is 36?
\end{one-digit-list}

\item [3.] Consider the same situation as in Problem~1 except that at point
\m{x = - 3.0} the particle of mass \m{M = 60} starts out with speed \m{v = 0}.
\begin{one-digit-list}
\item [a.] What is its kinetic energy?
\item [b.] What is its potential energy?
\item [c.] What is its total energy?
\item [d.] What is the force (magnitude and direction) on the particle?
\item [e.] What is the magnitude of its acceleration?
\item [f.] What is the direction of its acceleration?
\item [g.] When it gets to \m{x = 0}, what is its speed?
\item [h.] When it gets to \m{x = 14/3}, what is its speed?
\item [i.] What is its acceleration (magnitude and direction) at \m{x = 14/3}?
\item [j.] How far will it go before it turns around?
\item [k.] Explain your answer to (j).
\end{one-digit-list}

\item [4.] A potential energy function is given by \m{E_p(r) = - 1/2\,k r^2}
where \m{r} is the distance from the origin of the coordinate system to a point
in space (\m{k} is a constant).
\begin{one-digit-list}
\item [a.] Find the magnitude of the force on a particle placed at a point
which is at a distance \m{r} from the origin.
\item [b.] Find the direction of this force.
\end{one-digit-list}

\item [5.] \NullItem
\begin{one-digit-list}
\item [a.] If the potential energy is \m{E_p = k/r}, \m{k} being a constant
equal to 10\,N\unit{m\up{2}}, find the force on the particle when it is at the point
\m{x = 0}, \m{y = 1}, \m{z = 2} (all in meters).
\item [b.] What is the direction of the force?
\end{one-digit-list}
\end{one-digit-list}

\BriefAns

\begin{one-digit-list}
\item [1.] See this module's \textit{text}.
\item [2.] See this module's \textit{Problem Supplement}, problem~1.
\item [3.] See this module's \textit{Problem Supplement}, problem~2.
\item [4.] See this module's \textit{Problem Supplement}, problem~3.
\item [5.] See this module's \textit{Problem Supplement}, problem~4.
\end{one-digit-list}

}% /Sect
