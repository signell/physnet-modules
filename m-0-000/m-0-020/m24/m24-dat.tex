\revhist{5/29/85 mpm; 4/30/87, pss; 4/1/91, pss; 12/19/91, pss; 2/14/92, pss;
         6/29/93, pss; 10/11/94, pss; 3/27/97, pss; 2/22/99, pss}
%
\defModTitle{\ph{Relativistic Momentum:} \ph{Particle Decays}}
\defCtAuthor{Peter Signell}
\defIdAuthor{P.\,Signell, Dept.\,of Physics, Mich.\,State Univ}
%
\defLG{True}
\defRD{4}
%
\defIdItems{
    \IdVersEval{2/1/2000}{B1}
    \IdHours{1}
    \begin{InputSkills}
    \item [1.]  Work simple conservation of momentum problems \prrqone{0-14}.
    \item [2.]  Use the relativistic form for total energy in conservation of energy
    problems \prrqone{0-23}.
    \item [3.]  Expand the square root function in a Taylor series about a given point
    \prrqone{0-4}.
    \end{InputSkills}
    %
    \begin{KnowledgeSkills}
    \item [K1.] Reduce the expression for relativistic momentum to its non-relativistic
    form, using the general expression for Taylor's Series for the expansion of a
    function about a point.
    \item [K2.] Show that \m{\vect{F}\,=\,m\vect{a}} is generally valid only for
    \m{v^2\,\ll\,c^2}.
    \end{KnowledgeSkills}
    %
    \begin{ProblemSolvingSkills}
    \item [S1.] Given a particle's rest mass and velocity, calculate its relativistic
    momentum and energy.
    \item [S2.] Use conservation of energy and momentum to work decay problems,
    1 particle \m{\Rightarrow} 2 particles, in the center of mass frame.
    \end{ProblemSolvingSkills}
    %
    \begin{RequiredResources}
    \item [1.]  M.\,Alonso and E.\,J.\,Finn, \textit{Physics}, Addison-Wesley (1970),
    or R.\,Resnick, \textit{Basic Concepts in Relativity and Early Quantum Theory},
    John Wiley \& Sons (1972), or R.\,T.\,Weidner and R.\,L.\,Sells,
    \textit{Elementary Modern Physics}, Allyn and Bacon (1980).
    For availability, see this module's \textit{Local Guide}.
    \end{RequiredResources}
}