\revhist{8/6/84, wcl; 3/16/88, pss; 9/7/91, pss; 9/29/94, pss; 11/30/94, pss; 2/11/02, pss}

\Sect{}{}{\SectType{ProblemSet}}{

\noindent Problems~6 and 7 also occur in this module's \textit{Model Exam}.

\begin{one-digit-list}
\item [1.] Two cars on a straight section of road are traveling parallel to
each other.
Car~1 has a speed of 60.0\unit{mi/hr} and is traveling eastward, while Car~2 has a
speed of 45\unit{mi/hr}.
Calculate the relative velocity of Car~2 with respect to Car~1 when:
\begin{one-digit-list}
\item [a.] the two cars are traveling the same direction;
\item [b.] the two cars are traveling in opposite directions.
\end{one-digit-list}

\item [2.] A speed boat capable of traveling 35\unit{m/s} in still water attempts
to cross a river 1500\unit{m} wide with a current moving at a speed of 28\unit{m/s}.
Determine the direction the boat must travel relative to the current if the
resultant path of the boat is to be straight across the river, perpendicular
to the banks.
How long will the trip take?

\item [3.] A particle \m{Q} has a position, measured with respect to coordinate
system \m{O}, of:
%
\Eqn{}{\vect{r} = (6 t^2 - 4 t) \uvec{x} - 3 t^3 \uvec{y} + 3 \uvec{z}\,.}
%
With respect to another coordinate system O', this same particle's position
is:
%
\Eqn{}{\vect{r} = (6 t^2 + 3t)\uvec{x} - 3 t^3\,\uvec{y} + 3 \uvec{z}\,.}
%
(Note: all distances are in meters, all times are in seconds.)
\begin{one-digit-list}
\item [a.] Rewrite the two position vector equations with double subscripts and
proper units.
\item [b.] Find the time at which the two observers' coordinate axes coincide.
\item [c.] Find the velocity of \m{O'} with respect to \m{O}. \help{5}
\item [d.] Find the distance between \m{O} and \m{O'} at time \m{t = 3\unit{s}}.
\item [e.] Find the distance between \m{Q} and \m{O} at time \m{t = 3\unit{s}}.
\item [f.] Find the distance between \m{Q} and \m{O'} at time \m{t = 3\unit{s}}.
\item [g.] Find the angle between \m{\vect{r}_{QO}} and \m{\vect{r}_{QO'}} at
\m{t = 3\unit{s}}.
\item [h.] Find the angle between \m{\vect{r}_{QO}} and \m{\vect{r}_{QO'}}, as seen
looking along the \m{z}-axis, at \m{t = 3\unit{s}}.
\item [i.] Draw a rough sketch showing \m{\vect{r}_{QO}}, \m{\vect{r}_{QO'}},
and \m{\vect{r}_{OO'}}, as they appear if you are looking along the z-axis toward
the origin.
Label the positions of \m{Q}, \m{O}, and \m{O'}.
\end{one-digit-list}

\item [4.] An air traffic controller is tracking, via radar, two planes
cruising at the same altitude.
At a particular time Plane \m{A} is 12\unit{miles} away, bearing {60\degrees} (measured
clockwise from north) and is traveling due north at 250\unit{mi/hr}.
Plane \m{B} is 10\unit{miles} out, bearing {37\degrees} and is traveling due east at
550\unit{mi/hr}.
\begin{one-digit-list}
\item [a.] Draw a sketch of the position and velocity vectors of each plane
with respect to the controller's position.
\item [b.] Express the position vector of each plane, and the position of plane B
relative to plane \m{A}, as functions of time.
\item [c.] Will the two planes collide if each plane holds its course?
If so, how long after the time of the above initial conditions will the
collision take place?
Would you recommend this calculation be made by hand calculator, slide rule,
or computer?
\item [d.] What is the velocity of plane \m{B} relative to plane \m{A}?
\end{one-digit-list}

\item [5.] In 1821 H.\,M.\,S.\,Clorinda, a British frigate, attempted to
capture the Estrella del Sur, a Spanish schooner.
At a particular point of the chase the schooner, well out of range, changed
course to cross Clorinda's bow and head for San Juan, Puerto Rico on a NW
course (N\,{45\degrees}\,W) at a speed of 8\unit{knots}.
Clorinda, heading N\,{11.25\degrees}\,W at 6\unit{knots}, saw the schooner 3\unit{points}
off the starboard bow (i.e. at a bearing of {33.75\degrees} measured
clockwise from dead ahead) at a range of 0.25\unit{nautical miles}.
Maximum cannon shot range was 2\unit{cable lengths} (0.20\unit{nautical miles}).
\begin{one-digit-list}
\item [a.] Sketch the situation at the time of the schooner's course change.
Include the velocities of each ship and the position of the Estrella
del Sur relative to Clorinda.
\item [b.] Calculate the velocity of the schooner relative to the frigate.
\item [c.] Express the position of the schooner relative to the frigate as a
function of time.
\item [d.] Express the distance of the schooner relative to the frigate as a
function of time.
\item [e.] What is the distance of closest approach of the two ships?
Will the schooner ever be in range?
\item [  ] \begin{tabular}{r c l}
Note: nautical mile & = & \m{1.0 \times 10^1\unit{cable lengths}} \\
               knot & = & nautical mile/hr.
\end{tabular}
\end{one-digit-list}

\item [6.] You are a small-craft airplane pilot.
You have pointed your plane due west and maintained a speed of 120\unit{mi/hr}
as registered on your air-speed indicator (speed with respect to the air)
for two hours.
However, there is a wind blowing out of the north-east at
\m{4.0\times10^1\unit{mi/hr}}.
Calculate the distance you are off your desired due-west course at the
end of the two hours.

\item [7.] Relative to the origin of a laboratory reference frame \m{L}, an
object \m{A}'s position varies with time according to
%
\Eqn{}{\vect{r}_A = 5 t^2 \uvec{x} + 8 \uvec{y}\,,}
%
while an observer \m{O} sees
%
\Eqn{}{\vect{r}_A = 7 t^2 \uvec{x} - 3 t^{-1} \uvec{z}\,.}
%
\begin{one-digit-list}
\item [a.] Rewrite the above vector equations, putting in double subscripts and proper units.
Assume that all distance quantities are in meters, all time quantities in hours.
\item [b.] Find the speed (magnitude of the velocity) of \m{O} relative to \m{L} at time \m{t}.
\item [c.] Find the magnitude of the acceleration of \m{O} relative to \m{L} at time \m{t}.
\item [d.] Draw a rough sketch of the position and velocity vectors of \m{O}, as seen by \m{L}
           at \m{t = 90\unit{min}}, as they would appear looking down the \m{z}-axis toward the origin.
\end{one-digit-list}
\end{one-digit-list}

\BriefAns

\begin{one-digit-list}
\item [1.] \NullItem
\begin{one-digit-list}
\item [a.] 15\unit{mi/hr}, westward
\item [b.] 105\unit{mi/hr}, westward
\end{one-digit-list}

\item [2.] {143\degrees} from the direction of the current.
The trip will take 71\unit{seconds}.

\item [3.] \NullItem
\begin{one-digit-list}
\item [a.] \m{\vect{r}_{QO} = \left[ (6\unit{m/s\up{2}}) t^2 -
              (4\unit{m/s}) t \right] \uvec{x} +
              (-3\unit{m/s\up{3}}) t^3 \uvec{y} + (3\unit{m})\uvec{z}}
\item [  ] \m{\vect{r}_{QO'} = \left[ (6\unit{m/s\up{2}}) t^2 +
              (3\unit{m/s}) t\right] x + (-3\unit{m/s\up{3}}) t^3 \uvec{y} +
              (3\unit{m}) \uvec{z}}
\item [b.] \m{t = 0}
\item [c.] \m{\vect{v}_{O'O} = (- 7\unit{m/s}) \uvec{x}}
\item [d.] \m{r_{OO'} = 21\unit{m}}
\item [e.] \m{r_{QO} = 91.29\unit{m}}
\item [f.] \m{r_{QO'} = 102.66\unit{m}}
\item [g.] {10.46\degrees}  \help{1}
\item [h.] {10.46\degrees}  \help{2}
\item [i.] \CharacterUnframedFigure{m11gr03}
\item [j.] In this problem, \m{v_{O'O} = - 7\unit{m/s}}, so \m{v} is not zero at any
time.
\end{one-digit-list}

\item [4.] \NullItem
\begin{one-digit-list}
\item [a.] \CharacterUnframedFigure{m11gr04}
\item [b.] \m{\vect{r}_A = (10.4\unit{mi})\uvec{x} + \left[ 6\unit{mi} +
              (250\unit{mi/hr}) t\right] \uvec{y}}
\item [  ]   \m{\vect{r}_B = \left[ 6\unit{mi} + (550\unit{mi/hr}) t\right]
                \uvec{x} + (8\unit{mi}) \uvec{y}}
\item [  ]   \m{\vect{r}_{BA} = \vect{r}_B - \vect{r}_A =
                \left[ (550\unit{mi/hr}) t - 4.4\unit{mi}\right] \uvec{x} +
                \left[2\unit{mi} - (250\unit{mi/hr}) t\right] \uvec{y}}
\item [c.] The planes will collide in 29\unit{seconds}. A computer would be advisable.
\item [d.] \m{\vect{v}_{BA} = 604\unit{mi/hr}}, bearing {114\degrees}.
\end{one-digit-list}

\newpage
\item [5.] \NullItem
\begin{one-digit-list}
\item [a.] \CharacterUnframedFigure{m11gr05}
\item [b.] \m{\vect{v}_{21} = (- 4.48\unit{knots}) \uvec{x} + (- 0.228\unit{knots}) \uvec{y}}
\item [c.] \m{\vect{r}_{21} = \left[ 0.096\unit{mi} + (- 4.48\unit{knots}) t\right] \uvec{x} +
                             \left[ 0.231\unit{mi} + (- 0.228\unit{knots}) t \right] \uvec{y}}
\item [d.] \m{r_{21} =\newline
              \left( \left[0.096\unit{mi} + (- 4.48\unit{knots}) t\right]^2
              + \left[ 0.231\unit{mi} + (- 0.228\unit{knots}) t\right]^2\right)^{1/2}}
\item [e.] The minimum value of \m{r_{21}} is 0.226\unit{miles}.
           This is greater than 0.2\unit{mi}, so the schooner will never be in range.
\end{one-digit-list}

\item [6.] 56.6\unit{mi}

\item [7.] \NullItem
\begin{one-digit-list}
\item [a.] \m{\vect{r}_{AL} = (5\unit{m/hr\up{2}}) t^2 \uvec{x} + (8\unit{m}) \uvec{y}}
\item [  ] \m{\vect{r}_{AO} = (7\unit{m/hr\up{2}}) t^2 \uvec{x} + (-3\unit{m}\unit{hr})t^{-1} \uvec{z}}
\item [b.] \m{v_{OL} = \left( 16\unit{m\up{2}}\unit{hr\up{-4}} t^2 +
                          9\unit{m\up{2}}\unit{hr\up{2}} t^{-4} \right)^{1/2}}
\item [c.] \m{a_{OL} = \left( 16\unit{m\up{2}/hr\up{4}} +
                            36\unit{m\up{2}}\unit{hr\up{2}} t^{-6} \right)^{1/2}}
%\item [d.] \m{x, y} in meters: \CharacterUnframedFigure{m11gr07}
\newpage
\item [d.] \m{x, y} in meters: \CharacterUnframedFigure{m11gr07}
\end{one-digit-list}
\end{one-digit-list}

}% /Sect
