\revhist{3/16/88, pss; 4/1/91, pss; 9/9/91, pss; 5/7/92, pss; 9/29/94, pss;
         10/12/94, pss; 11/30/94, pss; 9/15/95, pss; 2/28/96, pss; 4/7/97, pss;
		 2/22/99, pss; 3/30/99, pss; 2/11/02, pss}
%
\defModTitle{\ph{Relative Linear Motion,} \ph{Frames of Reference}}
\defCtAuthor{Peter Signell and William Lane}
\defIdAuthor{Peter Signell and William C. Lane, Department of Physics}
%
\defIdItems{
    \IdVersEval{2/11/2002}{1}
    \IdHours{1}
    \begin{InputSkills}
    \item [1.]  Add and subtract vectors using their Cartesian components \prrqone{0-2}.
    \item [2.]  Given a vector described either in terms of Cartesian components or
    magnitude and direction, determine the other vector description \prrqone{0-2}.
    \item [3.]  Given an object's position as a function of time, find its velocity and
    acceleration as functions of time \prrqone{0-8}.
    \item [4.]  Detect errors in symbolic equations by checking dimensions \prrqone{0-8}.
    \end{InputSkills}
    %
    \begin{KnowledgeSkills}
    \item [K1.] Vocabulary: frame of reference, observer.
    \end{KnowledgeSkills}
    %
    \begin{ProblemSolvingSkills}
    \item [S1.] Given the position vectors of both an object and an observer as seen by
    a second observer, find the position, velocity and acceleration vectors of
    the object as seen by the first observer.
    Use suitable notation for labelling observers and observed.
    Sketch any of the vectors with respect to any frame of reference, as
    requested.
    \item [S2.] Given a kinematical equation containing both symbols and numbers plus a
    word description of the units involved, properly insert the units into the
    equation and carry out any appropriate units algebra.
    \item [S3.] In any given kinematical problem, determine whether \m{a = dv/dt}
    is valid.
    \end{ProblemSolvingSkills}
}