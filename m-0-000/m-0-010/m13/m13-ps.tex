\revhist{5/13/85, mpm; 10/13/91, pss; 9/29/94, pss}

\Sect{}{}{\SectType{ProblemSet}}{

\noindent Problem~4 also occurs in this module's \textit{Model Exam}.

\begin{one-digit-list}
\item [1.] A neutral pion, \mpi\up{0}, has a lifetime of
           \m{8.4 \times 10^{-17}\unit{sec}} in its own rest frame before decaying
           into two photons.
           However in a laboratory experiment, where the pions are observed to
           be moving, their lifetime is measured to be
           \m{4.2 \times 10^{-16}\unit{sec}}.
           Assuming this discrepancy to be due to time dilation, calculate the
           speed of the pions.

\item [2.] \Quote{There was a young fencer named Fisk, whose style was incredibly
           brisk.
           So fast was his action, that Lorentz space contraction foreshortened
           his foil to a disk.}
           Assuming his 1.00\unit{m} long fencing foil only contracted to 10.0\unit{cm}
           (roughly the size of a good steak knife), how fast was Fisk lunging?

\item [3.] At the moment of the birth of a set of twins, one child is placed in
           a spaceship that rapidly accelerates to 0.866\unit{c} (0.866 times the
           speed of light), and travels to Alpha Centauri, 4.33\unit{light years}
           away (measured in the rest frame of the earth.)
\begin{one-digit-list}
\item [a.] Calculate the age of each twin when the traveler reaches Alpha
           Centauri, as measured by the twin on earth.
\item [b.] Calculate the distance from Alpha Centauri to earth, as measured by
           the traveling twin.
\item [c.] How old is the traveling twin, as measured by himself, when he
           reaches Alpha Centauri?
\item [d.] How old does the traveling twin perceive the earthbound twin to be?
\end{one-digit-list}

\item [4.] Suppose you have a twin who is an astronaut.
           The twin travels at speed 0.9999\unit{c} to the vicinity of a star which
           is 60\unit{light years} away from earth (one \unit{light year} is the distance
           light travels in one year).
\begin{one-digit-list}
\item [a.] Find your age and your twin's age at the time you observe that your
           twin reaches the star.
\item [b.] Show that the length, time and velocity observed by your twin check.
\end{one-digit-list}

\end{one-digit-list}

\BriefAnsNewPage

\begin{one-digit-list}
\item [1.] \m{v = 0.98\unit{c}}

\item [2.] \m{v = 0.995\unit{c}}

\item [3.] \NullItem
\begin{one-digit-list}
\item [a.] Age of stationary twin: 5\unit{yrs}.; Age of traveling twin: 2.5\unit{yrs}.
\item [b.] 2.165\unit{light years}
\item [c.] 2.5\unit{yrs}. 
\item [d.] 1.25\unit{yrs}. 
\end{one-digit-list}

\item [4.] \NullItem
\begin{one-digit-list}
\item [a.] Present age \m{+ 60.006\unit{yr}}., present age \m{+ 0.85\unit{yr}}.
\item [  ] Twin observes distance of 0.848507\unit{light years}, time of 0.84859\unit{yr}.,
           velocity of \m{-0.9999\unit{c}}.
\item [b.] Then \m{(0.848507/0.84859)\unit{c} = 0.9999\unit{c}}: check.
\end{one-digit-list}
\end{one-digit-list}

}% /Sect
