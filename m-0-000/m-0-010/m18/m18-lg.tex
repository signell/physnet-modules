\revhist{4/12/91, pss; 10/3/94, pss; 8/25/96, pss}

\Sect{}{}{\SectType{LocalGuide}}{

\LgSect{A DEMONSTRATION}{
%\begin{center}\textbf{\Large{A DEMONSTRATION}}\end{center}
%
You are to predict the direction of a particular Coriolis force acting on the
CBI Consultant, explaining in detail (to the Consultant) how you manipulated
the cross products in order to make the prediction.
You are then to do a demonstration on the Consultant to prove that you are
correct.

Practice this demonstration on yourself or a friend before trying it on the
Consultant: you get only one chance!

Demonstration: You will have the Consultant note the sideways deflection of his
or her {\bf fist} while pulling it inward or outward during rotation.
You are to explain the direction of the deflection to the Consultant, pointing
out the direction of each vector in the Coriolis force and how the \Quote{cross
products} produce the observed deflection.

You will ask the Consultant to sit in the rotating chair in the corner of the
CBI Consulting Room.
You will ask the Consultant to keep his or her head on the axis of rotation at
all times.

Then you will describe what is going to take place and you will tell the
Consultant the directions of the various vectors involved and what direction
that predicts for the Coriolis force.

You will ask the Consultant to fully extend his or her arm and you will
then rotate the chair slowly in the direction you specified in advance to the
Consultant.
Have the Consultant do several of: arm in, arm out, CW rotation, CCW rotation.
In each case you predict the direction of the Coriolis force in advance.

When finished, have the Consultant give you a signed receipt describing how
well you did, on a scale of 0-4, which you can attach to the written
part of the exam.
}% /LgSect
%
\LgSect{READINGS}{
%
\begin{center}
(See next page for readings-skills cross-reference)
\end{center}
%
\begin{itemize}
\item []  \NullItem
\begin{itemize}
\item [\bf MOD:]  This module.
\item [\bf AF:]  M.\,Alonso and E.\,Finn, {\em Physics}, Addison-Wesley
(1970).
Section~6.5 includes a beautiful Tiros satellite photograph showing
wind-driven clouds spiraling in toward a low-pressure center, plus Foucault
pendulum diagrams.
\item [\bf HRI:] D.\,Halliday and R.\,Resnick, {\em Physics}, Part I,
Wiley (1977).
Section~6-4 includes a discussion of \Quote{Forces and Pseudo-Forces,} not
including the Coriolis force.

\item [\bf BO:] V. Barger and M. Olsson, \textit{Classical Mechanics},
McGraw-Hill Book Co., N.Y. (1973).
Sections~6-1 to 6-4 include a picture and derivations for the Foucault
pendulum and a short discussion of trade winds and cyclonic effects.

\item [\bf AM:] J. Ames and F. Murnahan, \textit{Theoretical Mechanics}, Dover
Publ., Inc., N.Y. (1957).
Section~67 includes the full theory of the Foucault pendulum.

\item [\bf KKR:] C. Kittel, \textit{Mechanics (Berkeley Physics Course -
Vol. 1)}, W. Knight, and M. Ruderman, McGraw-Hill Book Co., N.Y. (1965).
Pages~84-91 contain a useful discussion of the definition of vector angular
velocity.
The \Quote{Coriolis acceleration} is given with the opposite sign to that
generally used.
The derivation does not use vector notation and it is noted that this
method produces \Quote{some tedious algebra.}

\item [\bf JM:] J. Marion, \textit{Classical Dynamics}, Academic Press, N.Y.
(1970).
Chapter~11 has a footnote at the bottom of page~346 that is priceless and
well worth a trip to the library.
Figure~12-4, showing cyclonic deflection, is quite clear.
\end{itemize}
\end{itemize}
}% /LgSect
%
\LgSect{READINGS-SKILLS CROSS REFERENCE}{
%
\begin{center}\begin{tabular}{r r l}
\multicolumn{1}{c}{\underline{Skill}} & \underline{Ref.} & \underline{Items} \smallskip \\
K1 (derive forces)  & MOD & Sect.\,2 \\
                    & JM  & Ch.\,11 \\
\\
K2 (cyclonic rot.)  & MOD & Sect.\,4 \\
                    & JM  & Ch.\,11 \\
                    & BO  & Ch.\,6 \\
\\
K3 (prevail. winds) & MOD & Sect.\,5 \\
                    & BO  & Ch.\,6 \\
\\
K4 (Foucault pend.) & MOD & Sect.\,3 \\
                    & AF \\
                    & BO  & Ch.\,6 \\
\\
K5 (Demo., Cor. f.) & MOD & ME \\
\end{tabular}\end{center}
}% /LgSect
}% /Sect