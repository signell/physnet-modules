\revhist{7/16/85; mpm; 10/4/89, pss; 8/5/91, pss; 9/29/94, pss; 11/13/97, pss; 7/25/99, abs;
         11/13/2000, pss; 12/15/2000, pss; 3/29/01, kag}
%
\Sect{1}{Introduction}{\SectType{TextMultiPara}}{
%
\pcap{1}{a}{The Laws of Motion}
This module provides a detailed introduction to Newton's three laws of
motion\Index{Newton's laws of motion}\Index{laws of motion, Newton's}.
Newton's laws are rules that tell how an object's velocity and hence its
energy and momentum can be changed.
They tell under what circumstances an object's velocity is constant, when it
will change, and what happens to whatever is causing it to change.

\pcap{1}{b}{How the Laws are Presented}
Each law is stated and then illustrated with a simple idealized experiment.
Such experiments are included to give you an intuitive feeling for the
meaning of the laws.
The use of each law is illustrated with an example.
Further applications of the laws are given in the last section of the text.

\pcap{1}{c}{Why We Believe the Laws of Motion}
It is important to realize that the laws of motion are not \Quote{proved} in the
usual sense of the word and that they are most certainly not proved by the
examples given in this module.
We believe in the laws of motion--that is, we are willing to keep using
them---because of the fact that all of the laws and definitions taken
together make up a system that enables one to predict correctly what will
happen in an experiment.
Literally millions of such experiments have been done, both very simple ones
and exceedingly complex ones, and within their range of applicability, these
laws have always given the answers produced by the experiments.
}% /Sect
%
\Sect{2}{Newton's First Law of Motion}{\SectType{TextMultiPara}}{
%
\pcap{2}{a}{Introduction}
It is common knowledge that a sliding object tends to slow down and
eventually stop.
The ancient Greeks felt this was easy to understand since they believed that
the natural state of most objects was at rest; and in fact, they devoted
considerable effort in attempting to explain how an object could keep moving
when it was not being pushed.
Since Newton's time, scientists have realized that the thing to explain is
the cause of the acceleration, not the cause of the motion.
Newton's first law of motion
\Index{Newton's first law of motion}\Index{first law of motion, Newton's}emphasizes
this change in attitude by stating
that there must be a cause for an object's acceleration, while no cause is
needed for constant velocity.

\pcap{2}{b}{Newton's First Law of Motion}
Newton's first law \Index{Newton's first law of motion}is:
%
\textbox{A completely isolated object, that is, one subject to no external
influences, has a constant vector velocity (\m{\vect{a} = 0}).}
%
Thus an isolated object has no acceleration; if it is moving, it simply
keeps on moving.
No explanation is needed for motion; rather, an explanation is needed for
acceleration, for a \textit{change} in velocity.
Of course, being at rest is a special case of moving with a constant
velocity -- a velocity of zero.
Here is another way of stating the first law of motion\Index{first law of motion, Newton's}:
If an object is accelerating, then it is being acted on by an external influence.
The following two paragraphs show that this law is not really contrary to
common experience.

\pcap{2}{c}{Friction Causes Acceleration}
A series of simple experiments shows that friction \Index{friction}is one reason why many
objects slow down.
Consider a block sliding across a table.
It is certainly true that unless you keep pushing it, it slows down and
eventually stops.
However, if you polish the surfaces, you will find that the block slides
farther before it comes to a stop.
If you oil or grease the surface lightly, you find that the block slides
even farther before it comes to rest.
If you use a very smooth surface and an object that \Quote{floats} on a cushion
of air (such as a hovercraft), so that almost all friction is
eliminated, you can get the object to slide incredible distances before it
comes to rest.
If the object is taken into space where there is essentially no friction
and is given an initial velocity, as nearly as we can tell, it simply does
not slow down, ever.
Thus friction is one cause of acceleration (actually, deceleration).

\pcap{2}{d}{Other Causes of Accelerations}
There are cases where an object undergoes an acceleration that is not caused
by friction; however, in these cases it is possible to devise experiments to
show that the acceleration is due to some other external influence acting on
the object.
One common example concerns the fact that objects fall; that is, they
accelerate toward the earth.
We interpret this as being due to the gravitational attraction
\Index{acceleration| due to gravity}that the earth exerts on the object.
%
\Footnote{1}{See \Quote{The Law of Universal Gravitation} (MISN-0-101).}
%
Today we do experiments in space which show that this attraction gets weaker
as an object gets farther from earth.
From this we conclude that, far from the earth, an object would not fall
toward the earth.
(Newton was able to deduce this from considerations of data on the solar
system.)
Thus we conclude that objects fall because of the influence of the earth.
In a similar manner, it is possible to show that in every case in which an
object undergoes an acceleration, the acceleration is due to some external
influence acting on the object.
This is just the first law of motion.

\pcap{2}{e}{The Use of the First Law of Motion}
It is often stated that Newton's first law is just a special case of his
second law, but this is not true.
All kinematic and dynamic measurements must be made with respect to some
system of reference.
Only in certain systems of reference \Index{system of reference}(nonaccelerated ones) will our laws of
particle dynamics \Index{particle dynamics}be found to hold.
%
\Footnote{2}{See \Quote{Centripetal and '\m{g}' Forces in Circular Motion}
(MISN-0-17), \Quote{Classical Mechanics in Rotating Frames of Reference}
(MISN-0-18), and \Quote{The Equivalence Principle} (MISN-0-110).}
%
Newton's first law gives a test to apply to a measuring system to verify
that it is a suitable system in which to use our laws of particle dynamics\Index{dynamics| particle}.
Observe an isolated object from your own reference system\Index{reference system}; if the object has
a constant vector velocity, then you will find that all of particle dynamics
holds in your reference system.
}% /Sect
%
\Sect{3}{Newton's Second Law of Motion}{\SectType{TextMultiPara}}{
%
\pcap{3}{a}{The Second Law of Motion}
Newton's first law of motion states that an isolated object has no
acceleration.
His second law of motion \Index{Newton's second law of motion}gives
the relationship between the acceleration
of an object and the forces acting on it.
The second law of motion \Index{second law of motion, Newton's}is:
%
\textbox{The (vector) acceleration \Index{acceleration| related to force}of
an object is proportional to the
(vector) force \Index{force| related to acceleration}acting on it.}
%
As an equation this is:
%
\Eqn{1}{\vect{F} = m\,\vect{a}\,,}
%
where \m{m} is the mass of the object.
Thus, in the case of a single force applied to an object, if you know any
two of the three quantities \vect{F}, \m{m}, or \vect{a} you can immediately
solve \Eqnref{1} for the third.
Here is an illustrative exercise:

\tryit \textit{Exercise 1}: The wind exerts a force of 100\unit{newtons} in a
northward direction on a sailboat having a mass of 250\unit{kilograms}.
What is the acceleration of the boat?
(Don't worry about the units of force and mass at this time; they will be
discussed in Sect.\,5.)
The acceleration will come out in \unit{m/s\up{2}} for these units. \help{1}
%
\Footnote{3}{The answers and detailed solutions to the exercises are given in
this module's \textit{Special Assistance Supplement} if you need help.}
%

\pcap{3}{b}{Mass}
The mass \Index{mass}of an object is an intrinsic property of the object.
It is a scalar and is always positive.
The usual way to find the mass of an object is to weigh \Index{weight}it and calculate the
mass from:
%
\TwoEqns{2}{\text{Weight} & = \text{ mass }\times\text{ constant}}
           {w & = m\,g\,.}
%
The value of \m{g} for \Eqnref{2} is given in the table below for several
combinations of units.
Mass and weight are discussed at length elsewhere.
%
\Footnote{4}{See \Quote{Mass and Weight} (MISN-0-64) which provides a careful
discussion of the definitions of weight and mass and of the differences
between the two.}
%
\begin{center}\begin{tabular}{| l | l | l |}\hline
Units of Weight              &  Units of Mass                   & Value of \m{g}\Index{g, value of} \\ \hline
\unit{newtons}\Index{newton} & \unit{kilograms}\Index{kilogram} & 9.81\unit{newtons/kilogram}       \\
\unit{pounds}\Index{pound}   &   \unit{kilograms}               & 2.21\unit{pounds/kilogram}        \\
\unit{pounds}                &     \unit{slugs}\Index{slug}     & 32.2\unit{pounds/slug}            \\
\unit{dynes}\Index{dyne}     &     \unit{grams}                 & 981\unit{dynes/gram}              \\ \hline
\end{tabular}\end{center}

\pcap{3}{c}{Experimental Verification of Newton's Second Law}
A number of simple experiments demonstrate that Newton's second law is
consistent with both your intuitive notions and with nature.
You know that if you push an object its velocity changes.
By doing a simple experiment involving a block sliding on a more or less
frictionless surface and using a spring scale, you can establish that the
block accelerates in the direction of the force and that the acceleration is
proportional to the force (see \Figref{1}).
You can easily devise a number of similar experiments involving pushes,
pulls, and simple measuring devices.
In all cases, you will find that the acceleration is proportional to the
magnitude of the force as determined by the measuring device and that this
magnitude corresponds roughly to your own physiological impression of
\Quote{how hard} you have pushed or pulled the object.
This is just what the second law says.

\CaptionedFullFramedFigure{1}{(a) An experimental arrangement for
measuring the relationship between force and acceleration.
(b) A typical result---the second law.}{m14gr01}

\pcap{3}{d}{A Statement of the Superposition Principle}
If you do an experiment to investigate what happens when two or more forces
are simultaneously applied to the same object, you will find that the
acceleration of the object is the same as that caused by a single force
equal to the vector sum of the individual forces being applied to the object.
The experimental result is often called the \Quote{superposition principle} for
forces\Index{superposition principle, for forces}\Index{forces| superposition principle for}.
A somewhat formal statement of it is:
%
\textbox{Two or more forces applied simultaneously to a particle have the
same effect as a single force equal to the vector sum of the individual
forces.}
%
An example of this is illustrated in \Figref{2}.

\pcap{3}{e}{A Restatement of the Second Law of Motion}
Using the superposition principle, the second law of
motion\Index{Newton's second law of motion}\Index{second law of motion, Newton's} can be stated
as:
%
\textbox{The acceleration of an object is proportional to the total vector
force acting upon it.}
%
As an equation:
%
\Eqn{3}{\sum\,\vect{F} = m\,\vect{a}\,.}
%
The summation \m{\sum},\Index{summation sign} pronounced
\Quote{sigma\Index{sigma},} reminds you to take the vector
sum of all the forces applied to the object.
Here is an exercise that will give you a little practice in using this law:

\enlargethispage{1pc}
\tryit \textit{Exercise 2}. Forces of 25unit{newtons} north, 15\unit{newtons} east, and
40.0\unit{newtons} south are applied to a 5.0\unit{kg} object.
Show that the object has an acceleration of 4.24\unit{m/s\up{2}} in a direction {45\degrees}
south of east. \help{2}\,
%
\Footnote{5}{Additional help on the second law is given in Sect.\,1a of this module's
\textit{Special Assistance Supplement}.}

\CaptionedFullFramedFigure{2}{An example of the superposition of forces.
The acceleration of the object acted upon by the three forces shown on the
left is the same as the acceleration caused by \vect{R}, the vector sum of the
three forces.}{m14gr02}
}% /Sect
%
\Sect{4}{Newton's Third Law of Motion}{\SectType{TextMultiPara}}{
%
\pcap{4}{a}{The Third Law of Motion}
When two objects interact, such as when two billiard balls collide, each one
of them exerts a force on the other.
Newton's third law of motion
\Index{Newton's third law of motion}\Index{third law of motion, Newton's}
relates the force the first object exerts on
the second to the force the second exerts on the first.
The law is:
%
\textbox{To every action\Index{action} there is always opposed an equal reaction\Index{reaction}: or, the
mutual actions of two bodies upon each other are always equal and in
opposite directions.}
%
Putting this in other words:
%
\textbox{If object \m{A} exerts a force on object \m{B}, then object \m{B}
exerts an equal but oppositely directed force on object \m{A}.
Further, these forces both lie along the line joining the two bodies.}
%
This law is sometimes casually quoted as \Quote{action equals reaction.}

\CaptionedFullFramedFigure{3}{A demonstration of the third law of motion.}{m14gr03}

\pcap{4}{b}{A Demonstration of the Third Law}
\Figref{3} shows an experimental arrangement that is frequently used to
demonstrate the third law of motion.
Two identical blocks (or carts) are tied together by a string with a light
spring compressed between them.
The surfaces are very smooth, so the blocks slide very freely.
The blocks are initially at rest, so according to the first law they remain
at rest as long as the string is intact.
Now, if the string is very carefully broken - it is traditional to burn it
with a laser beam---the objects spring apart.
BIG DEAL!!!
However, careful measurements reveal two surprising (?) facts for two
identical objects:
%
\textbox{(a) The objects always move apart in the precisely opposite
directions.}
\textbox{(b) The magnitudes of the two accelerations are always precisely the
same.}
%
Hence, when the acceleration is over, the velocities of the two identical
objects are equal but opposite: \m{\vect{v}_1 = - \vect{v}_2}.
This same result occurs for big objects and for small ones, for cases when
the spring is lightly compressed and when it is strongly compressed, and for
cases where the spring is set up to pull as well as to push, as long as the
two objects are truly identical and are isolated from all other
influences.
%
\Footnote{6}{The case where the two objects are not identical is treated in
\Quote{Momentum: Conservation and Transfer} (MISN-0-15).}
%
Since our two objects have the same mass and equal but opposite
accelerations, then, according to the second law, the force the first block
exerts on the second (via the spring) must be equal and opposite to the
force the second block exerts on the first.
This is the third law of motion.

\pcap{4}{c}{A Comment About the Third Law}
It is very important to realize that although the third law says that forces
always occur in pairs, each member of a pair always acts on a different
object from that acted on by the other member of the pair.
(If they were to act on the same body, then it would be impossible ever to
accelerate a body, since the resultant force would always be zero!)

\pcap{4}{d}{Two Simple Examples of the Third Law}
As a simple example, imagine that you are hitting a tennis ball.
Your racket exerts a force on the ball, causing it to accelerate.
At the same time, the ball exerts an equal but opposite force on the racket,
causing it to accelerate in the opposite direction (slowing it down a bit).
Thus the action and reaction act on different bodies and cause each body to
be accelerated (but oppositely).

When you throw an object, such as a ball, you must exert a force on it to
accelerate it.
In return, the ball exerts an equal force on you in the opposite direction.
As a result you \Quote{recoil}\Index{recoil} in the opposite direction from which you threw the
ball.
This recoil is particularly noticeable when you throw a very heavy object,
or when you are standing on a slippery surface--such as a sheet of ice.
Also, see \Figref{4}.

\CaptionedFullFramedFigure{4}{A novel way to propel a skateboard.}{m14gr04}
}% /Sect
%
\Sect{5}{Systems of Units}{\SectType{TextMultiPara}}{
%
\pcap{5}{a}{Introduction}
In order to use the laws discussed in this module, it is necessary to have
all relevant quantities, (distance, time, mass and force) in a consistent
set of units.\Index{units| of force}
The set of consistent units that is in use throughout most of the world is
the SI\Index{force| SI units of} (\textit{International System}) of the ISO (\textit{International
Organization for Standardization}).
The U.S. Government has put the SI units on the Web at the NIST (National Institute for
Standards and Technology) site for \textit{Constants,
Units, and Uncertainty}: \print{\PrintUrl{http://physics.nist.gov/cuu/}}\web{\url{http://physics.nist.gov/cuu/}}.
The so-called \textit{English system}, based on pounds, feet, etc., is in partial use in
the U.S. and a few other countries.  An older \Quote{cgs} system, based on the centimeter,
gram, and second, is still used occasionally.

\pcap{5}{b}{SI Units}
Using the second law of motion, you can see that the dimensions of force are
%
\Eqn{}{\unit{mass} \; \times \; \unit{length} \; \times \; \unit{time\up{-2}}\,.}
%
In the SI system of units\Index{force| SI units of}\Index{units| SI system of},
length is measured in meters, mass in \unit{kilograms}, and time in \unit{seconds}.
Thus force has units of
%
\Eqn{}{\unit{kilogram} \; \unit{meter} \; \unit{second\up{-2}}\,.}
%
This quantity occurs so often that it has been given the name \Quote{newton.}\Index{newton}
Thus:
%
\TwoEqns{}{\unit{newton} & = \unit{kilogram\;meter\;second\up{-2}}}
          {              & = \unit{kg\,m\,s\up{-2}}\,.}
%
The abbreviation for both \unit{newton} and \unit{newtons} is \unit{N}.
To get a rough idea of how large one \unit{newton} of force is, just remember that
a quart of milk has a weight of roughly ten\unit{newtons}.

\pcap{5}{c}{The CGS System}
In the old cgs\Index{cgs system of units} (centimeter-gram-second) system,\Index{force| cgs units of} length is measured in
\unit{centimeters}, mass in \unit{grams}, and time in \unit{seconds}.\Index{units| cgs system of}
The unit of force is the \unit{dyne}, defined as:
%
\TwoEqns{}{\unit{dyne} & = \unit{gram\,centimeter\,second}\up{-2}}
          {            & = 1\unit{g\,cm\,s\up{-2}}\,.}
%
A dyne\Index{dyne} is a very small unit of force.
The weight of a nickel is roughly 5,000\unit{dynes}, so a one-dyne force
is a very, very weak force.

\pcap{5}{d}{The English System}
In the English system of units\Index{English system of units}, mass is measured in slugs, length in feet,
and time in seconds.\Index{force| English units of}\Index{units| English system of}
The unit of force is the \unit{pound}\Index{pound}and it is defined as:
%
\Eqn{}{\unit{pound} = \unit{slug\,foot\,second\up{-2}}\,.}
%
You are already aware of how strong a one-pound force is, but you are
probably unacquainted with the slug.\Index{slug}
An object having a \textit{mass} of one slug has a \textit{weight} of roughly
32\unit{pounds}.

\pcap{5}{e}{A Table of Names of Units}
The table below gives you a brief summary of these units and their
abbreviations (\Quote{abb}).\Index{table| of units}\Index{units| table of}
%
\begin{center}
\begin{tabular}{|l | l | c | l | c | l | c|} \hline
          & \multicolumn{2}{c|}{SI units} & \multicolumn{2}{c|}{cgs system} &
\multicolumn{2}{c|}{English system} \\ \cline{2-7}
 Quantity & Unit     & abb & Unit       & abb & Unit   & abb  \\ \hline
 \unit{length}   & \unit{meter}    &  m  & \unit{centimeter} & \unit{cm}  & \unit{foot}   &   \unit{f}  \\
 \unit{time}     & \unit{second}   &  s  & \unit{second}     & \unit{s}   & \unit{second} &   \unit{s}  \\
 \unit{mass}     & \unit{kilogram} & kg  & \unit{gram}       & \unit{gm*} & \unit{slug}   & (none) \\
 \unit{force}    & \unit{newton}   & N   & \unit{dyne}       & \unit{dyn} & \unit{pound}  & \unit{lb}   \\ \hline
\end{tabular}\end{center}
* Note that in SI units the gram is a derived quantity and its abbreviation is \Quote{g} (use
context to tell when \Quote{g} means \Quote{gram} and when it means the gravitational constant.)

%
You can find the definitions of these systems and many useful conversion
factors in any standard reference on physical quantities.
%
\Footnote{7}{See the {\em CRC Handbook of Physics and Chemistry},
Chemical Rubber Publishing Co., Cleveland, Ohio.}
%
}% /Sect
%
\Sect{6}{More Applications}{\SectType{TextMultiPara}}{
%
\pcap{6}{a}{Introduction}
This section contains several more applications of the laws of motion to
practical situations.
Additional applications will be found in this module's Problem Supplement.
Applications of the laws of motion to more complex situations and to
situations involving two or more objects are found
elsewhere.
%
\Footnote{8}{To see the full power of the laws of motion you must apply them in
practical situations.
Many interesting situations are treated in \Quote{Applications of Newton's Second
Law, Frictional Forces} (MISN-0-16).
Why is it so hard to put a large space station into orbit around the earth?
This problem is treated in \Quote{Mass Changing With Time: Rockets and Other
Examples} (MISN-0-19).}
%

\pcap{6}{b}{An Object at Rest}
The net force\Index{net force} acting on an object at rest\Index{object| at rest} must be zero.
As an example, consider a motionless ball sitting on a table.
Since we have said that the ball remains at rest\Index{rest, object at}, it has no acceleration.
According to the second law, this means that the total (net) force acting
on it is zero.
This does not mean that there are no forces acting on the ball; it does
mean that the vector sum of all the forces acting on it is zero.
In fact, for the ball resting on a table we know there are at least two
forces acting on it: its weight is acting downward and the table is exerting
an upward force on the ball (these forces are shown in \Figref{5}).
If these are the only two forces acting on the ball, we have, successively:
%
\TwoEqns{}{\vect{F}_t + \vect{F}_w & = 0\,,}
          {\vect{F}_t             & = -\,\vect{F}_w\,.}
%
\textbox{In general, for any object at rest, you immediately know
that the vector sum of all the forces acting on it is zero.}
%
In any specific case, you will have to work out how many and what forces
are acting on an object of interest that is at rest.
Then knowledge of all but one force will enable you to calculate that last
force.

\CaptionedFullFramedFigure{5}{The forces acting on a ball at rest.}{m14gr05}

\pcap{6}{c}{An Object Moving at a Constant Velocity}
When an object is moving at a constant (vector) velocity,\Index{object| moving at constant velocity} the net force
acting on it must be zero, just as for an object at rest.
An airplane moving at a constant velocity has no acceleration, hence we
must get zero for the vector sum of all the forces acting on the plane: the
weight of the plane, plus the lift due to the wings, plus the push due to
the engine, plus the air drag--see \Figref{6}.
In the notation defined in \Figref{6}, this result is:
%
\Eqn{}{\vect{F}_D + \vect{F}_P + \vect{F}_L + \vect{F}_W = 0\,.}
%
\CaptionedFullFramedFigure{6}{The forces acting on an airplane.}%
{m14gr06}
%
Since being at rest is just a special case of moving at a constant
velocity---here with a constant velocity of zero, it should be no surprise
that you know the same amount about the forces acting on the object as you
do in the ball-at-rest situation, except that in the case of the airplane
the forces may appear to be more readily identifiable.

\pcap{6}{d}{\Quote{Person in Elevator} Problem}
A very common type of problem involves an object (a person, a light bulb, a
monkey, etc.) that is hung by a string from the ceiling of an elevator\Index{elevator problems}
(\Figref{7}).
The problem is to calculate the tension in the string and find how it depends
on the acceleration of the elevator.

\CaptionedFullFramedFigure{7}{A hammer hung from an elevator ceiling.}{m14gr07}

\noindent\textit{A Qualitative Discussion:}\newline
This problem can be solved by a straightforward application of the second
law of motion.
There are only two forces acting on the object: its weight acting downward
and the upward pull of the string; these forces are shown in \Figref{8}.

The object accelerates with the elevator.
Thus if the elevator accelerates upward, so does the object.
But if the object is accelerating upward, then according to the second law
the net force on the object must be upward.
This means that the upward pull must be greater than the downward weight.\Index{weight, in accelerating elevator}
%
\Eqn{}{F_p > F_w; \qquad \text{ accelerating upward}\,.}
%
If the elevator is not accelerating, neither is the object and there is no
net force on the object.
Thus the pull must be equal to the weight.
%
\Eqn{}{F_p = F_w; \qquad \text{ no acceleration}\,.}
%
If the elevator is accelerating downward,\Index{weight, in accelerating elevator} then so is the object.
This means that the net force on the object must be downward, so the weight
must be greater than the upward pull.
%
\Eqn{}{F_p < F_w; \qquad \text{ accelerating downward}\,.}
%
(The problem becomes more complicated if the elevator accelerates downward
faster than the acceleration of free fall, 9.8\unit{m/\m{s^2}}, but we will assume that
this does not happen here.

\CaptionedFullFramedFigure{8}{Forces acting on an object hung in an elevator.}{m14gr08}

\noindent\textit{The Qualitative Solution:}\newline
The expression for the pull of the string can be found by picking a
coordinate system and using the second law of motion.
We pick a coordinate system with the x axis downward as shown in \Figref{8}.
Thus a downward acceleration is positive and an upward acceleration is
negative.
The second law of motion is:
%
\Eqn{}{\sum\,\vect{F} = m\,\vect{a}\,.}
%
The \m{x} component of the weight is \m{+F_w} while the \m{x} component of the pull
is \m{-F_p} (it is upward).
Hence the \m{x} component equation is:
%
\Eqn{}{F_w - F_p = m\,a.}
%
Solving for F\m{_p} we obtain.
%
\Eqn{}{F_p = F_w - m\,a; \qquad (\uvec{x}\text{ taken downward})\,.}
%
Check over the sign conventions.
Then see if you agree that, if the positive \m{x} direction were upward, you
would obtain:
%
\Eqn{}{F_p = F_w + m\,a; \qquad (\uvec{x}\text{ taken upward})\,.}
%
\noindent\textit{Variations on This Problem:}\newline
In one variation on the problem, the string is replaced by a spring scale,
which then reads the apparent \Quote{weight} of the object as the elevator
accelerates.
If the acceleration of the elevator happens to be 9.8\unit{m/s\up{2}} downward, in
\Quote{free fall,} then the object appears to be weightless.
In other variations, the object is placed on the floor; this is essentially
the same problem, only the pull of the string is replaced by the upward push
of the floor, etc.

\pcap{6}{e}{The Horse}
The third law of motion says that if you push on the wall, the wall pushes
on you with an equal push in the opposite direction.
Thus there are two equal and opposite forces here, but only one of them is
on you.
Be sure you understand this point.
Discuss it with others if you are unsure about it.
Any confusion here will surely result in incorrect problem solutions later.
Try your understanding on this problem:

There is a famous tale often used to illustrate the third law.
There was an iceman's horse who read Newton's writings (in Latin, yet) while
his master was making deliveries.
One day, when the iceman returned, the horse refused to try to move since he
knew it would do no good.
He reasoned as follows:
%
\begin{itemize}
\item [  ] \Quote{The second law says
%
\Eqn{}{\sum\,\vect{F} = m\,\vect{a}\,.}
%
The third law says that for every action there is an equal and opposite
reaction.
Thus if I pull on the wagon, the wagon pulls equally hard on me in the
opposite direction.
Hence the vector sum of these two forces is zero and there is no
acceleration.
Hence I cannot get the wagon to start moving.}
\end{itemize}
%
The horse was adamant; no amount of beating would make it go.
Finally Newton was called in and he quickly persuaded the horse to move on,
pulling its wagon.
How?
Can you convince the horse?
\help{10}
}% /Sect
%
\Sect{}{Acknowledgments}{\SectType{Acknowledgments}}{\NsfAcknowledgment}% /Sect

