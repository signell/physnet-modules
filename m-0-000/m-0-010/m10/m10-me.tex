\revhist{9/10/89, pss; 2/22/91, pss; 2/24/93, pss; 5/9/94, pss; 6/4/98, pss}

\Sect{}{}{\SectType{ModelExam}}{

\noindent
\textbf{Important Note:} Make each problem have these three separate parts:
\begin{itemize}
\item [a.] draw a one-body diagram with the forces \Quote{as given};
\item [b.] draw a diagram similar to the one in part (a), but with the forces now
represented only by their components, where one axis is along the direction of
acceleration and the axes are labeled;
\item [c.] write Newton's-second-law equations for the force
           components and solve for the components of the unknown force.
           Convert the answer components to magnitude and angle-to-the-acceleration direction.
\end{itemize}

\begin{one-digit-list}
\item [1.] \ItemFigure{An airplane climbs upward in a direction {53\degrees}
above the horizontal as shown in the figure.
At the particular instant illustrated, the acceleration of the plane is
24\unit{ft/sec\up{2}} and is parallel to the direction of the velocity.
Determine the contact force of the plane on the 160\unit{pound}
pilot.}{m10gr23}

\item [2.] \ItemFigure{A 16\unit{ton} truck has a deceleration of 8.0\unit{ft/s\up{2}}
as it coasts up a flat {14.5\degrees} grade.
Determine the force of the road on the truck.}{m10gr24}
\end{one-digit-list}

\BriefAns

\begin{one-digit-list}
\item [1.] See this module's \textit{Problem Supplement}, problem~5.

\item [2.] See this module's \textit{Problem Supplement}, problem~6.
\end{one-digit-list}

}% /Sect
