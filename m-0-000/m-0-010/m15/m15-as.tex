\revhist{8/8/91, pss; 3/11/93, pss; 5/9/94, pss; 11/7/97, pss; 7/15/99, abs; 4/23/02, pss;
         5/1/02, pss}

\Sect{}{}{\SectType{SpecialAssistance}}{

\enlargethispage{2cm}

\AsItem{1}{PS-II-problem~2b}
{Note: answers are in [S-6].
\begin{itemize}
\item [1.] If \m{\vect{p}_k} is the momentum of particle \m{k}, then the total
           momentum \vect{P} is given by: \vect{P} = \linefill{1.0in} .
%
\item [2.] Differentiating with respect to time gives:
\item [  ] \m{d\vect{P}/dt = \sum_{k=1}^N} \linefill{1.0in} .
%
\item [3.] According to Newton's second law, \m{d\vect{p}_k/dt = \vect{F}_{R,k}}
           where \m{\vect{F}_{R,k}} is \linefill{1.0in} .
%
\item [4.] So \m{d\vect{P}/dt} may be expressed in terms of the \m{\vect{F}_{R,k}}
           as:
\item [  ] \m{d\vect{P}/dt = \vect{F}_{R,1} + \linefill{1.0in} = \sum_{k=1}^N \linefill{1.0in}}
%
\item [5.] Since the system is isolated, each particle interacts with only the
other \m{N-1} particles.
Thus,
\item [  ] \m{\vect{F}_{R,1} = \vect{F}_{2 \text{ on } 1} + \vect{F}_{3 \text{ on } 1} +
                    \ldots + \vect{F}_{N \text{ on } 1}}
\item [  ] \m{\vect{F}_{R,2} = \vect{F}_{1 \text{ on } 2} + \vect{F}_{3 \text{ on } 2} +
                    \ldots + \vect{F}_{N \text{ on } 2}}
\item [  ] \m{\ldots}
%\item [  ] \m{\vect{F}_{R,N-1} = \vect{F}_{1 \text{ on } N-1} + \vect{F}_{3 \text{ on } N-1} +
%                    \ldots + \vect{F}_{N \text{ on } N-1}}
\item [  ] \m{\vect{F}_{R,N} = \vect{F}_{1 \text{ on } N} + \vect{F}_{3 \text{ on } N} +
                    \ldots + \vect{F}_{N-1 \text{ on } N}}
\item [  ] where \m{\vect{F}_{j,k}} is \linefill{1.0in} .
%
\item [6.] \m{d\vect{P}/dt = \vect{F}_{2 \text{ on } 1} + \vect{F}_{3 \text{ on } 1} +
                   \ldots + \vect{F}_{N \text{ on } 1}}
\item [  ] \m{\phantom{d\vect{P}/dt} = +\vect{F}_{1 \text{ on } 2} + \vect{F}_{3 \text{ on } 2} +
                   \ldots + \vect{F}_{N \text{ on } 2} +\ldots}
\item [  ] \m{\phantom{d\vect{P}/dt =} + \vect{F}_{1 \text{ on } N} +
           \vect{F}_{2 \text{ on } N} + \ldots + \vect{F}_{N-1 \text{ on } N}}
%
For each force \m{\vect{F}_{j \text{ on } k}} in the sum above, there is a force
\m{\vect{F}_{k \text{ on } j}} which appears once and only once.
If the sum on the right side of this equation is regrouped so that these
paired forces are added (i.e., \m{\vect{F}_{j \text{ on } k}} +
\m{\vect{F}_{k \text{ on } j}}), then each pair sums to {\linefill{1.0in}} because of Newton's
{\linefill{1.0in}} law.
%
\item [7.] So \m{d\vect{P}/dt =} \linefill{1.0in} and \vect{P} is {\linefill{1.0in}}\,.
\end{itemize}
}

\AsItem{2}{PS-II-problem~5}
{You are probably subtracting the momentum vectors incorrectly.
 Here is a right way:
 \begin{one-digit-list}
 \item [1.] Calculate \m{\vect{p}_i} and \m{\vect{p}_f}.
 \item [2.] Write: \m{\Delta \vect{p} = \vect{p}_f - \vect{p}_i =
            \vect{p}_f + ( - \vect{p}_i )}.
 \item [3.] Draw \m{\vect{p}_f} and label it.
 \item [4.] At the head of \m{\vect{p}_f}, start the tail of \m{( - \vect{p}_i)}.
 \item [5.] Draw the rest of the vector \m{( - \vect{p}_i)} (which is just
            \m{\vect{p}_i} with the direction reversed) and label it.
 \item [6.] Draw a vector connecting the tail of \m{\vect{p}_f} to the head
            of \m{( - \vect{p}_i)}.
 \item [7.] Label this new vector: \m{\Delta \vect{p} =
            \vect{p}_f + ( - \vect{p}_i ) = \vect{p}_f - \vect{p}_i}.
 \item [8.] On the sketch, break \m{\Delta \vect{p}} into its (approximate)
            \m{x}- and \m{y}-components.
            Note the sizes and signs of the two components.
 \end{one-digit-list}
 Now calculate the \m{x}- and \m{y}-components of \m{\Delta \vect{p}}, noting that
 getting the contribution of \m{p_{i,y}} involves taking the negative of a
 negative number.
 \help{11}
}

\AsItem{3}{PS-II-problem~8}
{\begin{one-digit-list}
 \item [1.] Did you properly convert each of the given weights to masses before
            putting them into the expression for momentum?
 \item [2.] Did you calculate the total momentum, before the collision, using
            vector addition (getting its magnitude by taking the square root of
            the sum of the squares of the \m{x}- and \m{y}-components)?
 \item [3.] Did you set the \Quote{after} momentum equal to the \Quote{before} momentum?
 \item [4.] Did you set the \Quote{after} momentum equal to \m{mv}?
 \item [5.] Did you remember that the total mass after the collision is just
            the sum of the masses of the two?
 \end{one-digit-list}
 \help{8}
}

\AsItem{4}{PS-II-problem~10}
{Solving this problem symbolically involves a lot of algebra.
 We suggest solving this problem numerically.
 To do this, calculate the numerical magnitude of the {\em momentum} of each of
 the outgoing particles separately and draw the initial momentum and final
 momenta, roughly to scale, on a sketch of the scattering.
 Voila!
 For this special case the rest of the solution takes only a few lines.
}

\AsItem{5}{PS-II-problem~11}
{Apply conservation of momentum separately to the \m{x}- and \m{y}-components of
 momentum, leaving symbols for the (unknown) \m{x}- and \m{y}-components of the
 momentum of the larger mass after the collision.
 That gives you two equations.
 Solve for the two components, combine them to form a vector, and find the
 magnitude of the vector.
 Finally, divide that by the mass to get the speed.
 Get the angle from the ratio of the components. \help{17}
}

\AsItem{6}{[S-1]}
{{\bf Answers}:

\begin{itemize}
\item [1.] \m{\vect{p}_1 + \vect{p}_2 + \ldots + \vect{p}_N =
           \sum_{k=1}^N \vect{p}_k}

\item [2.] \m{d\vect{p}_k/dt}.
\item [3.] The resultant force on particle \m{k}.

\item [4.] \m{d\vect{P}/dt = \vect{F}_{R,1} + \vect{F}_{R,2} + \ldots +
           \vect{F}_{R,N} = \sum_{k=1}^N \vect{F}_{R,k}}.

\item [5.] The force particle \m{j} exerts on particle \m{k}.

\item [6.] Zero, third.

\item [7.] 0, constant (or conserved).
\end{itemize}
}

\AsItem{7}{PS-II-problem~12}
{Note that an integral between two points is the area under the curve
 between those same two points.
 \help{18}
}

\AsItem{8}{[S-3]}
{The \m{x}- and \m{y}-components of momentum are conserved separately. \help{15}
}

\AsItem{9}{PS-II-problem~14}
{The unit \Quote{lb} is a unit of weight, not mass.
}

\AsItem{10}{PS-II-problem~15}
{\m{Mv = M(0.8v)\cos 30\degrees + 2M(u)\cos\theta} \newline
 \m{0 = M(0.8v)\sin 30\degrees - 2M(u)\sin\theta} \newline
 Solve for \m{\theta}. \help{12}
}

\AsItem{11}{[S-2]}
{\m{\vect{p}_i = \left( \dfrac{0.64\unit{lb}}{32\unit{ft/s\up{2}}} \right)
            \left[ (30\unit{ft/s})\cos 45\degrees\,\uvec{x}
                  -(30\unit{ft/s})\sin 45\degrees\,\uvec{y} \right]}\newline
 That's enough help!
}

\AsItem{12}{[S-10]}
{Divide all terms by (\m{Mu}), then replace (\m{v/u}) by the symbol \m{a}
 (an arbitrary symbol).
 That leaves two equations in two unknowns (\m{a} and \m{\theta}).
 Solve for \m{\theta}.
 To an unwarranted number of digits, \m{\theta = 52.5\degrees}.
 \furtherhelp{21}
}

\AsItem{13}{PS-II-problem~5}
{If you claim \Quote{I don't know how to get the average force,} then you did
 not pay attention to this module's text or even bother to skim over it
 to find the relevant material.
}

\AsItem{14}{PS-II-problem~5}
{Use Newton's third law.
}

\AsItem{15}{[S-8]}
{\m{\vect{p}_s + \vect{p}_w =
     \left( \dfrac{150\unit{lb}}{32\unit{ft/s\up{2}}} \right)
                                       (5.0\unit{ft/s})\,\uvec{x} +
     \left( \dfrac{100\unit{lb}}{32\unit{ft/s\up{2}}} \right)
                                       (4.0\unit{ft/s})\,\uvec{y}} \newline
 \m{\vect{p}_{w+s} = \left( \dfrac{250\unit{lb}}{32\unit{ft/s\up{2}}} \right)
                      (v_x\,\uvec{x} + v_y\,\uvec{y})} \newline
 Now solve for \m{v_x} and \m{v_y}.
}

\AsItem{16}{AS-II-problem~9}
{Do problems~1-8 successfully before attempting this one.
}

\AsItem{17}{[S-5]}
{\m{Mv = M(v/2)\cos 120\degrees + 3.0MV\cos\theta} \newline
 \m{0 =  M(v/2)\sin 120\degrees + 3.0MV\sin\theta} \newline
 Solve for \m{\theta}. \help{19}
}

\AsItem{18}{[S-7]}
{\m{\Delta v_x = \dfrac{1}{m} (400\unit{N})(0.02\unit{s}) =
                      1.6\unit{N\,s}}\newline
 \m{\Delta v_y = \dfrac{1}{m} (400\unit{N})(0.015\unit{s}) =
                      1.2\unit{N\,s}}
}

\AsItem{19}{[S-17]}
{Divide all terms by (\m{Mv}), then replace (\m{V/v}) by the symbol \m{a}
 (an arbitrary symbol).
 That leaves two equations in two unknowns (\m{a} and \m{\theta}).
 Solve for \m{\theta}.
}

\AsItem{20}{PS-II-problem~14}
{\m{\vect{F}_{av} = \left( \dfrac{1}{.40\unit{s}} \right)
           \left( \dfrac{160\unit{lb}}{32\unit{ft/s\up{2}}} \right)
           \left[ (23\unit{ft/s})(\cos 34\degrees \,\uvec{y} +
                                  \sin 34\degrees \,\uvec{x})
                 -(-11\unit{ft/s})\,\uvec{y} \right]} \newline
 \m{\phantom{F_{av}} = (376\,\uvec{y} + 160\,\uvec{x})\unit{lb}}
}

\AsItem{21}{[S-12]}
{Solving two equations that each contain the same two unknowns:
 \begin{itemize}
 \item [1.] Solve one of the equations for either of the unknowns
 (pick one).  Call it unknown~\#1.
 \item [2.] Substitute that solution for unknown~\#1 into the other
 equation and solve for the other unknown. Call it unknown~\#2.
 Notice that the solution you just found for unknown~\#2 does not contain
 unknown~\#1.
 \item [3.] Now substitute the solution for unknown~\#2 back into
 the solution for unknown~\#1.
 You now have solutions for both of the unknowns.
 \end{itemize}
}

}% /Sect
    