\revhist{12/17/90, pss; 8/9/91, pss; 3/12/93, pss; 5/9/94, pss; 0/25/96, pss;
         11/7/97, pss; 11/13/97, pss; 1/5/01, pss; 10/26/01, pss; 4/23/02, pss; 5/1/02, pss;
         5/13/02, pss}

\Sect{}{}{\SectType{ProblemSet}}{

This supplement has two distinct parts.

Part I contains four problems, each with a problem statement followed by
a set of questions about the problem.
Each set of questions is followed by a set of answers.
Work through the set of questions for a problem, then compare your answers
to the ones given.
If they agree, great!
If they don't, make sure that you reconcile them.

Part II of this supplement contains problems with less detailed
questions and answers.

{\bf PART I}\newline
A. One-dimensional Collision-Force Calculation \dotfill PS1 \newline
B. Two-dimensional Collision-Force Calculation \dotfill PS2 \newline
C. One-dimensional Conservation of Momentum Problem \dotfill PS3 \newline
D. Two-dimensional Conservation of Momentum Problem \dotfill PS5 \newline

{\bf PART II}\newline
Problems~1-12 \dotfill PS8 \newline

\ProbSet{PART I}{
%
\xpcap{}{A}{One-dimensional Collision-Force Calculation}\newline
Problem: Estimate the average resultant force on a paratrooper (\m{\text{mass } = 79\unit{kg}})
who hits the ground falling vertically at a speed of 11\unit{m/s} (25\unit{mi/hr}).
Assume that he is a super-trooper and does not fall when he lands, but just
lets his knees bend to increase the collision time.

\begin{one-digit-list}
\item [1.] The average resultant force on the trooper is
           \m{\Delta\vect{p}/\Delta t}, where \m{\Delta\vect{p}} is his
           change in momentum while stopping and \m{\Delta t} is \OneInchAnswer . \answer{1}

\item [2.] \m{\Delta \vect{p}} = \OneInchAnswer . \answer{2}

\item [3.] Give a rough estimate for \m{\Delta t}.
           \m{\Delta t\,\simeq}\,\OneInchAnswer . \answer{3}

\item [4.] \m{\vect{F}_{R,av}} \OneInchAnswer . \answer{4}

\item [5.] How does this force compare to his weight?
           \m{F/mg} = \OneInchAnswer . \answer{5}
\end{one-digit-list}

\xpcap{}{B}{Two-dimensional Collision-Force Calculation}\newline
\TextAndFigure{Problem: A 3200\unit{lb} car sliding on ice at 41\unit{mi/hr}
(60.1\unit{ft/s}) hits a concrete wall and bounces off, as shown, with a speed of 
22\unit{mi/hr} (32.3\unit{ft/s}).  If the collision time is measured from high speed 
movies to be 0.50\unit{s}, determine the average force (magnitude and direction) 
on the car by the wall.}{m15gr06}

\begin{one-digit-list}
\item [1.] Since \m{\sum \vect{F}_{av} = \Delta \vect{p} / \Delta t}, you must
           determine \m{\Delta\vect{p}}.
           To do so, draw a vector triangle showing the car's initial momentum,
           final momentum, and change in momentum. \answer{6}

\item [2.] Determine the magnitude of the momentum change.
           Hint: Pay close attention to the vector triangle just constructed.
           \m{|\Delta\vect{p}|} = \OneInchAnswer . \answer{7}

\item [3.] Determine the angle between \m{\Delta\vect{p}} and the normal to the
           wall.
           Angle = \OneInchAnswer . \answer{8}

\item [4.] Determine the magnitude of the average force on the car by the wall.
           \m{|\sum\vect{F}_{av}|} = \OneInchAnswer . \answer{9}

\item [5.] Show, on the diagram, the average force on the wall by the car.
           Indicate its magnitude and direction. \answer{10}
\end{one-digit-list}

\xpcap{}{C}{One-dimensional Conservation of Momentum Problem}\newline
Problem: A mass \m{m} moving with a speed \m{w} to the right collides head-on with
a mass \m{M} moving with a speed \m{V} to the left.
After the collision, the two masses remain together.
Determine the velocity \m{u} of the composite mass in terms of \m{m}, \m{M}, \m{w},
\m{V}, and \m{\uvec{x}}, a unit vector to the right.

\begin{one-digit-list}
\item [1.] Read the problem carefully.
           Draw before and after sketches that depict the problem.
           Label the various masses and velocities with appropriate symbols. \answer{11}

\item [2.] Is it appropriate to assume the momentum of the two-particle system
           to be conserved during the collision?
           What assumptions, if any, must be made so that you can use momentum
           conservation? \answer{12}

\item [3.] Assuming momentum to be conserved, write a vector expression which
           relates \m{m}, \m{M}, \m{w}, \m{V}, \m{\uvec{x}}, and \vect{u}. \answer{13}

\item [4.] Solve for \vect{u}.
           What can you say about the direction for \vect{u}?
           \vect{u} = \OneInchAnswer . \answer{14}

\item [5.] If the magnitude of the momentum of \m{m} before the collision is \OneInchAnswer
           than that of \m{M}, the composite particle moves to the right.
           If the two initial momenta are \OneInchAnswer, the composite particle remains
           at rest; and if the initial momentum of \m{m} is \OneInchAnswer that of \m{M}, the
           final motion is to the left. \answer{15}

\item [6.] Consider the special case for \m{M} initially at rest; i.e., \m{V} = 0.
           Show how a measurement of the initial and final speeds (w,u) could
           be used to determine the ratio of the two masses. \answer{16}

\item [7.] For the mass determination just discussed, suppose the data shows
           that \m{u \ll w}.
           Which inequality is appropriate: \m{M \ll m} or \m{M \gg m}? \answer{17}

\item [8.] If you did not do so already, write a brief discussion explaining
           these results by considering the known results of collisions
           between small and large objects. \answer{18}
\end{one-digit-list}

\xpcap{}{D}{Two-dimensional Conservation of Momentum Principle}\newline
\TextAndFigure{Problem: A mass \m{m} moving with a speed \m{u} collides with a mass
\m{M} at rest.
After the collision, each moves as shown.
Determine the speeds after the collision, if \m{m = 0.15\unit{kg}}, \m{M = 0.20\unit{kg}},
\m{\theta = 39\degrees}, \m{\phi = 74\degrees}, \m{u = 20.0\unit{m/s}}.}{m15gr11}

\begin{two-digit-list}
\item [1.]  Read the problem carefully.
            Consider the two particle system consisting of \m{m} and \m{M}.
            Is this an isolated system?
            Why? \answer{19}

\item [2.]  If you use momentum conservation to relate the known and unknown
            quantities, what assumptions about the system are you making? \answer{20}

\item [3.]  Under what condition is this assumption valid, even if the system
            is not isolated? \answer{21}

\item [4.]  Construct a vector diagram illustrating the momentum conservation
            equation: \m{m \vect{u} = m \vect{w} + M \vect{V}}. \answer{22}

\item [5.]  \ItemFigure{The task now is to use this vector diagram to solve
            for \m{w} and \m{V} in terms of the known quantities \m{m}, \m{M}, \m{u},
            \m{\theta}, \m{\phi}.
            Solutions are easily obtained by using the law of sines.
            Write out this relationship for the triangle shown. \answer{23}}{m15gr12}

\item [6.]  Now use this result to solve for \m{w}.
            Hint: Recall that \m{\sin(\pi - \alpha) = \sin\alpha} for any value
            of \m{\alpha}. \answer{24}

\item [7.]  Solve for \m{V}. \answer{25}

\item [8.]  \ItemFigure{You could have solved for \m{w} and \m{V} by writing
            down conservation of momentum equations.
            Using the momentum conservation triangle, write out the two
            corresponding scalar equations:}{m15gr13}

            components parallel to initial momentum: \m{m u = }\,?

            components perpendicular to initial momentum: \m{0 = }\,? \answer{26}

\item [9.]  Use the second equation to solve for MV in terms of m, w, \m{\theta},
            \m{\phi}.
            Substitute this into the first, and solve the resulting equation
            for \m{w}.
            Your answer will simplify if you use the trig identity:

            \m{\sin(\theta + \phi) = \sin\theta\,\cos\phi + \cos\theta\,\sin\phi}. \answer{27}

\item [10.] Now complete the solution by substituting the expression for \m{w}
            into the equation for \m{V}. \answer{28}

\item [11.] Note that the expressions for \m{w} and \m{V} obtained by two alternate
            means agrees.
            Now determine the values for \m{w} and \m{V}. \answer{29}

\item [12.] \ItemFigure{Notice that the vector triangle drawn to display
            conservation of momentum graphically was very important in solving
            this problem.
            In these types of problems, you will often find that the law of
            sines or law of cosines from trigonometry is a time-saver over the
            component method.}{m15gr14}

            These trigonometric relations are summarized in the figure and
            the following equations.
            The laws are applied to each vertex in turn.

\begin{center}\begin{tabular}{c p{0.4in} c}
{\bf Law of Sines} & & {\bf Law of Cosines} \\
\m{\dfrac{A}{\sin\alpha} = \dfrac{B}{\sin\beta} =
\dfrac{C}{\sin\gamma}} & & \m{C^2 = A^2 + B^2 - 2 A B \cos\gamma} \\
 & & \m{B^2 = A^2 + C^2 - 2 A C \cos\beta} \\
 & & \m{A^2 = B^2 + C^2 - 2 B C \cos\alpha} \\
\end{tabular}\end{center}
\end{two-digit-list}
}% /ProbSet

\ProbSet{PART II}{

\newlength{\tempval}
\setlength{\tempval}{\textwidth}
\addtolength{\tempval}{-1.0cm}
\begin{center}\fbox{\parbox{\tempval}{\bf If you get stuck on any problem in
this part, you must go back and work the problems in Part I of this
Supplement successfully before continuing.}}\end{center}

Note: Items 13-15 also occur in this module's \textit{Model Exam}.

\begin{two-digit-list}
\item [1.] \NullItem
\begin{one-digit-list}
\item [a.] Define the momentum of a system of particles.
\item [b.] Give the SI and English units of momentum.
\item [c.] Express momentum units in terms of force and time units (SI and
           English). \answer{30}
\end{one-digit-list}

\item [2.] \NullItem
\begin{one-digit-list}
\item [a.] Consider an isolated three-particle system.
           Use Newton's laws to show that the total momentum for this system
           does not change with time.
\item [b.] Extend the argument of part (a) to an isolated \m{N}-particle
           system. \answer{31} \help{1}
\end{one-digit-list}

\item [3.] The only external forces on a two-particle system act in the
           \m{y}-direction.
           Show that although this system is not isolated, the \m{\uvec{x}} and
           \m{\uvec{z}} components of the system momentum are constant in time. \answer{32}

\item [4.] Explain why a baseball player often moves his gloved hand in the
           direction of motion of the ball as he catches it. \answer{33}

\item [5.] \ItemFigure{A spinning rubber ball (weight = 0.64\unit{lb}) bounces
           from the floor as shown.
           The collision time is measured to be 0.010\unit{s}.
           Let \m{\uvec{x}} and \m{\uvec{y}} be horizontal and vertical unit vectors
           as shown.}{m15gr17}

\begin{one-digit-list}
\item [a.] Calculate the ball's change in momentum during the collision, in
           terms of \m{\uvec{x}} and \m{\uvec{y}}. \help{2}
\item [b.] Determine the average resultant force on the ball during the
           collision.
           Express in terms of \m{\uvec{x}} and \m{\uvec{y}}. \help{13}
\item [c.] What objects exert force on the ball during the collision?
\item [d.] What is the average force on the floor during the collision?
           Express in terms of \m{\uvec{x}} and \m{\uvec{y}}. \help{14}
\item [e.] Is the floor frictionless?
           Explain. \answer{34}
\end{one-digit-list}

\item [6.] Explain why a collision can be analyzed using momentum conservation
           even though the colliding objects do not constitute an isolated
           system. \answer{35}

\item [7.] \ItemFigure{Two masses which repel each other are released at
           \m{t = 0} from rest on a horizontal frictionless surface.
           When the two are sufficiently separated, their interaction is
           negligible, and each is observed to move with a constant velocity.
           Let \vect{w} be this constant velocity observed for \m{m}, and
           \vect{V} the corresponding velocity for \m{M}.}{m15gr18}

\begin{one-digit-list}
\item [a.] What is the total momentum for this two mass system at \m{t} = 0?
\item [b.] Does this total momentum change for \m{t > 0}?
           Why?
\item [c.] Express the total momentum in terms of \m{m}, \m{M}, \vect{w}, and
           \vect{V}.
\item [d.] What can you deduce about the direction of \vect{w} and \vect{V}?
           Explain.
\item [e.] Suppose the mass \m{m} is known and the velocities \vect{w} and
           \vect{V} are measured.
           Express \m{M} in terms of \m{m}, \m{w}, and \m{V}. \answer{36}
\end{one-digit-list}

\item [8.] \ItemFigure{Showboat Steve is walking at 5.0\unit{ft/s} while
           Winsome Wylene's speed as she approaches him is 4.0\unit{ft/s}.
           Just before Showboat would have accelerated to pass in front of
           Winsome, they reach an icy (frictionless) patch on the walkway and
           suffer their inevitable collision.
           Holding desperately onto each other, they slide off together,
           hopefully to happier times.}{m15gr19}

           If Showboat weighs 150\unit{lb} and Winsome 100.0\unit{lb}, determine their
           blissful velocity (magnitude and direction) after the collision.
           \answer{37} \help{3}

\item [9.] A child (mass = 40.0\unit{kg}) standing at rest on a smooth frozen lake
           throws a 0.60\unit{kg} ball toward a friend directly to the east.
           The initial speed of the ball is 10.0\unit{m/s}.
\begin{one-digit-list}
\item [a.] Neglecting air resistance, what external forces act on the
           child-ball system during and after the throw?
\item [b.] If the ball is moving horizontally just after release, determine the
           child's velocity (magnitude and direction) after release.
\item [c.] If the ball's velocity immediately after release is {36.87\degrees}
           above horizontal, calculate the child's speed after release. \answer{38}
\end{one-digit-list}

\item [10.] A 2.0\unit{kg} mass moving with a velocity of 15\unit{m/s}, east,
            collides with a 1.0\unit{kg} mass initially at rest.
            After the collision, the larger mass has a speed of 8.5\unit{m/s} and
            the smaller is observed to be moving with a speed of 17.0\unit{m/s}.
            Calculate the scattering angle for the larger mass; i.e., the angle between
            its final and initial velocities. \answer{39} \help{4}

\item [11.] A mass \m{M} moving with a speed \m{V} in a particular direction collides with
            a larger object, of mass \m{3.0\,M}, that is initially at rest.
            After the collision the smaller mass is observed to be moving in a direction that
            makes an angle of {120\degrees} to its original direction, and it is now moving
            with a speed \m{V}/(2.0).
            Determine the speed and direction of motion of the larger mass after the collision.
            Specify the direction of motion by giving the angle it makes to the pre-collision direction
            of motion of the smaller mass. \answer{40} \help{5}

\item [12.] A 5.0\unit{kg} mass moving with a velocity of
            \m{(3.2 \uvec{x} - 2.4 \uvec{y})\unit{m/s}} at \m{t = 0} is acted upon by a
            force \m{\uvec{x} F_x(t) + \uvec{y} F_y(t)} where \m{F_x(t)} and \m{F_y(t)}
            are shown in the graphs below.
            Determine the object's velocity at \m{t = 0.030\unit{s}}. \answer{41}

            Hint: Recall that \m{\Delta\vect{p} = \vect{F}_{R,av} \Delta t =
            \int_0^{\Delta t} \vect{F}_R(t)\,dt}.

            \CenteredUnframedFixedFigure{m15gr20}\newline
            \help{7}

\item [13.] At an instant when a 0.20\unit{kg} particle has position, velocity, and
            acceleration given by \m{(2.1 \uvec{x} - 4.8 \uvec{y})\unit{m}},
            \m{( 1.2 \uvec{x} + 1.6 \uvec{z})\unit{m/s}}, and
            \m{(-2.0 \uvec{y} + 1.5 \uvec{z})\unit{m/s\up{2}}}:
\begin{one-digit-list}
\item [a.] Calculate its momentum.
\item [b.] Calculate the rate at which its momentum is changing. \answer{42}
\end{one-digit-list}

\item [14.] \ItemFigure{A diving competitor (weight = 160\unit{lb}) has a
            downward velocity of 11\unit{ft/s} just before hitting the board.
            When contact with the board ceases, 0.40\unit{s} later, the diver's
            velocity is 23\unit{ft/s} at an angle of {34\degrees} with the vertical.

            Calculate the magnitude of the average resultant force on the diver
            while in contact with the board. \answer{43} \help{20}}{m15gr21}

\item [15.] \ItemFigure{A mass \m{M} moving with a speed \m{v} collides with
            a mass \m{2 M} initially at rest.
            After the collision the two move as shown.
            Determine \m{\theta}. \answer{44} \help{10}}{m15gr22}
           
\end{two-digit-list}

\xpcap{}{}{Answers}
%
\begin{two-digit-list}
\item [1.] The time elapsing between the instant of contact with the ground and
the instant his downward motion stops.

\item [2.] \m{m(\vect{v}_\text{final} - \vect{v}_\text{initial}) =
           (79\unit{kg}) \cdot (0\unit{m/s} - 11\unit{m/s}\text{, down}) =
           869\unit{kg\,m/s}\text{, up}}.

\item [3.] His average speed while stopping will be approximately one-half
           his initial speed; i.e., \m{v_{av} \simeq 5.5\unit{m/s}}.
           Hence his stopping time will be approximately
           \m{\Delta t \simeq x/v_{av}}, where \m{x} is the distance his \Quote{body}
           travels during the deceleration.
           In a full-knee bend, a body might fall approximately 0.8\unit{m}, so
           \m{\Delta t \simeq (0.8\unit{m})/(5.5\unit{m/s}) \simeq 0.1\unit{s}}.

\item [4.] \m{\Delta \vect{p} /\Delta t \simeq
           (869\unit{kg\,m/s}\text{, up})/(0.1\unit{s}) = 9 \times 10^3\unit{N}\text{, up}}.

\item [5.] \m{9000\unit{N}/(79\unit{kg} \times 9.8\unit{m/s\up{2}}) \simeq 10}.

\item [6.] \ItemFigure{\m{m = w/g = 100\unit{slugs}}

           \m{p_i = mv_i = 6010\unit{lb\,s}}

           \m{p_f = mv_f = 3230\unit{lb\,s}}}{m15gr07}

\item [7.] Since the angle between \m{\vect{p}_i} and \m{\vect{p}_f} is {90\degrees},
           the Pythagorean theorem gives

           \m{|\Delta \vect{p}| = \sqrt{p_i^2 + p_f^2 } = 6.8 \times 10^3 \unit{lb\,s}}

\item [8.] \ItemFigure{\m{\alpha = 60\degrees - \theta}

           \m{\tan\theta = p_f/p_i = 0.537}

           \m{\theta = 28\degrees}

           \m{\alpha = 32\degrees}}{m15gr08}

\item [9.] \m{\dfrac{|\Delta \vect{p}|}{\Delta t} =
           6823\unit{lb\,s}/0.50\unit{s} = 1.4\times10^4\unit{lb}}

\item [10.] \noindent\CenteredUnframedFixedFigure{m15gr09}

\item [11.] \noindent\newline\CenteredUnframedFixedFigure{m15gr10}
           Notice that in the sketch we have presumed the composite particle to
           be moving to the right after the collision.
           Can you, in fact, determine the final direction of motion from what
           you are given?
           You will see shortly.

\item [12.] Since the stated problem makes no comment as to whether the
           two-particle system is isolated, you have no assurance that the
           system momentum is not changing.
           However, if we assume the time of the collision to be sufficiently
           short, then any momentum change by external forces will be
           negligible; and momentum conservation can then be used to relate
           momenta immediately before and after the collision.
           In any laboratory situation, the validity of such an assumption
           must always be considered carefully.

\item [13.] \m{m w \uvec{x} - M V \uvec{x} = (m + M) \vect{u}}.

           Did you miss the sign on the \m{MV} term?

           Don't forget it's moving in the negative \m{\uvec{x}} direction.

\item [14.] \m{\left(m w - M V/ m + M\right) \uvec{x}}.

           If \m{m w > M V}, \vect{u} is in the \m{\uvec{x}} direction.

           If \m{m w = M V}, \vect{u} is zero.

           If \m{m w < M V}, \vect{u} is in the negative \m{\uvec{x}} direction.

\item [15.] greater than, equal in magnitude, less than.

\item [16.] If \m{V = 0}, \m{m w = (m + M) u}; or \m{m(w - u) = M u} and
           \m{M/m = (w - u)/u}.

\item [17.] \m{\dfrac{M}{m} =
           \dfrac{w - u}{u} \simeq \dfrac{w}{u} \gg 1};
           or \m{M \gg m}?

\item [18.] If the two move off very slowly after the collision (\m{U \ll w}),
           then \m{m} must have collided with a much larger mass; e.g., a ping
           pong ball colliding with a locomotive at rest imparts verrrry little
           speed to the locomotive \m{\ldots} even on frictionless tracks!
           On the other hand, if an at-rest mass M were much smaller than an
           incoming mass \m{m}, the collision would have little effect on the
           speed of \m{m}.
           (Now you give the appropriate ping pong ball-locomotive analogy).

\item [19.] You cannot tell from the statement of the problem.

\item [20.] The system is isolated; i.e., the only force on \m{m} is that by \m{M},
           and vice versa.

\item [21.] If the collision time is sufficiently short, then external forces
           will not cause a measurable change in the system momentum during the
           collision.

\item [22.] \noindent\CenteredUnframedFixedFigure{m15gr15}\newline

\item [23.] \ItemFigure{\m{\dfrac{A}{\sin\alpha} = \dfrac{B}{\sin\beta} =
           \dfrac{C}{\sin\gamma}}}{m15gr16}

\item [24.] \m{\dfrac{mw}{\sin\phi} = \dfrac{mu}{\sin\gamma}}

           \m{w = \dfrac{\sin\phi}{\sin\gamma} u}

           But \m{\gamma = \pi - (\theta + \phi)}

           and \m{\sin(\pi - \theta - \phi) = \sin(\theta + \phi)},

           so \m{w = \dfrac{\sin\phi}{\sin(\theta + \phi)} u}.

\item [25.] \m{V = \dfrac{m u \sin\theta}{M \sin(\theta + \phi)}}

\item [26.] \m{m u = m w \cos\theta + M V \cos\phi}

           \m{0 = m w \sin\theta - M V \sin\phi}

\item [27.] \m{M V = \dfrac{m w \sin\theta}{\sin\phi}}

           \m{m u = m w \left[\cos\theta + \dfrac{\cos\phi\,\sin\theta}{\sin\phi}\right] =
           m w \left[\dfrac{\sin\phi\,\cos\theta + \cos\phi\,\sin\theta}{\sin\phi}\right]}

           \m{m u = m w \dfrac{\sin(\phi + \theta)}{\sin\phi}}

           \m{w = \dfrac{\sin\phi}{\sin(\phi + \theta)}\,u}

\item [28.] \m{M V = \dfrac{m \sin\theta}{\sin\phi}
            \dfrac{u \sin\phi}{\sin(\theta + \phi)} =
            \dfrac{m u\sin\theta}{\sin(\theta + \phi)}}

            \m{V = \dfrac{m u \sin\theta}{M\sin(\theta + \phi)}}

\item [29.] \m{w = 21\unit{m/s}}; \m{V = 1.0\times10^1\unit{m/s}}.

\item [30.] \NullItem
\begin{one-digit-list}
\item [a.] \m{\vect{P} = \sum_k m_k \vect{v}_k} where
\item [  ] \vect{P} = momentum of system

           \m{m_k} = mass of particle \m{k}\newline

           \m{\vect{v}_k} = velocity of particle \m{k}
\item [b.] \unit{kg\,m/s}; \unit{slug\,ft/s}
\item [c.] \unit{N\,s} ; \unit{lb\,s}
\end{one-digit-list}

\item [31.] \NullItem
\begin{one-digit-list}
\item [a.] \m{\vect{P} \equiv \vect{p}_1 + \vect{p}_2 + \vect{p}_3}
\item [] \m{d\vect{P}/dt = d\vect{p}_1/dt + d\vect{p}_2/dt + d\vect{p}_3/dt}

         \m{\phantom{d\vect{P}/dt} = \vect{F}_{R1} + \vect{F}_{R2} + \vect{F}_R3}

         \m{\phantom{d\vect{P}/dt} = \vect{F}_{2,1} + \vect{F}_{3,1} +
         \vect{F}_{1,2} + \vect{F}_{3,2} + \vect{F}_{1,3} + \vect{F}_{2,3}}

         \m{\phantom{d\vect{P}/dt} = (\vect{F}_{2,1} + \vect{F}_{1,2}) +
         (\vect{F}_{3,1} + \vect{F}_{1,3}) + (\vect{F}_{3,2} + \vect{F}_{2,3})}

         \m{\phantom{d\vect{P}/dt}= 0}.
\item [b.] See this module's \textit{Special Assistance Supplement}, [S-1].
\end{one-digit-list}

\item [32.] \m{\vect{F} = d\vect{P}/dt} may be written in scalar form as:

           \m{F_x = dP_x/dt}, \qquad \m{F_y = dP_y/dt}, \qquad \m{F_z = dP_z/dr}.

           Given \m{F_x = F_z = 0}, then \m{dP_x/dt = dP_z/dt = 0} so
           \m{P_x} and \m{P_z} remain constant.

\item [33.] By letting the gloved hand move with the ball, he increases the
           \Quote{collision time} required to change the ball's momentum to zero,
           thereby reducing the average force on the ball, and hence (by
           Newton's third law) reducing the average force on his hand.

\item [34.] \NullItem
\begin{one-digit-list}
\item [a.] \m{(-0.078 \uvec{x} + 0.62 \uvec{y})\unit{lb\,s}}.
\item [b.] \m{(-7.8 \uvec{x} + 62 \uvec{y})\unit{lb}}.
\item [c.] floor, earth (weight).
\item [d.] \m{(7.8 \uvec{x} - 62 \uvec{y})\unit{lb}}.
\item [e.] No.
           It exerts a horizontal force on the ball.
           A frictionless surface can exert normal forces only.
\end{one-digit-list}

\item [35.] For most collisions the collision time is so small and consequently
           the collision forces so large, that the effect of external forces
           can be neglected.

\item [36.] \NullItem
\begin{one-digit-list}
\item [a.] Zero.
\item [b.] No, the resultant force on the system is zero.
\item [c.] \m{m \vect{w} + M \vect{V} = 0}.
\item [d.] In opposite directions since \m{\vect{V} = - (m/M) \vect{w}}.
\item [e.] \m{M = (w/V)\,m}.
\end{one-digit-list}

\item [37.] \ItemFigure{\m{V = 3.4\unit{ft/s}}.

           \m{\theta = \tan^{-1}\left(\dfrac{1.6}{3.0}\right) = 28\degrees}.}{m15gr23}

\item [38.] \NullItem
\begin{one-digit-list}
\item [a.] Weight (by earth) down on child and ball.
\item [  ] Upward force on child by ice.
\item [b.] 0.15\unit{m/s}, west.
\item [c.] 0.12\unit{m/s}, west.
\end{one-digit-list}

\item [39.] {28\degrees}.

\item [40.] speed = \m{\dfrac{\sqrt{7.0}}{6.0} V = 0.44 V}.

            \m{\theta = \tan^{-1}\dfrac{\sqrt{3.0}}{5.0} = -19\degrees}.

\item [41.] \m{(4.8 \uvec{x} - 1.2 \uvec{y})\unit{m/s}}.

\item [42.] \NullItem
\begin{one-digit-list}
\item [a.] \m{( 0.24 \uvec{x} + 0.32 \uvec{z})\unit{kg\,m/s}}.
\item [b.] \m{(-0.40 \uvec{y} + 0.30 \uvec{z})\unit{kg\,m/s\up{2}}}.
\end{one-digit-list}

\item [43.] \m{4.1\times10^2\unit{lb}}.

\item [44.] \m{(5\times10^1)\degrees}.
\end{two-digit-list}
}% /ProbSet
%
}% /Sect