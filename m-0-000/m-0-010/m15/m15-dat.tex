\revhist{9/9/91, pss; 11/19/91, pss; 3/11/93, pss; 3/29/93, pss; 5/9/94, pss;
         10/24/94, pss; 10/3/95, lae; 8/24/96, pss; 9/25/96, pss; 11/13/97, pss;
         2/22/99, pss; 12/18/2000, pss; 1/5/01, pss; 4/13/02, pss; 4/23/01, pss;
         5/6/02, pss; 5/13/02, pss}
%
\defModTitle{\ph{Momentum:} \ph{Conservation and Transfer}}
\defCtAuthor{James \inits{M.}Tanner, Georgia Institute of Technology}
\defIdAuthor{James M.\,Tanner, School of Physics, Georgia Institute of Technology,
Atlanta, GA}
%
\defIdItems{
    \IdVersEval{5/13/2002}{1}
    \IdHours{1}
    \begin{InputSkills}
    \item [1.]  State Newton's laws of motion and use them in the solution of
    one-particle dynamics problems where all forces are constant \prrqone{0-14}.
    \item [2.]  Use SI and English units in the solution of one-particle dynamics
    problems \prrqone{0-14}.
    \item [3.]  Add vectors graphically and formally \prrqone{0-2}.
    \end{InputSkills}
    %
    \begin{KnowledgeSkills}
    \item [K1.] Define the momentum of a particle and of a system of particles.
    State the SI and English units of momentum and express them in terms of force
    and time units.
    \item [K2.] Using Newton's laws and the definition of momentum, show that the
    momentum of an isolated system of particles does not change with time.
    \item [K3.] Show that conservation of momentum in collisions requires that each
    Cartesian component of momentum is conserved separately.
    \end{KnowledgeSkills}
    %
    \begin{ProblemSolvingSkills}
    \item [S1.] Given an object's mass, its velocity before and after a collision, and
    data relevant to the collision time, calculate the average value of the
    resultant force on the object during the collision; and, given a collision
    time and the average resultant force on an object during the collision,
    calculate the object's change in momentum.
    \item [S2.] Given information about two or more particles' momenta before and after
    a collision, use momentum conservation to determine quantities (speed,
    directions, etc.) related to the collision.
    \end{ProblemSolvingSkills}
    %
}