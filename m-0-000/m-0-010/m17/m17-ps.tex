\revhist{7/17/85, mpm; 12/17/90, pss; 6/2/91, pss; 9/8/94, pss; 3/21/95, pss;
         4/26/95, pss; 11/8/95, pss; 11/13/97, pss; 4/15/02, pss; 4/23/02, pss}

\Sect{}{}{\SectType{ProblemSet}}{
\begin{one-digit-list}
\item [1.] The space station in \textit{2001: A Space Odyssey} is constructed
in the shape of two wheels connected by an axle (see \Figref{4} in this module's \textit{Text} for
a drawing of one of the wheels and part of the connecting axle).
If the \Quote{wheels} are each 300\unit{m} in diameter, what rotational period would
be necessary in order to simulate the familiar force of gravity at the
earth's surface?
\item [2.] As in Problem 1 but for the SKYLAB satellite: what rotational period would be necessary for this
satellite, which is about 16\,m in diameter?
\item [3.] \TextAndFigure{A ball of mass \m{m} is attached to the end of a
string of length \m{\ell}; the other end is tied to the ceiling.
The ball is set into motion in a circular path as shown.
\begin{one-digit-list}
\item [a.] Make a one-body diagram for the ball.
\item [b.] What must its speed be for the string to make a given angle
\m{\theta} with the vertical?
\item [c.] Find the tension in the string.
\end{one-digit-list}}{m17gr05}
\item [4.] You are rounding a turn of radius 0.500\unit{mile} (2,640\unit{ft}) at
175\unit{mi/hr} (257\unit{ft/s}) in the Indy~500.
In the following, neglect effects due to rotation of the earth and air
resistance.
\begin{one-digit-list}
\item [a.] Derive the ideal track banking angle in symbols, then in degrees.
Sketch a clear one-body force diagram showing all forces, plus the
resultant, that act on the car.
\item [b.] Explain why this banking angle is ideal.
\item [c.] Calculate the force of the car on you, in multiples of your body
weight.
\end{one-digit-list}
\item [] The Indy~500 track was actually built in 1909 and banked for
100\unit{mi/hr} (147\unit{ft/s}).
\begin{one-digit-list}
\item [d.] Calculate its actual banking angle.
Sketch a clear one-body force diagram showing the resultant force for this
case (100\unit{mi/hr}).
\item [e.] Sketch a clear one-body force diagram showing the force needed to
keep your 175\unit{mi/hr} car at constant radius, as well as the part of that
force supplied by the horizontal component of the actual track's support.
Also show the sideways force parallel to the track which must be supplied by
static and skidding friction between tires and track.
This is the difference between the above two forces.
\item [f.] Identify the centripetal and centrifugal forces associated with
your body, in the sense of stating what body is acting and what body is
acted on in each case.
\end{one-digit-list}
\item [5.] \TextAndFigure{A ball of mass \m{m} rolls on the inside of the
frictionless circular cone of height \m{h} and base of radius \m{R}.
The axis of the cone is vertical, and the apex points down.
That ball is set into motion in a horizontal circular path of radius \m{r}.
(\m{0\,<\,r\,<\,R})
\begin{one-digit-list}
\item [a.] Draw a one-body diagram for the ball.
\item [b.] What is the required speed \m{v} of the ball?
\end{one-digit-list}}{m17gr06}
\end{one-digit-list}
\BriefAns
\begin{one-digit-list}
\item [1.] \m{T = 2\pi\sqrt{r/g} = \pi\sqrt{2d/g} = 25}\,s; \m{d} = diameter
\item [2.] \m{T = \pi\sqrt{2d/g} = 5.7}\,s\
\item [3.] \NullItem
\begin{one-digit-list}
\item [a.] \CharacterUnframedFigure{m17gr07}
\item [b.] \m{v = \sqrt{\ell\,g\,\sin^2\theta/\cos\theta}} \help{1}
\item [c.] \m{T = m\,g\,\sec\theta} \help{2}
\end{one-digit-list}
\item [4.] \TextAndFigure%
{a. \m{\tan\theta = v^2/(Rg)}
 \begin{one-digit-list}
% \item [a.] \m{\tan\theta = v^2/(Rg)}
 \item [  ] \m{\theta = 38.02\degrees}.
 \item [b.] Track must not exert a sideways force parallel to itself.
            Then there will be no tendency for a car to skid sideways.
 \item [c.] 1.27~times your weight. \help{14}%
 \end{one-digit-list}%
}{m17gr08}
\medskip

\TextAndFigure%
{\begin{one-digit-list}
 \item [d.] \m{\theta_0 = 14.35\degrees}.
 \end{one-digit-list}%
}{m17gr09}
\bigskip

\TextAndFigure%
{\begin{one-digit-list}
 \item [e.] \ 
 \end{one-digit-list}%
}{m17gr10}
\medskip

\begin{one-digit-list}
  \item [f.] car door (or shoulder belt) on you, you on car door (or shoulder
            belt)
 \end{one-digit-list}
%
\item [5.] \NullItem
\begin{one-digit-list}
\item [a.] The force diagram looks similar to that of problem~3a but with
           the force \m{T} replaced by a force normal (perpendicular) to
           the frictionless surface.
\item [b.] \m{\sqrt{rgh/R}} \help{4}
\end{one-digit-list}
\end{one-digit-list}

}% /Sect