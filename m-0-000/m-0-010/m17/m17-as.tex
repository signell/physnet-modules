\revhist{12/17/90, pss; 6/3/91, pss; 9/8/94, pss; 4/26/95, pss; 11/8/95, pss; 12/15/2000, pss}

\Sect{}{}{\SectType{SpecialAssistance}}{

\AsItem{1}{PS-Problem~3b}
{\begin{one-digit-list}
 \item [1.] Determine the actual acceleration the mass undergoes, in terms
 of given and desired quantities.
 Get rid of \m{r} by substituting \m{\ell \sin{\theta}}.
 \item [2.] Multiply the acceleration by the mass to get the force acting
 on the mass.
 \item [3.] Draw the force on the diagram in the problem statement.
 \item [4.] Did you properly point the force toward the center of the circle?
 \item [5.] Write the total force on the mass as \vect{T}, the string's
 (unknown) force, plus \m{m\vect{g}}: these are the only forces acting on the
 mass.
 \item [6.] Set the force deduced from the acceleration equal to the force
 deduced from adding the acting forces.
 This produces a vector equation.
 \item [7.] Rewrite the vector equation as two single-component equations (one
 for the x-components, one for the y-components), each involving \m{\theta}.
 \item [8.] Eliminate \m{T} between the two equations, leaving one equation.
 \item [9.] Solve for \m{v^2}.
 \end{one-digit-list}
}

\AsItem{2}{PS-Problem~3c}{\m{\sec{\theta} \equiv 1 / \cos{\theta}}}

\AsItem{4}{PS-Problem~5b}
{\begin{one-digit-list}
 \item [1.] Completely solve problem~4 before attempting this one.
 \item [2.] The surface of the cone is said to be \Quote{frictionless,} so the cone surface
 only exerts a force normal to itself.  That's the physics.  Now using trigonometry,
 we find that the force of the surface on the ball is at an angle of
 \m{\tan^{-1}{[R/h]}} from the horizontal.
 \end{one-digit-list}
}

\AsItem{7}{TX-2c}
{%
 \Eqn{}{\dfrac{\left[(50\unit{mi/hr})(5280\unit{ft/mi})(1/3600\unit{s/hr})\right]^2}
             {(390\unit{ft})(32\unit{ft/s\up{2}})} = 0.43}
 %
}

\AsItem{8}{TX-3c}
{Taking components as stated:
 \TwoEqns{}{N \sin\theta & = F_r = \dfrac{m v^2}{r}}
           {N \cos\theta & = m g}
 %
}

\AsItem{9}{TX-3d}{Solve \Eqnref{1} for \m{v}.}

\AsItem{10}{TX-4a}
{The equations are exactly the same as in [S-8] but with \m{N} replaced by
 \m{T}, the tension in the string.
}

\AsItem{11}{TX-4b}{The equations are exactly the same as in [S-8].}

\AsItem{12}{TX-3c}
 {\Eqn{}{ \tan^{-1}\theta = \tan^{-1}\left(\dfrac{v^2}{r g}\right) =
         \tan^{-1}0.43 = 23\degrees}
 \help{7}
}

\AsItem{13}{TX-4e}
{The acceleration of the pilot, at that point, is 1\,\m{g} (32\unit{ft/s\up{2}})
 upward.
 This is obviously a \Quote{1\unit{gee}} acceleration.
}

\AsItem{14}{PS-Problem~4c}
{\Eqn{}{\vect{F}_\text{on you} = \vect{F}_\text{earth on you} + 
                                 \vect{F}_\text{car on you} = 
                                 m_\text{you}\,\vect{a}_\text{you}}
 %
 What \Emph{is} your acceleration?
}

}% /Sect
