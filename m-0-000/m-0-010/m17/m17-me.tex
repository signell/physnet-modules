\revhist{7/17/85, mpm; 6/2/91, pss; 3/18/92, pss; 9/8/94, pss}

\Sect{}{}{\SectType{ModelExam}}{

\begin{one-digit-list}
\item [1.] See Exam Skills K1-K2, this module's \textit{ID Sheet}.
The exam may include one or more of these skills, or none.
\item [2.] You are rounding a turn of radius 0.500\unit{mile} (2,640\unit{ft}) at
175\unit{mi/hr} (257\unit{ft/s}) in the Indy 500.
In the following, neglect effects due to rotation of the earth and air
resistance.
\begin{one-digit-list}
\item [a.] Derive the ideal track banking angle in symbols, then in degrees.
Sketch a clear one-body force diagram showing all forces, plus the resultant,
that act on the car.
\item [b.] Explain why this banking angle is ideal.
\item [c.] Calculate the force of the car on you, in multiples of your body
weight.
\item [] The Indy~500 track was actually built in 1909 and banked for 100\unit{mi/hr}
(147\unit{ft/s}).
\item [d.] Calculate its actual banking angle.
Sketch a clear one-body force diagram showing the resultant force for this
case (100\unit{mi/hr}).
\item [e.] Sketch a clear one-body force diagram showing the force needed to
keep your 175\unit{mi/hr} car at constant radius, as well as the part of that force
supplied by the horizontal component of the actual track's support.
Also show the sideways force parallel to the track which must be supplied by
static and skidding friction between tires and track.
This is the difference between the above two forces.
\item [f.] Identify the centripetal and centrifugal forces associated with your
body, in the sense of stating what body is acting and what body is acted on in
each case.
\end{one-digit-list}
\end{one-digit-list}

\BriefAns

\begin{one-digit-list}
\item [1.] See this module's text.
\item [2.] See this module's \textit{Problem Supplement}, Problem~4.
\end{one-digit-list}

}% /Sect
